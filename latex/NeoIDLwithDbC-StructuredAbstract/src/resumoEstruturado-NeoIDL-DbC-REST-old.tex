\documentclass{article}
\usepackage[brazil]{babel} % Suporte para o Portugu�s
\usepackage[utf8]{inputenc} % Suporte para acentua��o sem necessidade dos
\usepackage{blindtext}
\usepackage[inline]{enumitem}

\newcommand{\absdiv}[1]{%
  \par\addvspace{.5\baselineskip}% adjust to suit
  \noindent\textbf{#1}\quad\ignorespaces
}

\begin{document}

{\large {\bf CONTRATOS REST ROBUSTOS E LEVES: UMA ABORDAGEM EM
DESIGN-BY-CONTRACT COM NEOIDL}}

\begin{abstract}

\emph{Contexto.}
A demanda por integração entre sistemas de várias origens e tecnologias diversas
fez aumentar a adoção de soluções baseada em computação orienta a serviços,
sendo o uso de \textit{Serviços Web} a estratégia mais comum para implementar 
serviços, em particular com a adoção crescente do estilo arquitetural REST.
Por outro lado, REST ainda não dispõe de uma notação padrão para especificação
de contratos, e linguangens como Swagger, YAML e WADL cumprem com o
único propósito de descrição de serviços, porém apresentam uma significativa 
limitação: essas linguagens 
são voltadas para computadores, tendo escrita e leitura complexas para
humanos--o que prejudica a abordagem \textit{Contract-first} inerente da 
orientação a serviços. Tal limitação motivou a especificação da 
linguagem NeoIDL,\footnote{Além de ser uma linguagem (\textit{Domain Specif
Language}), a NeoIDL também possui um framework de geração de código para outras linguagens de propósito amplo.} 
que foi concebida
com o objetivo de ser mais expressiva para humanos, além de
prover suporte a modularização e herança.
\emph{Problema.} Entretanto, nenhuma dessas linguagens, incluindo a NeoIDL, dão
suporte a contratos robustos, como os possíveis de serem descritos em
linguagens ou extensões de linguagens com suporte a \emph{Design-by-Contract},
presentes tipicamente (ou exploradas) no paradigma de orientação a objetos.
\emph{Objetivos.}
O objetivo geral deste trabalho é investigar o uso de construções de
\textit{Design-by-Contract} no contexto de computação orientada a serviços,
verificando a viabilidade e utilidade de sua adoção na especificação de
contratos e implementação de serviços REST.
% \absdiv{Método}
% 
% Após ampla revisão bibliográfica, sobretudo dos temas SOC, REST,
% \textit{Design-by-Contract}, a hipótese da aplicabilidade de DbC em
% especificação de contratos REST foi levantada. Com o propósito de validar essa
% hipótese, a sintaxe da NeoIDL foi extendida para suportar DbC e, em seguida,
% implementadas regras de transformação que traduzem as construções de DbC em
% código de validação para o \textit{framework Python
%  Twisted}. Paralelamente, foi conduzida pesquisa junto a desenvolvedores
%  experientes sobre a sua aceitação de especificações de contratos REST com
%  \textit{Design-by-Contract} na NeoIDL. Adicionalmente, foi realizada uma
%  análise empírica sobre a expressividade e reuso da NeoIDL em si.
\emph{Resultados e Contribuições.} 
Essa dissertação contribui com uma extensão da NeoIDL para DbC, contemplando
dois tipos de precondição e pós-condição: uma básica, que valida os 
parâmetros de requisição de entrada e dados de saída; outra baseada em serviços,
em que outros serviços são acionados para validar se a capacidade do serviço 
deve ou não ser executada (ou se foi executada adequadamente, em caso de
pós-condição). %Essa lógica foi expressa na sintaxe da NeoIDL e possibilitou a
%transformação para serviços com esse comportamento. 
Foram realizados dois estudos empíricos. Um primeiro estudo visando demonstrar
que a NeoIDL cumpre com seus
requisitos de expressividade e reuso, no contexto de utilização de Comando e
Controle no Exército Brasileiro. O segundo, sobre percepção de utilidade e
facilidade de uso também demostrou resultados satisfatórios em termos
simplicidade e controle sobre o código gerado.
% 
% \absdiv{Conclusões}
% 
% O trabalho demonstrou que o conceito de \textit{Design-by-Contract} se aplica
% também ao paradigma de orientação a serviços. O presente trabalho tem como
% principais limitações a ausência até o presente momento do uso em contexto real
% da NeoIDL com \textit{Design-by-Contract} e não implementação de plugins para
% outras linguagens.

\end{abstract}



% 
% Aim.  The purpose of this study was to test the hypothesis that structured abstracts might also be
%           appropriate for a particular psychology journal.
% 
% Method.  24 traditional abstracts from the Journal of Educational Psychology were re-written in a
%           structured form.  Measures of word length, information content and readability were made
%           for both sets of abstracts, and 48 authors rated their clarity.
% 
% Results.  The structured abstracts were significantly longer than the original ones, but they were also
%          significantly more informative and readable, and judged significantly clearer by these
%          academic authors.
% 
% Conclusions.  These findings support the notion that structured abstracts could be profitably introduced
%          into many journals.
% 


\end{document}