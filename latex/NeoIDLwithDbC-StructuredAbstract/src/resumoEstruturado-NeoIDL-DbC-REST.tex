\documentclass{article}
\usepackage[brazil]{babel} % Suporte para o Portugu�s
\usepackage[utf8]{inputenc} % Suporte para acentua��o sem necessidade dos
\usepackage{blindtext}
\usepackage[inline]{enumitem}

\newcommand{\absdiv}[1]{%
\par\addvspace{.5\baselineskip}% adjust to suit
\noindent\textbf{#1}\quad\ignorespaces
}

\hyphenation{NeoIDL}

\begin{document}

\bf  CONTRATOS REST ROBUSTOS E LEVES: UMA\\ ABORDAGEM EM DESIGN-BY-CONTRACT COM
NEOIDL

\begin{abstract}

\emph{Contexto.}
A demanda por integração entre sistemas de origens diversas e tecnologias
variadas fez aumentar a adoção de soluções baseadas em computação orienta a
serviços, sendo o uso de servi\-ços Web a estratégia mais comum para implementar
serviços, em particular com a adoção crescente do estilo arquitetural REST.
Por outro lado, REST ainda não dispõe de uma notação padrão para especificação
de contratos e linguangens como Swagger, YAML e WADL cumprem com o
único propósito de descrição de serviços, porém apresentam uma significativa
limitação: essas linguagens
são voltadas para computadores, tendo escrita e leitura complexas para
humanos -- o que prejudica a abordagem \textit{Contract-first}, prática
estimulada na orientação a serviços. Tal limitação motivou a especificação da
linguagem NeoIDL\footnote{Além de ser uma linguagem (\textit{Domain Specific
Language}), a NeoIDL também possui um framework de geração de código para outras linguagens de propósito
amplo.}, que foi concebida com o objetivo de ser mais expressiva para humanos,
além de prover suporte a modularização e herança.
\emph{Problema.} Entretanto, nenhuma dessas linguagens, incluindo a NeoIDL, dá
suporte a contratos robustos, como os possíveis de serem descritos em
linguagens ou extensões de linguagens com suporte a \emph{Design-by-Contract},
presentes tipicamente (ou exploradas) no paradigma de orientação a objetos.
\emph{Objetivos.}
O objetivo geral deste trabalho é investigar o uso de construções de
\textit{Design-by-Contract} no contexto de computação orientada a serviços,
verificando a viabilidade e utilidade de sua adoção na especificação de
contratos e implementação de serviços REST.
% \absdiv{Método}
%
% Após ampla revisão bibliográfica, sobretudo dos temas SOC, REST,
% \textit{Design-by-Contract}, a hipótese da aplicabilidade de DbC em
% especificação de contratos REST foi levantada. Com o propósito de validar essa
% hipótese, a sintaxe da NeoIDL foi extendida para suportar DbC e, em seguida,
% implementadas regras de transformação que traduzem as construções de DbC em
% código de validação para o \textit{framework Python
%  Twisted}. Paralelamente, foi conduzida pesquisa junto a desenvolvedores
%  experientes sobre a sua aceitação de especificações de contratos REST com
%  \textit{Design-by-Contract} na NeoIDL. Adicionalmente, foi realizada uma
%  análise empírica sobre a expressividade e reuso da NeoIDL em si.
\emph{Resultados e Contribuições.}
Essa dissertação contribui tecnicamente com uma extensão da NeoIDL para DbC, contemplando
dois tipos de precondição e pós-condição: uma básica, que valida os
parâmetros de requisição de entrada e dados de saída; e outra baseada em serviços,
em que composições de serviços são acionados para validar se a
capacidade do serviço deve ou não ser executada (ou se foi executada adequadamente, em caso de
uma pós-condição). %Essa lógica foi expressa na sintaxe da NeoIDL e possibilitou a
%transformação para serviços com esse comportamento.
Sob a perspectiva de validação empírica, esta dissertação contribui com dois estudos.
Um primeiro, com o intuito de verificar
os requisitos de expressividade e reuso da NeoIDL, sendo realizado no
domínio de Comando e Controle em parceria com o Exército Brasileiro. O segundo
estudo teve como maior interesse a análise da percepção de utilidade e
facilidade de uso das construções DbC propostas para a NeoIDL,
levando a respostas positivas em termos de simplicidade e aceitação dos efeitos
sobre código gerado a partir de especificações NeoIDL.
%
% \absdiv{Conclusões}
%
% O trabalho demonstrou que o conceito de \textit{Design-by-Contract} se aplica
% também ao paradigma de orientação a serviços. O presente trabalho tem como
% principais limitações a ausência até o presente momento do uso em contexto real
% da NeoIDL com \textit{Design-by-Contract} e não implementação de plugins para
% outras linguagens.

\end{abstract}


\end{document}
