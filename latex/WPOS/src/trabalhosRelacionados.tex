\section{Trabalhos Relacionados}\label{sec:trabRelacionados}

Verifica-se que, embora definido já há alguns anos, DBC continua
sendo objeto de pesquisa \cite{poyias2014design}, \cite{rubio2013verifying},
\cite{belhaouari2012design}. Nesse dois últimos casos, o estudo de caso está
associado a controle de acesso, cenários aderentes a que se pretende
atingir com esta pesquisa.

Durante a pesquisa bibliográfica, muito embora o primeiro princípio SOA
estabelecido por \cite{erl2008soa} recomende \textit{Contract First}, não se
identificou publicação que associasse diretamente \textit{Design-by-Contract} à
especificação de contratos SOA.

Os trabalhos que mais se aproximam são os de \cite{ling2003describing} e
\cite{heckel2005towards}. O primeiro define uma forma de \textit{Design-by-Contract} para
arquiteturas de \textit{workflow}. O autor considera, porém, que para grandes
arquiteturas, a complexidade aumenta o risco de falha de \textit{design}. Já
\cite{heckel2005towards} foca na especificação de modelos para DbC em Web
Services, não tratando a especificação concreta do contrato. Além disso, se
restringe a Web Services SOAP.

