\section{Introdução e Caracterização do Problema}\label{sec:introducao}

A computação orientada a serviços ( \emph{Service-oriented computing, SOC)} tem
se mostrado uma solução de \textit{design} de \textit{software} que favorece o
alinhamento às mudanças constantes e urgentes nas insituições
\cite{chen2008towards}. Os benefícios de SOC estão diretamente relacionados ao
baixo acoplamento dos serviços que compõem a solução, de forma que as partes
(nesse caso serviços) possam ser substituídas e evoluídas facilmente, ou ainda
rearranjadas em novas composições. Contudo, para que isso seja possível, é
necessário que os serviços possuam contratos bem definidos e independentes da
implementação.

Por outro lado, as linguagens de especificação de contratos para SOA apresentam
algumas limitações. Por exemplo, a linguagem WSDL (\emph{Web-services
description language}) \cite{zur2005developing} é considerada uma solução
verbosa que desestimula a abordagem \textit{Contract First}. Por essa razão,
especificações WSDL são usualmente derivadas a partir de anotações em código
fonte.
Além disso, os conceitos descritos em contratos na linguagem WSDL não são
diretamente mapeados aos elementos que compõem as interfaces do estilo
arquitetural REST (\emph{Representational State Transfer}).
Outras alternativas para REST, como Swagger e
RAML\footnote{http://raml.org/spec.html}, usam linguagens de propósito geral (em
particular JSON) adaptadas para especificação de contratos. Ainda que façam uso
de contratos mais sucintos que WSDL, essas linguagens não se
beneficiam da clareza típica das linguagens específicas para esse fim (como IDLs CORBA) e não oferecem
mecanismos semânticos de extensibilidade e modularidade.

Com o objetivo de mitigar esses problemas, a linguagem \neoidl foi proposta
para simplificar a especificação de serviços REST com mecanismos de modularização,
suporte a anotações, herança em tipos de dados definidos pelo desenvolvedor, e
uma sintaxe simples e concisa semelhante às \textit{Interface Description
Languages} -- IDLs -- presentes em \textit{Apache Thrift}\texttrademark e
CORBA\texttrademark. Por outro lado, a \neoidl, da mesma forma que WSDL, Swagger
e RAML não oferece construções para especificação de contratos formais de
comportamento como os presentes em linguagens que suportam DBC (\emph{Design by
Contract})~\cite{meyer1992applying}, como JML, Spec\# e Eiffel.

Dessa forma, os objetivos dessa pesquisa envolvem inicialmente investigar o uso
de construções DBC no contexto da computação orientada a serviços e conduzir uma
revisão sistemática da literatura para identificar os principais trabalhos que
lidam com a relação entre DBC e SOC. Em seguida especificar e implementar novas
construções para a linguagem \neoidl, de tal forma que seja possível especificar
contratos mais precisos; além de definir regras de transformação das novas
construções \neoidl para diferentes tecnologias (como Twisted) que suportam a
implementação de serviços em REST.
Por fim, conduzir uma validação empírica da proposta usando uma abordagem mais
exploratória, possivelmente usando a estratégia \emph{pesquisa-ação}.


