\noindent {\large {\bf RESUMO}}

\vspace{5mm}

\noindent {\bf CONTRATOS REST ROBUSTOS E LEVES: UMA ABORDAGEM EM
DESIGN-BY-CONTRACT COM NEOIDL } 
 
\vspace{5mm} 

\noindent  {\bf
Autor: Lucas Ferreira de Lima}

\noindent {\bf Orientador: Rodigo Bonifácio de Almeida}

\noindent {\bf Programa de Pós-Graduação em Engenharia Elétrica}

\noindent {\bf Brasília, julho de 2016 }


\emph{Contexto.}
A demanda por integração entre sistemas heterogêneos fez aumentar a adoção de
soluções baseadas em computação orientada a serviços -- SOC, sendo o uso de
serviços Web a estratégia mais comum para implementar serviços, com a adoção crescente do estilo arquitetural REST.
Por outro lado, REST ainda não dispõe de uma notação padrão para especificação
de contratos e linguagens como Swagger, YAML e WADL cumprem com o
único propósito de descrever serviços, porém apresentam uma significativa
limitação: são voltadas para computadores, tendo escrita e leitura complexas para
humanos -- o que prejudica a abordagem \textit{Contract-first}, prática
estimulada em SOC. Tal limitação motivou a especificação da
linguagem NeoIDL\footnote{Além de ser uma linguagem (\textit{Domain Specific
Language}), a NeoIDL também possui um framework de geração de código para outras linguagens de propósito
amplo.}, concebida com o objetivo de ser mais expressiva para humanos,
além de prover suporte a modula\-ri\-za\-ção e herança.
\emph{Problema.} Nenhuma dessas linguagens, incluindo a NeoIDL, dá
suporte a contratos robustos, como os possíveis de serem descritos em
linguagens ou extensões de linguagens com suporte a \designbycontract{},
exploradas tipicamente no paradigma de orientação a objetos.
\emph{Objetivos.}
O objetivo geral deste trabalho é investigar o uso de construções de
\textit{Design-by-Contract} no contexto de SOC,
verificando a viabilidade e utilidade de sua adoção na especificação de
contratos e implementação de serviços REST.
% \absdiv{Método}
%
% Após ampla revisão bibliográfica, sobretudo dos temas SOC, REST,
% \textit{Design-by-Contract}, a hipótese da aplicabilidade de DbC em
% especificação de contratos REST foi levantada. Com o propósito de validar essa
% hipótese, a sintaxe da NeoIDL foi extendida para suportar DbC e, em seguida,
% implementadas regras de transformação que traduzem as construções de DbC em
% código de validação para o \textit{framework Python
%  Twisted}. Paralelamente, foi conduzida pesquisa junto a desenvolvedores
%  experientes sobre a sua aceitação de especificações de contratos REST com
%  \textit{Design-by-Contract} na NeoIDL. Adicionalmente, foi realizada uma
%  análise empírica sobre a expressividade e reuso da NeoIDL em si.
\emph{Resultados e Contribuições.}
Essa dissertação contribui tecnicamente com uma extensão da NeoIDL para DbC, contemplando
dois tipos de precondição e pós-condição: uma básica, que valida o valor de
atributos e dados de saída; e outra baseada em
serviços, em que composições de serviços são acionadas para validar se o serviço
deve ser executado (ou se foi executado adequadamente, em caso de pós-condições). %Essa lógica foi expressa na sintaxe da NeoIDL e possibilitou
% a transformação para serviços com esse comportamento.
Sob a perspectiva de validação empírica, esta dissertação contribui com dois estudos.
Um primeiro, verificou
os requisitos de expressividade e reuso da NeoIDL, sendo realizado no
domínio de Comando e Controle em parceria com o Exército Brasileiro. O segundo,
teve maior interesse na análise da percepção de utilidade e facilidade de uso
das construções DbC propostas para a NeoIDL, levando a respostas positivas em
termos de facilidade de uso e aceitação.
%
% \absdiv{Conclusões}
%
% O trabalho demonstrou que o conceito de \textit{Design-by-Contract} se aplica
% também ao paradigma de orientação a serviços. O presente trabalho tem como
% principais limitações a ausência até o presente momento do uso em contexto real
% da NeoIDL com \textit{Design-by-Contract} e não implementação de plugins para
% outras linguagens.

  
\clearpage



