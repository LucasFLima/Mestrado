%%%%%%% Arquivo geral - recebe as informa��es dos preliminares, monta o sum�rio
%%%%%%% a lista de tabelas, a lista de figuras. � necess�rio entrar com arquivo de s�mbolos
%%%%%%% e abreveaturas. Recebe ainda as informa�oes do texto principal, refer�ncias 
%%%%%%% bibliogr�ficas e ap�ndices.
% A classe do documento est� definida em modelo.tex, bem como o layout da folha.

%%%%%%%%%%%%%%%%%%%%% pend�ncias:
%%%%%%%%%%%%%%%%%%%%% - tirar o negrito e os dois pontos das legendas de figuras e tabelas;
%%%%%%%%%%%%%%%%%%%%% - retirar os [ ] das refer�ncias nas refer�ncias e no caso das citacoes
%%%%%%%%%%%%%%%%%%%%% - substitu�-los por ( ).
%%%%%%%%%%%%%%%%%%%%% - retirar a palavra Ap�ndice do t�tulo dos nos Ap�ndices.


\documentclass[a4paper,12pt,oneside]{teseft}

\pagestyle{plain}

\usepackage{indentfirst} % Identar tamb�m o primeiro par�grafo de cada se��o

%\userpackage{}

%\usepackage[brazilian]{babel}
%\usepackage[utf8]{inputenc}
%\usepackage[T1]{fontenc}

\usepackage[brazil]{babel} % Suporte para o Portugu�s
\usepackage[utf8]{inputenc} % Suporte para acentua��o sem necessidade dos
% comandos especiais.
%\usepackage{mathrsfs} % estabelece fontes para it�lico em modo matem�tico
\usepackage{graphicx}
%\usepackage{flafter}
%\usepackage[all]{xy}
%\usepackage[none]{hyphenat} % N�o utiliza separa��o de s�labas.

%Dimens�es da p�gina ok
\paperwidth=210mm
\paperheight=297mm

%\voffset=-11.4mm % dist�ncia de 15mm ao topo da p�gina (1in+(-11.4mm)=14mm).
\voffset=-11.4mm
% topmargin - distancia acima do in�cio do cabe�alho (no caso 5 mm)
\topmargin=5mm
\headheight=0mm % topmargin, headheight e headsep totalizam 10mm, 
%que, somados aos 15mm 
%%%%%%%%%%%%%%%%% anteriores, resultam nos 25mm especificados para a margem superior.
\headsep=5mm 
% headsep - separa�ao vertical entre o cabe�alho e o in�cio do corpo do texto.
%\textheight=247mm % 297mm-2*25mm, ou seja, a altura da p�gina menos o espa�o reservado
%%%%%%%%%%%%%%%%%%% para as margens superior e inferior.
\textheight=247mm
%\footskip=12mm    % assim, sobram cerca de 25mm-12mm=13mm entre a numera��o de p�gina e
%%%%%%%%%%%%%%%%%%% a parte inferior do papel. 
\footskip=9mm

%\hoffset=-10.4mm % dist�ncia de 15mm � parte esquerda da p�gina.
\hoffset=-10.4mm % dist�ncia de 20mm � parte esquerda da p�gina.
%\oddsidemargin=10mm % que, somados aos 15mm anteriores, resultam nos 25mm especificados.
%\evensidemargin=10mm
\oddsidemargin=15mm
\evensidemargin=15mm

%\textwidth=155mm % 210mm-30mm-25mm (30mm lado esquerdo 25 lado direito).
%\textwidth - largura do texto
\textwidth=155mm

\marginparsep=0mm
\marginparwidth=0mm % n�o � reservado espa�o para notas de margem lateral.

%Espa�amento 1,5
\linespread{1.3}

%%%%%%%%%%%%%%%%%%%%%%%%%%% 


% o pacote gr�fico ainda nao foi acrescentado
%\usepackage[dvips]{graphics}

% estou acrescentando os comandos abaixo: sao necess�rios
% para os tipos de par�grafos adotados: sem identa�ao e com uma linha
% em branco entre cada um.
%\setlength{\parindent}{0pt}
%\setlength{\parskip}{0.7cm plus 0.5ex minus 0.2ex}
% 

\usepackage{enumerate}
%\usepackage{enumitem} 
\usepackage{graphicx,color}
\usepackage{pdfpages}
\usepackage{booktabs}
\usepackage{listings}
\usepackage{multirow}
\usepackage[table]{xcolor}
\usepackage{latexsym}

\usepackage{xcolor}
\usepackage{stackengine}
\newlength\llength
\llength=1.38ex\relax

\hyphenation{NeoIDL a-li-nha-das des-cre-ver re-a-li-za-do MessageType
cor-res-pon-den-te ge-ra-do do-cu-men-ta-da res-pos-tas mo-de-lo
a-pre-sen-ta-do cons-tan-tes}
% u-ti-li-da-de re-a-li-za-\c{c}\~{a}o subs-ti-tu-\'{i}-das}

\newenvironment{joincode}  {\let\orighscode=\hscode
  \let\origendhscode=\endhscode
  \def\endhscode{\def\hscode{\endgroup\def\@currenvir{hscode}\\}\begingroup}
    \orighscode\def\hscode{\endgroup\def\@currenvir{hscode}}}  {\origendhscode
  \global\let\hscode=\orighscode
  \global\let\endhscode=\origendhscode}

\newcommand{\neoidl}{NeoIDL}
\newcommand{\neocortex}{NeoCortex}
\newcommand{\bnfc}{\texttt{BNFConverter}}
\newcommand{\designbycontract}{\textit{Design-by-Contract}}
\newcommand{\twisted}{\textit{Twisted}}
\newcommand{\ws}{\textit{Web Service}}
\newcommand{\wss}{\textit{Web Services}}
\newcommand{\CdFirst}{\textit{Code-first}}
\newcommand{\CtFirst}{\textit{Contract-first}}
\newcommand{\framework}{\textit{framework}}
\newcommand{\method}[1]{\texttt{#1}}

\newcommand{\emptyP}{\mbox{$\epsilon$}}
\newcommand{\terminal}[1]{\mbox{{\texttt {#1}}}}
\newcommand{\nonterminal}[1]{\mbox{$\langle \mbox{{\sl #1 }} \! \rangle$}}
\newcommand{\arrow}{\mbox{::=}}
\newcommand{\delimit}{\mbox{$|$}}
\newcommand{\reserved}[1]{\mbox{{\texttt {#1}}}}
\newcommand{\literal}[1]{\mbox{{\texttt {#1}}}}
\newcommand{\symb}[1]{\mbox{{\texttt {#1}}}}
 

\definecolor{light-gray}{gray}{0.85}
\definecolor{dark-yellow}{RGB}{215,167,0}

\newcommand{\hilight}[1]{\colorbox{light-gray}{#1}}
 
 



\lstdefinelanguage{json}{
  basicstyle=\normalfont\ttfamily,
  numbers=left,
  numberstyle=\scriptsize,
  stepnumber=1,
  numbersep=8pt,
  showstringspaces=false,
  breaklines=true,
  frame=top,
   literate=
  *{0}{{{\color{numb}0}}}{1}
  {1}{{{\color{numb}1}}}{1}
  {2}{{{\color{numb}2}}}{1}
  {3}{{{\color{numb}3}}}{1}
  {4}{{{\color{numb}4}}}{1}
  {5}{{{\color{numb}5}}}{1}
  {6}{{{\color{numb}6}}}{1}
  {7}{{{\color{numb}7}}}{1}
  {8}{{{\color{numb}8}}}{1}
  {9}{{{\color{numb}9}}}{1}
  {:}{{{\color{punct}{:}}}}{1}
  {,}{{{\color{punct}{,}}}}{1}
  {\{}{{{\color{delim}{\{}}}}{1}
  {\}}{{{\color{delim}{\}}}}}{1}
  {[}{{{\color{delim}{[}}}}{1}
  {]}{{{\color{delim}{]}}}}{1},
}



\lstdefinelanguage{NeoIDL}{
  sensitive = true,
  keywords={},
  otherkeywords={% Operators
    >, <, ==
  },
  keywords = [2]{module, resource, enum, annotation, for, import, entity, path,
  @get, @post, @put, @delete, require, ensure, otherwise, call, and, or, not},
  keywordstyle=[1]\color{dark-yellow}\textbf,
  keywordstyle=[2]\color{black}\textbf,
  numbers=left,
  numberstyle=\scriptsize,
  stepnumber=1,
  numbersep=8pt,
  showstringspaces=false,
  breaklines=true,
  frame=top,
  comment=[l]{//},
  morecomment=[s]{/*}{*/},
  commentstyle=\color{purple}\ttfamily,
  stringstyle=\color{red}\ttfamily,
  morestring=[b]',
  morestring=[b]"
  }

\lstdefinelanguage{PythonTwisted}{
  sensitive = true,
  keywords = {classe, object, def, return},
  numbers=left,
  numberstyle=\scriptsize,
  stepnumber=1,
  numbersep=8pt,
  showstringspaces=false,
  breaklines=true,
  frame=lines,
  comment=[l]{\#},
  morecomment=[s]{/*}{*/},
  commentstyle=\color{purple}\ttfamily,
  stringstyle=\color{red}\ttfamily,
  morestring=[b]',
  morestring=[b]"
  }
  

\lstdefinelanguage{Haskell}{
  sensitive = true,
  keywords = {data, type, module, where, do, IO, import},
  numbers=left,
  numberstyle=\scriptsize,
  stepnumber=1,
  numbersep=8pt,
  showstringspaces=false,
  breaklines=true,
  frame=lines,
  }
  
  \lstdefinelanguage{HaskellSimple}{
  sensitive = true,
  keywords = {data, type, module, where, do, IO, import},
  showstringspaces=false,
  breaklines=true,
  frame=lines,
  }
  
  \lstdefinelanguage{Eiffel}{
  sensitive = true,
  keywords = {class, feature, create, require, ensure, end, is, INTEGER,
  requires, ensures}, numbers=left,
  numberstyle=\scriptsize,
  stepnumber=1,
  numbersep=8pt,
  showstringspaces=false,
  breaklines=true,
  frame=top,
  comment=[l]{--},
  morecomment=[s]{/*}{*/},
  commentstyle=\color{purple}\ttfamily,
  }
  

  \lstdefinelanguage{JML}{
  sensitive = true,
  keywords = {class, public, static, int, return, requires, ensures},
  numbers=left,
  numberstyle=\scriptsize,
  stepnumber=1,
  numbersep=8pt,
  showstringspaces=false,
  breaklines=true,
  frame=top,
  }
  
   \lstdefinelanguage{SpecSharp}{
  sensitive = true,
  keywords = {class, public, require, ensure, true, int},
  numbers=left,
  numberstyle=\scriptsize,
  stepnumber=1,
  numbersep=8pt,
  showstringspaces=false,
  breaklines=true,
  frame=top,
  comment=[l]{//},
  morecomment=[s]{/*}{*/},
  commentstyle=\color{purple}\ttfamily,
  }
  
    
   \lstdefinelanguage{CodeContract}{
  sensitive = true,
  keywords = {class, public, private, require, ensure, true, int,
  Contract, Requires, Ensures, return}, numbers=left,
  numberstyle=\scriptsize,
  stepnumber=1,
  numbersep=8pt,
  showstringspaces=false,
  breaklines=true,
  frame=top,
  comment=[l]{//},
  morecomment=[s]{/*}{*/},
  commentstyle=\color{purple}\ttfamily,
  }
  
  \lstdefinelanguage{Perl}{
  sensitive = true,
  keywords = {use, my, split, local, undef, open, die, binmode, close, print,
  lc, if, exists, else, for keys, exists}, numbers=left,
  numberstyle=\scriptsize,
  stepnumber=1,
  numbersep=8pt,
  showstringspaces=false,
  breaklines=true,
  frame=top,
  comment=[l]{#},
  morecomment=[s]{/*}{*/},
  commentstyle=\color{purple}\ttfamily,
  stringstyle=\color{red}\ttfamily,
  morestring=[b]',
  morestring=[b]",
  }
  


\begin{document}
\pagenumbering{roman}
%\frontmatter
\thispagestyle{empty}

\begin{bf}
\hspace{10mm}
\vfill

\begin{center}
\rule{155mm}{0.1mm}

T�TULO DA DISSERTA��O EM MAI�SCULAS

\vspace{10mm}

AUTOR DA DISSERTA��O

\vspace{15mm}

DISSERTA��O DE MESTRADO EM ENGENHARIA EL�TRICA

\vspace{10mm}

DEPARTAMENTO DE ENGENHARIA EL�TRICA

\rule{155mm}{0.1mm}
\end{center}

\hspace{10mm}
\vfill
\end{bf}

\pagebreak

\thispagestyle{empty}
\hspace{10mm}

\pagebreak
 Uma capa semelhante a esta dever� ser usada na encaderna��o.


%=======================================================================
% Folha de t�tulo da disserta��o

\thispagestyle{empty}
\setcounter{page}{1}
\begin{center}
{\normalsize {\bf UNIVERSIDADE DE BRAS\'{I}LIA\\
FACULDADE DE TECNOLOGIA\\
DEPARTAMENTO DE ENGENHARIA EL\'{E}TRICA }}

\vfill

% INSERIR AQUI O T�TULO DA DISSERTACAO
{\large {\bf  CONTRATOS REST ROBUSTOS E LEVES: UMA ABORDAGEM EM
DESIGN-BY-CONTRACT COM NEOIDL }}


\vfill

% INSERIR AQUI O NOME DO AUTOR DA DISSERTA��O
{\large {\bf LUCAS FERREIRA DE LIMA}}

\vspace{20mm}

% INSERIR AQUI O NOME DO ORIENTADOR: (NOME)
{\normalsize {\bf ORIENTADOR: RODRIGO BONIFÁCIO DE ALMEIDA }}

\vspace{20mm}

{\normalsize {\bf DISSERTAÇÃO DE MESTRADO EM\\
% INSERIR ABAIXO O NOME DO PROGRAMA DE P�S-GRADUACAO
ENGENHARIA ELÉTRICA  }}

\vspace{10mm}

% inserir o n�mero da publica��o distinguindo entre disserta�ao (DM) e tese (TD)
{\normalsize {\bf PUBLICAÇÃO: MTARH.DM - 017 A/99 }}


\vspace{10mm}

% inserir o local, m�s e ano da publica�ao
{\normalsize {\bf BRASÍLIA/DF: JULHO - 2016.    }  }
\end{center}

\pagebreak

\thispagestyle{empty}
\hspace{10mm}
\addtocounter{page}{-1}

\pagebreak

%=======================================================================

\input{aprova}
% \input{aprova2}


\noindent \begin{bf} \MakeUppercase{Ficha Catalográfica} \end{bf}

% \vspace{10mm}

\noindent\framebox[155mm][l]{\begin{tabular}{l}
% inserir  NOME DO AUTOR (�ltimo nome seguido de v�rgula e dos primeiros nomes)
  \hspace{2mm}  LIMA, LUCAS FERREIRA DE \\
  % t�tulo da disserta�ao (identar o resto do t�tulo em 21 mm, caso o mesmo nao
  % caiba em uma linha), acrescentar [Distrito Federal] e o ano. 
   \hspace{2mm} Contrato REST robustos e leves: uma abordagem em
   Design-by-Contract\\ com NeoIDL.
   [Distrito Federal] 2016.\\
%  numero de paginas de preliminares e texto, tamanho da p�gina e outros dados padroes.
  \hspace{2mm} xvii, 98p., 297 mm (ENE/FT/UnB, Mestre, Engenharia Elétrica,
  2016).  \\
  \hspace{2mm} Dissertação de  Mestrado - Universidade de Brasília. \\
  \hspace{2mm} Faculdade de Tecnologia. \\
%  
 % nome do departamento
  \hspace{2mm}  Departamento de Engenharia Elétrica. \\
    % palavra-chave 1
  \begin{tabular}{ll}
  \hspace{1mm}  1. Computação orientada a serviços  & 
  % palavra-chave 2
    ~ ~ ~ ~ ~ 2. \designbycontract{}\\
  % palavra-chave 3  
  \hspace{2mm}3. \neoidl{}
  %palavra-chave 4
  & ~ ~ ~ ~ ~ 4. REST \\
  % texto padrao
  \hspace{2mm}I. ENE/FT/UnB &~ ~ ~ ~ ~ II. Título (série) \end{tabular}
  \end{tabular}}

  
\vspace{10mm}

\noindent \begin{bf} \MakeUppercase{Referência Bibliográfica} \end{bf}

\vspace{-1mm}

\noindent CONTRATOS REST ROBUSTOS E LEVES: UMA ABORDAGEM EM
DESIGN-BY-CONTRACT COM NEOIDL

% refer�ncia bibliogr�fica padrao para o autor
\noindent  LIMA, L. F. (2016). Contratos REST Robustos e Leves: Uma
Abordagem em Design-by-Contract com NeoIDL. Dissertação de Mestrado em
Engenharia Elétrica, Publicação 642/2016 DM PPGEE, Departamento de Engenharia Elétrica,
Universidade de Brasília, Brasília, DF, 82p.



\vspace{6mm}

\noindent \begin{bf} \MakeUppercase{Cessão de Direitos} \end{bf}

\vspace{5mm}

% nome do autor
\noindent NOME DO AUTOR: Lucas Ferreira de Lima.
\vspace{6mm}

% t�tulo da dissertacao
\noindent TÍTULO DA DISSERTAÇÃO DE MESTRADO: Contratos REST Robustos e Leves: Uma
Abordagem em Design-by-Contract com NeoIDL. 

\vspace{3mm}
% grau e ano
\noindent GRAU / ANO:~ ~ ~ Mestre / 2016

\vspace{5mm}

\noindent É concedida a Universidade de Brasília permissão para reproduzir
cópias desta dissertação de mestrado e para emprestar ou vender tais cópias
somente para propósitos acadêmicos e científicos. O autor reserva outros
direitos de publicação e nenhuma parte desta dissertação de mestrado pode ser
reproduzida sem a autorização por escrito do autor.

\vspace{5mm}

\noindent \underline{\hspace{65mm}}

\vspace{-2mm}
% nome do autor

\noindent  Lucas Ferreira de Lima
   \vspace{-2mm}

 % endere�o do autor
\noindent Cond. RK Conj. Antares Qd. L Cs. 48 Sobradinho 
    \vspace{-2mm}
    
\noindent 73.252-200 - Brasília - DF - Brasil.

\pagebreak

% p�gina em branco a seguir, sem numera�ao (empty)
%\thispagestyle{empty}
%\hspace{10mm}
%\addtocounter{page}{-1}

%\pagebreak

%\vspace{5mm}

\noindent {\large \begin{bf} \MakeUppercase{Dedicatória} \end{bf}}

\vspace{13cm}



\begin{quote}
  \hspace{7cm} Este trabalho é dedicado a ...\\ 
  
  \vspace{-10mm}
   \hspace{7cm}   continuação ............ 
\end{quote}

% Deus
% Família (pais, irmãos, esposa, filhos, cunhados e todos os outros)
% Professores Rodrigo, Edna, Uirá - orientação
% Professores Rafael, Puttini e Paulo 
% Coordenadores André Noll, Ugo Dias e Kleber Melo 
% Apoio: Adriana Reis e Igor 
% Equipe do Exército, na pessoa do Thiago Mael, com destaque para Leandro
% Loriato
% Equipe da SOF, na pessoa do Marco
% Equipe do TSE


\clearpage
\pagebreak

% p�gina em branco a seguir, sem numera�ao (empty)
%\thispagestyle{empty}
%\hspace{10mm}
%\addtocounter{page}{-1}

%\pagebreak

%\vspace{5mm}

\noindent {\large \begin{bf} \MakeUppercase{Agradecimentos} \end{bf}}

%\vspace{3cm}

\begin{quote}

Agradeço a Deus, pela fé e esperança para superar as adversidades.\\
A minha amada esposa Paula, pelo imenso companherismo, estímulo e por ter se
desdobrado nos momentos de minha ausência, em especial nos cuidados com nossas jóias
preciosas, nossos amados filhos João Filipe, Maria Luísa e Ana Júlia.\\
Aos meus pais, pela dedicação e carinho. Aos meus irmãos pelo exemplo
determinante. Ao suporte indispensável de minha sogra e sogro, cunhados e
sobrinhos.\\
Aos professores Rodrigo Bonifácio e Edna Canedo, pela orientação,
paciência, incentivo e aprendizado.\\
Aos professores Rafael Timóteo, Ricardo Puttini e Valério Aymoré, pelas
relevantes contribuições do conhecimento transmitido.\\
À equipe do TSE, pelo estímulo, colaboração, compreensão. Devo muito a vocês.\\
Aos professores da Universidade do Rio Grande no Norte, na pessoa do professor
Uirá Kulesza.\\
Aos professores coordenadores André Noll, Ugo Dias e Kleber Melo, e à
Adriana Reis e ao Igor pelo suporte no PPGEE.\\
Aos oficiais do exército brasileiro, na pessoa do Thiago Mael, pela enorme
colaboração e crítica técnica. Nota de destaque para Leandro Loriato, pelas
questões importantes debatidas.\\
À equipe da Secretaria de Orçamento Federal, na pessoa do Marcos
César.\\


% Deus
% Família (pais, irmãos, esposa, filhos, cunhados e todos os outros)
% Professores Rodrigo, Edna, Uirá - orientação
% Professores Rafael, Puttini e Paulo 
% Coordenadores André Noll, Ugo Dias e Kleber Melo 
% Apoio: Adriana Reis e Igor 
% Equipe do Exército, na pessoa do Thiago Mael, com destaque para Leandro
% Loriato
% Equipe da SOF, na pessoa do Marco
% Equipe do TSE

 
\end{quote}

\clearpage

\pagebreak

% p�gina em branco a seguir, sem numera�ao (empty)
%\thispagestyle{empty}
%\hspace{10mm}
%\addtocounter{page}{-1}

%\pagebreak


%%%%%%%%%%%%%%%%%%%%%%%%%%% Resumo e abstract %%%%%%%%%%%%%%%%%%%%%%%%
\noindent {\large {\bf RESUMO}}

\vspace{5mm}

\noindent {\bf CONTRATOS REST ROBUSTOS E LEVES: UMA ABORDAGEM EM
DESIGN-BY-CONTRACT COM NEOIDL } 
 
\vspace{5mm} 

\noindent  {\bf
Autor: Lucas Ferreira de Lima}

\noindent {\bf Orientador: Rodigo Bonifácio de Almeida}

\noindent {\bf Programa de Pós-Graduação em Engenharia Elétrica}

\noindent {\bf Brasília, julho de 2016 }


\emph{Contexto.}
A demanda por integração entre sistemas heterogêneos fez aumentar a adoção de
soluções baseadas em computação orientada a serviços -- SOC, sendo o uso de
serviços Web a estratégia mais comum para implementar serviços, com a adoção crescente do estilo arquitetural REST.
Por outro lado, REST ainda não dispõe de uma notação padrão para especificação
de contratos e linguagens como Swagger, YAML e WADL cumprem com o
único propósito de descrever serviços, porém apresentam uma significativa
limitação: são voltadas para computadores, tendo escrita e leitura complexas para
humanos -- o que prejudica a abordagem \textit{Contract-first}, prática
estimulada em SOC. Tal limitação motivou a especificação da
linguagem NeoIDL\footnote{Além de ser uma linguagem (\textit{Domain Specific
Language}), a NeoIDL também possui um framework de geração de código para outras linguagens de propósito
amplo.}, concebida com o objetivo de ser mais expressiva para humanos,
além de prover suporte a modularização e herança.
\emph{Problema.} Nenhuma dessas linguagens, incluindo a NeoIDL, dá
suporte a contratos robustos, como os possíveis de serem descritos em
linguagens ou extensões de linguagens com suporte a \designbycontract{},
exploradas tipicamente no paradigma de orientação a objetos.
\emph{Objetivos.}
O objetivo geral deste trabalho é investigar o uso de construções de
\textit{Design-by-Contract} no contexto de SOC,
verificando a viabilidade e utilidade de sua adoção na especificação de
contratos e implementação de serviços REST.
% \absdiv{Método}
%
% Após ampla revisão bibliográfica, sobretudo dos temas SOC, REST,
% \textit{Design-by-Contract}, a hipótese da aplicabilidade de DbC em
% especificação de contratos REST foi levantada. Com o propósito de validar essa
% hipótese, a sintaxe da NeoIDL foi extendida para suportar DbC e, em seguida,
% implementadas regras de transformação que traduzem as construções de DbC em
% código de validação para o \textit{framework Python
%  Twisted}. Paralelamente, foi conduzida pesquisa junto a desenvolvedores
%  experientes sobre a sua aceitação de especificações de contratos REST com
%  \textit{Design-by-Contract} na NeoIDL. Adicionalmente, foi realizada uma
%  análise empírica sobre a expressividade e reuso da NeoIDL em si.
\emph{Resultados e Contribuições.}
Essa dissertação contribui tecnicamente com uma extensão da NeoIDL para DbC, contemplando
dois tipos de precondição e pós-condição: uma básica, que valida o valor de
atributos e dados de saída; e outra baseada em
serviços, em que composições de serviços são acionadas para validar se o serviço
deve ser executado (ou se foi executado adequadamente, em caso de pós-condições). %Essa lógica foi expressa na sintaxe da NeoIDL e possibilitou
% a transformação para serviços com esse comportamento.
Sob a perspectiva de validação empírica, esta dissertação contribui com dois estudos.
Um primeiro, verificou
os requisitos de expressividade e reuso da NeoIDL, sendo realizado no
domínio de Comando e Controle em parceria com o Exército Brasileiro. O segundo,
teve como maior interesse a análise da percepção de utilidade e facilidade de uso das construções DbC propostas para a NeoIDL,
levando a respostas positivas em termos de simplicidade e aceitação dos efeitos
sobre código gerado a partir de especificações NeoIDL.
%
% \absdiv{Conclusões}
%
% O trabalho demonstrou que o conceito de \textit{Design-by-Contract} se aplica
% também ao paradigma de orientação a serviços. O presente trabalho tem como
% principais limitações a ausência até o presente momento do uso em contexto real
% da NeoIDL com \textit{Design-by-Contract} e não implementação de plugins para
% outras linguagens.

  
\clearpage




\noindent
{\large {\bf ABSTRACT}}

\vspace{5mm}
\noindent {\bf CONTRATOS REST ROBUSTOS E LEVES: UMA ABORDAGEM EM
DESIGN-BY-CONTRACT COM NEOIDL } 
 
\vspace{5mm} 

\noindent  {\bf
Autor: Lucas Ferreira de Lima}

\noindent {\bf Orientador: Rodigo Bonifácio de Almeida}

\noindent {\bf Programa de Pós-Graduação em Engenharia Elétrica}

\noindent {\bf Brasília, julho de 2016 }

\vspace{5mm}
\noindent
..

\vspace{5mm}
\noindent
..

\vspace{5mm}
\noindent
..

\vspace{5mm}
\noindent
..

\clearpage

%%%%%%%%%%%%%%%%%%%%%%%%%%% 

%
\tableofcontents
\listoftables
\listoffigures

\pagebreak

% \clearpage
%\hspace{1pt}\vspace{1pt}
%\pagebreak
\noindent {\large \begin{bf} LISTA DE S\'{I}MBOLOS, NOMENCLATURA E
ABREVIAÇÕES
\end{bf}}
%\addcontentsline{toc}{chapter}{LISTA DE S�MBOLOS, NOMECLATURAS E ABREVIA��ES}

\newcommand{\acrlista}[1]{\vspace{12pt}\noindent #1}

\acrlista{SOC: {Software Oriented Computing}, modelo arquitetural baseado em
serviços.}

\acrlista{DbC: {Design by Contract}, mecanismos de garantias com condições na
chamadas a métodos, funções, serviços, etc.}

\pagebreak


% fim do preambulo (preliminares) - a numera�ao a partir daqui � em ar�bico
% e come�ando da p�gina 1.

%%%%%%%%%%%%%%%%%% in�cio dos cap�tulos

% os dois comandos abaixo tiram a identa�ao nos par�grafos
% e indicam que o espa�o equivalente a uma linha de texto deve ser 
% deixado em branco entre par�grafos.
\setlength{\parindent}{0pt}
\setlength{\parskip}{0.7cm plus 0.5ex minus 0.2ex}
%

% Introdução - Estudos empíricos / DbC para NeoIDL
% Fundamentação teórica -
% NeoIDL - Objetivos / Resultados / Extensão da Linguagem / Avaliação empírica
% / Destacar as mudanças da sintaxe da NeoIDL para e o uso do BNFConverter
% Análise da Aceitação da NeoIDL - Método / GQM / Resultados
% Conclusão 

% 1. Introdução - DbC para SOC(NeoIDL), estudo empíricos


%%%%%%%%%%%%%%%%%% in�cio dos cap�tulos
% cap�tulo 1
\chapter{INTRODUÇÃO}
\pagenumbering{arabic}
\vspace{-6mm}

A computação orientada a serviços ( \emph{Service-oriented computing, SOC)} tem
se mostrado uma solução de \textit{design} de \textit{software} que favorece o
alinhamento às mudanças constantes e urgentes nas instituições
\cite{chen2008towards}. Nessa abordagem, os recursos de software são empacotados
como serviços, os quais são módulos bem definidos e auto-contidos, provêem funcionalidades negociais e com estado e
contexto independente \cite{papazoglou2007service}.

Os benefícios de SOC estão diretamente relacionados ao
baixo acoplamento dos serviços que compõem a solução, de forma que as partes
(nesse caso serviços) possam ser substituídas e evoluídas facilmente, ou ainda
rearranjadas em novas composições. Contudo, para que isso seja possível, é
necessário que os serviços possuam contratos bem definidos e independentes da
implementação.

A relação entre quem provê e quem consome o serviço se
dá por meio de um contrato. O contrato de serviço é o documento que descreve os
propósitos e as funcionalidades do serviço, como ocorre a troca de mensagens, condições sobre
como as operações são realizadas e informações sobre as operações \cite{erl2009web}.

Nesse contexto, a qualidade da especificação do contrato é fundamental para o
projeto de software baseado em SOC. Este trabalho de pesquisa aborda um aspecto
importante para a melhoria da robustez de contratos de serviços: a construção de
garantias mútuas por meio da especificação formal de contratos, agregando o
conceito de Desing-by-Contract.

\section{PROBLEMA DE PESQUISA}
\vspace{-6mm}

As linguagens de especificação de contratos para SOC apresentam
algumas limitações. Por exemplo, a linguagem WSDL (\emph{Web-services
description language}) \cite{zur2005developing} é considerada uma solução
verbosa que desestimula a abordagem \textit{Contract First}. Por essa razão,
especificações WSDL são usualmente derivadas a partir de anotações em código
fonte \textit{Code First}.
Além disso, os conceitos descritos em contratos na linguagem WSDL não são
diretamente mapeados aos elementos que compõem as interfaces do estilo
arquitetural REST (\emph{Representational State Transfer}).
Outras alternativas para REST, como Swagger e
RAML\footnote{http://raml.org/spec.html}, usam linguagens de propósito geral (em
particular JSON e YAML) adaptadas para especificação de contratos. Ainda que
façam uso de contratos mais sucintos que WSDL, essas linguagens não se
beneficiam da clareza típica das linguagens específicas para esse fim (como IDLs CORBA) e não oferecem
mecanismos semânticos de extensibilidade e modularidade.

Com o objetivo de mitigar esses problemas, a linguagem \neoidl{} foi proposta
para simplificar a especificação de serviços REST com mecanismos de modularização,
suporte a anotações, herança em tipos de dados definidos pelo desenvolvedor, e
uma sintaxe simples e concisa semelhante às \textit{Interface Description
Languages} -- IDLs -- presentes em \textit{Apache Thrift}\texttrademark e
CORBA\texttrademark. Por outro lado, a \neoidl{}, da mesma forma que WSDL,
Swagger e RAML não oferece construções para especificação de contratos formais de
comportamento como os presentes em linguagens que suportam DBC (\emph{Design by
Contract})~\cite{meyer1992applying}, como JML, Spec\# e Eiffel. Em outras
palavras, a \neoidl{}  admite apenas contratos fracos (\textit{weak contracts}),
sem suporte a construções como pré e pós condições.



\section{OBJETIVO GERAL}
\vspace{-6mm}

O objetivo geral de trabalho é investigar o uso de construções de
\designbycontract{} no contexto de computação orientada a serviços, verificando a
viabilidade e utilidade de sua adoção na especificação de contratos e
implentação de serviços REST.

\subsection{Objetivos específicos}
\vspace{-6mm}

\begin{enumerate}
  \item Realizar análise empírica de expressividade e reuso da especificação de
  contratos em \neoidl{} em comparação com \textit{Swagger}, a partir de contratos
  reais do Exército Brasileiro.
  \item Extender a sintaxe da \neoidl{} para admitir construções de 
  \designbycontract, com pré e pós condições para operações de serviços REST.
  \item Implementar um estudo de caso de geração de código em \textit{Python
  Twisted} com suporte a \designbycontract a partir de contratos especificados
  em \neoidl
  \item Coletar a percepção de desenvolvedores sobre a aceitação da
  especificação de contratos REST com \designbycontract na \neoidl{}
\end{enumerate}


\subsection{Justificativa e relevância}
\vspace{-6mm}

A necessidade de integração entre sistemas de várias origens e tecnologias fez
aumentar a adoção de soluções baseada em computação orienta a serviços. Isso de
deve justamente de busar tornar a interoperabilidade de soluções heterogênias
o menos acopladas possível, de modo a que mudanças nos requisitos de negócio ou
na inclusão de novos serviços seja simples, eficiente e rápida.

O uso de \ws{} é a forma mais comum de se implementar os serviços. O
desenvolvimetno de \ws{}, que era inicialmente construído sobre a abordagem
SOAP, com o tráfego de mensagens codificadas em XML, tem gradativamente se
intensificado no sentido da utilização de REST.

Uma dos principais benefícios do uso de SOC está na possibilidade de reuso de
seus componentes. Porém, reuso requer que serviços bem construídos e precisos em
relação a sua especificação \cite{jazequel1997design}. A qualidade e precisão do
contrato de serviço torna-se assim um elemento fudamental para que auferir os benefícios da
abordagem SOC.

Nesse contexto, REST não dispõe de um meio padrão para especificação de
contratos. Linguangens como Swagger, YAML e WAML cumprem com o propósito de
especificar contratos REST, porém padecem do mesmo problema: são voltados para
computadores e de escrita não trivial para humanos, o que prejudica a prática de
\CtFirst{}. A linguagem \neoidl{} foi concebida com o objetivo de ser mais
expressiva para humanos.

Todas essas linguagens tem, entretanto, um outra limitação em comum: não dão
suporte a contratos robustos, com garantias. A estratégia para superar essa
limitação foi de buscar no paradigma de orientação a objetos, que é uma das
principais influências de orientação a serviços \cite{erl2009web},
o conceito de \designbycontract{}. Ambas as abordagens, orientação a serviços e
a objetos, tem em comum a ênfase no reuso e comunicação entre componentes
(serviços e classes).

A proposta deste trabalho de
pesquisa de mestrado está justamente em incluir garantias na especificação de
contratos REST, extendendo a linguagem \neoidl{} para suportar construções de
\designbycontract{}.



\subsection{Estrutura}
\vspace{-6mm}

Este trabalho está organizado em quatro capítulos\ldots

% Introdução - Estudos empíricos / DbC para NeoIDL
% Fundamentação teórica - SOC / REST / Contratos / DbC - 10 páginas
% NeoIDL - Objetivos / Resultados / Extensão da Linguagem / Avaliação empírica
% Análise da Aceitação da NeoIDL - Método / GQM / Resultados
% Conclusão 


% 2. Referêncial Teórico -  SOC / REST / Contratos / DbC - 10 páginas
 
\chapter{REFERENCIAL TEÓRICO}
\vspace{-6mm}

\section{COMPUTAÇÃO ORIENTADA A SERVIÇO}
\vspace{-6mm}


As empresas precisam estar preparadas para responder rápida e
eficientemente a mudanças impostas por novas regulações, por aumento de
competição ou ainda para usufruir de novas oportunidades. No contexto atual, em que as
informações fluem de modo extremamente veloz, o tempo disperdiçado pelas organizações para se
adaptar a um novo cenário tem um preço elevado, gerando expressiva perda de
receita e, em um determinados casos, podendo causar a falência.

No campo das instituições governamentais, a eficiência na condução das ações do
Estado impõem que a estrutura de troca de informações entre os mais variados
entes seja continuamente adaptável, mutuamente integrada. Pode-se tomar como
exemplo a edição de nova lei que implique alteração no cálculo do tempo de
serviço para aposentadoria. A nova fórmula deve se propagar para ser
aplicada em várias intituições que compõem a máquina pública.

Nessas situações, os sistemas de informação das organizações devem possibilitar
que a dinâmica de adaptação ocorra sem demora, sob pena de, em vez de serem
ferramental para apoiar continuamente os processos de negócio, se tornem entrave
para a ágil incorporação dos novos processos. Por outro lado, a nova
configuração deve se manter integra e funcional com o já complexo cenário de TI.

A eficiência na integração entre as soluções de TI é determinante para que se
consiga alterar uma parte sem comprometer todo o ecossistema. A integração
possibilita a combinação de eficiência e flexibilidade de recursos para otimizar
a operação através e além dos limites de uma organização e a habilita para
inteoperar facilmente \cite{papazoglou2008service}.

A computação orientada a serviços -- SOC -- endereça essas necessidades em uma
plataforma que aumenta a flexibilidade e melhora o alinhamento com o negócio, a
fim de reagir rapidamente a mudanças nos requisitos de negócio. Para obter esses
benefícios, os serviços devem cumprir com determinados quesitos, que incluem
alta autonomia ou baixo acoplamento \cite{erl2008soa}. Assim, o paradigma de SOC
está voltado para o projeto de soluções preparadas para constantes mudanças,
substituindo-se pequenas peças -- os serviços -- por outras.

Portando, o objetivo da SOC é conceber um estilo de projeto, tecnologia e
processos que permitam às empresas desenvolver, interconectar e manter suas
aplicações e serviços corporativos com eficiência e baixo custo. Embora esses
objetivos não sejam novos, SOC procura superar os esforços prévios como
programação modular, reuso de código e técnicas de desenvolvimento orientadas a
objetos \cite{papazoglou2007serviceApprTechRechIss}.

As vertentes mais visionárias -- não ainda concretizada e utópica para muitos
pesquisadores -- da computação orientada a serviços prevêem uma coordenação de
serviços cooperantes por todo o mundo, onde os componentes possam ser conectados
facilmente em uma rede de serviços pouquíssimo acoplados e, assim, criar
processos de negócio dinâmicos e aplicações ágeis entre organizações e plataformas de
computação \cite{leymann2005combining}.


\subsection{Terminologia}
\vspace{-6mm}

\begin{description}
\item [Computação orientada a serviço] é um termo \textit{guarda-chuva} para
descrever uma nova geração de computação distribuída. Desse modo, é um conceito
que engloba várias coisas, como paradigmas e princípios de projeto, catálogo de
padrões de projeto, padronização de linguagem, modelo arquitetural específico, e
conceitos correlacionados, tecnologias e plataformas.
A computação orientada a serviços é baseada em modelos anteriores de computação
distribuída e os extendem com novas camadas de projeto, aspectos de governança,
e uma grande gama de tecnologias de implementações especializadas, em grande
parte baseadas em \ws{} \cite{erl2009web}.

\item [Orientação a serviço] é um paradígma de projeto cuja intenção é a criação
de unidades lógicas moldadas individualmente para podem serem utilizadas
conjutamente e repetidamente para se atender a objetivos e funções específicos
associados com SOA e computação orientada a serviço.

A lógica concebida de acordo com orientação a serviço pode ser designada de
\textbf{orientada a serviço}, e as unidades da lógica orientada a serviço são
referenciadas como \textbf{serviços}. Como um paradigma de computação
distribuída, a orientação a serviço pode ser comparada a orientação a objetos,
de onde advém várias de suas raízes, além da influência de EAI, BMP e \ws
\cite{erl2009web}.

A orientação a serviços é composta principalmente de oito princípios de projeto
(descritos na seção \ref{PrincipiosSOA}).

\item [Arquitetura orientada a serviço - SOA] representa um modelo arquitetural
cujo objetivo é elevar a agilidade e a redução de custos e ao mesmo tempo
reduzir o peso da TI para a organização. Isso é feito colocando o serviço no
como elemento central da representação lógica da solução \cite{erl2009web}.

Como uma arquitetura tecnológica, uma implementação SOA consiste da combinação
de tecnologias, produtos, APIs, extensões da infraestrutura, etc. A implantação
concreta de uma arquitetura orientada a serviço é única para cada organização,
entretanto é caracterizada pela introdução de tecnologias e plataformas que
suportam a criação, execução e evolução de soluções orientadas a serviços. O
resultado é a formação de um ambiente projetado para produzir soluções alinhadas
aos princípios de projeto de orientação a serviço.

Segundo Thomas Erl \cite{erl2009web}, o termo arquitetura orientada a serviço --
SOA -- vem sendo amplamente utilizado na mídia e nos produtos de divugação de
fabricantes que tem se tornado quase que sinônimo de computação orientada a
serviço -- SOC.

\item [Serviço] é a unidade da solução no qual foi aplicada a orientação a
serviço. É a aplicação da orientação dos princípios de projeto de orientação a
serviço que distigue uma unidade de lógica como um serviço comporada a outras
unidades de serviços que podem existir isoladamente como um objeto ou
componente \cite{erl2009web}.

Após a modelagem conceitual do serviço, os estágios de projeto e desenvolvimento
produzem um serviço que é programa de \textit{software} independente com
características específicas para suportar a realização dos objetivos associados
a computação orientada a serviço.

Cada serviço possui um contexto funcional distinto e é composto de uma lista
de capacidades relacionadas a esse contexto. Então um serviço pode ser
considerado um conjunto de capacidades descritas em seu contrato.


\item [Contrato de serviço] é o conjunto de documentos que expressam as
meta-informações do serviço, sendo a parte fundamental a que descreve a
sua interface técnica. Eles compõem o contrato técnico do serviço, cuja essência
é estabelecer uma API com as funcionalidades providas pelo serviço por meio de
suas capacidades \cite{erl2009web}.

Os serviços implementados como \ws SOAP normalmente são descritos em seu WSDL
\footnote{\ws{} \textit{Description Language}}, \textit{XML schemas} and
políticas (\textit{WS-policy}). Já os serviços implementados como \ws{} REST não
possuem uma linguagem padrão para especificação de contratos. Já foram propostas
algumas alternativas como WADL \cite{hadley2006web}, Swagger \cite{swaggerSite},
e \neoidl{} \cite{lima2015neoidl}.

O contrato de serviço também pode ser composto de documentos de leitura humana,
como os que descrevem níveis de serviços (\textit{SLA}), comportamentos e
limitações. Muitas dessas características também podem ser descritas em
linguagens formais (para processamento computacional).

No contexto de orientação a serviço, o projeto do contrato do serviço é de suma
importância de tal forma que o princípio de projeto contrato de serviço
padronizado é dedicado exclusivamente para se padronizar a criação dos contratos
de serviços \cite{erl2009web}.

\end{description}


\subsection{Modelo arquitetural}
\vspace{-6mm}

O 	

In contrast to conventional software architectures pri-
marily delineating the organization of a system in its
(sub)systems and their interrelationships, the SOA cap-
tures a logical way of designing a software system to
provide services to either end-user applications or other
services distributed in a network through published and
discoverable interfaces. SOA is focused on creating a
design style, technology, and process framework that will
allow enterprises to develop, interconnect, and maintain
enterprise applications and services efficiently and cost-
effectively.
\cite{papazoglou2007serviceApprTechRechIss}


Service orientation provides the underlying implementation that can make an on demand IT
operating environment a reality by supporting the functions of both integration and infra-
structure management [Lymann 2005a].
Service-Oriented Computing (SOC) utilizes services as the constructs to support the devel-
opment of rapid, low-cost and easy composition of distributed applications. Services are
autonomous, platform-independent computational entities that can be used in a platform
independent way. Services can be described, published, discovered, and dynamically as-
sembled for developing massively distributed, interoperable, evolvable systems. Services
perform functions that can range from answering simple requests to executing sophisticated
business processes requiring peer-to-peer relationships between possibly multiple layers of
service consumers and providers. Any piece of code and any application component de-
ployed on a system can be reused and transformed into a network-available service. Serv-
ices reflect a "service-oriented" approach to programming, based on the idea of composing
applications by discovering and invoking network-available services rather than building new
applications or by invoking available applications to accomplish some task [Papazoglou
2003]. Services are most often built in a way that is independent of the context in which
they are used. This means that the service provider and the consumers are loosely coupled.
This "service-oriented" approach is independent of specific programming languages or oper-
ating systems.


\cite{erl2008soaDesigPatterns}


\vspace{-6mm}

\subsection{Princípios SOA}
\label{PrincipiosSOA} 
\vspace{-6mm}

O paradigma de orientação a serviço é estruturada em oito princípios
fundamentais \cite{erl2009web}. São eles que caraterizam a abordagem SOA e a sua
aplicação fazem com que um serviço se diferencie de um componente ou de
um módulo.
Os contratos de serviços permeiam a maior parte destes princípios:

\begin{description}
\item[Contrato padronizado] - \textbf{Serviços dentro de um mesmo inventário
estão em conformidade com os mesmos padrões de contrato de serviço}. 
Os contratos de serviços são elementos fundamentais na arquitetura orientada
a serviço, pois é por meio deles que os serviços interagem uns com os outros e
com potenciais consumidores. Este princípio tem como foco principal o contrato de serviço e seus requisitos. O padrão de projeto \textit{contract firts} é
uma consequência direta deste princípio \cite{erl2009web}. 

\item[Baixo acomplamento] - \textbf{Os contratos de serviços impõem aos
consumidores do serviço requisitos de baixo acoplamento e são, os próprios
contratos, desacoplados do seu ambiente}. 
Este princípio também possui forte relação com o contratos de serviço, pois a
forma como o contrato é projetado e posicionado na arquitetura é que gerará o
benefício do baixo acoplamento. O projeto deve garantir que o contrato
possua tão somente as informações necessárias para possibilitar a compreensão e
o consumo do serviço, bem como não possuir outras características que gerem
acoplamento.

São considerados negativos, e que devem ser evitados, os acoplamentos  
\begin{enumerate}[label=(\alph*)] 
\item do contrato com as funcionalidades que ele suporta, agregando ao
contrato características dos processos que o serviço suporta,
\item do contrato com a sua implementação, invertendo a estratégia de conceber
primeiramente o contrato
\item do contrato com a sua lógica interna, expondo aos consumidores
características que levem os consumidores a inadivertidamente aumentarem o
acoplamento
\item do contrato com a tecnologia do serviço, causando impactos idesejáveis em
caso de substituição de tecnologia.
\end{enumerate}

Por outro lado, há um acoplamento positivo que o que gera dependência da
lógica em relação ao contrato \cite{erl2009web}. Ou seja, idealmente a
implementação do serviço deve ser derivada do contrato, pondendo se ter inclusive a geração de código a
partir do contrato.


\item[Abstração] - \textbf{Os constratos de serviços devem conter apenas
informações essenciais e as informações sobre os serviços são limitadas àquelas
publicadas em seus contratos}. O contrato é a forma oficial a partir da qual o
consumidor do serviço faz seu projeto e tudo o que está além do contrato deve
ser desconhecido por ele. Por um lado este princípio busca a ocultação
controlada de informações. Por outro, visa a simplificação de informações do
contrato de modo a assegurar que apenas informações essenciais estão
disponíveis.


\item[Reusabilidade]- \textbf{Serviços contém e expressam lógica agnóstica e
podem ser disponibilizados como recursos reutilizáveis}. Este princípio
contribui para se entender o serviço como um produto e seu contrato com uma API
genérica para potenciais consumidores. Essa abordagem aplicada ao projeto dos
serviços leva a desenhá-lo com lógicas não dependentes de processos de negócio
específicos, de modo a torná-los reutilizáveis em vários processos.

\item[Autonomia]- \textbf{Serviços exercem um elevado nível de controle sobre
o seu ambiente em tempo de execução}. O controle do ambiente não está ligado a
dependência do serviço à sua plataforma em termos de projeto, mas sim ao aumento
da confiabilidade sobre a execução e redução da dependência dos recursos não se tem controle. 
O que se busca é a previsibilidade sobre o comportamento do serviço.

\item[Ausência de estado] - \textbf{Serviços reduzem o consumo de recursos
restringindo a gestão de estado das informações apenas a quando for necessário}.
Este princípio visa reduzir ou mesmo remover a sobrecarga gerada pelo
gerenciamento do estado de cada operação, aumentando a escalabilidade da
plataforma de arquitetura orientação a serviço como um todo. Na composição do
serviço, o serviço deve armazenar apenas os dados necessários para completar o
processamento, enquanto se aguarda o processamento de outro serviço.

\item[Descoberta de serviço] - \textbf{Serviços devem conter metadados por meio
dos quais os serviços possam ser descobertos e interpretados}. Tornar cada
serviço de fácil descoberta e interpreção pelas equipes de projeto é o foco
deste princípio. Os próprios contratos de serviço devem ser projetados para
incorporar informações que auxiliem na sua descoberta.

\item[Composição] - \textbf{Serviços são participantes efetivos de composição,
independentemente do tamanho ou complexidade da composição}. O princípio da
composição faz com que os projetos de serviços sejam projetados para
possibilitar que eles se tornem participantes de composições. Deve-se levar em
conta, entretanto, os outros princípios no planejamento de uma nova composição,
considerando a complexidade de composições formadas.

\end{description}
 
 
% avaliar a necessidade de falar de design patterns
%\subsubsection{Padrões de projeto }
%\vspace{-6mm}

\ldots

\vspace{-6mm}

\section{Web Services}
\vspace{-6mm}

\ldots

\subsection{SOAP (W3C) }
\vspace{-6mm}

\ldots

\subsubsection{Especificação de contratos}
\vspace{-6mm}

\ldots

\vspace{-6mm}

\subsection{REST (Fielding)}
\label{secaoREST}
\vspace{-6mm}

A tecnologia REST é caracterizada principalmente pela
convenção da adoção dos métodos do protocolo HTTP para definição das operações, transferência de estado nas requisições.
- Ausência de padrão para especificação de contratos REST.
- Fragilidade de garantias por quem consome o serviço;
- Risco de a ausência de informações prejudicarem os consumidores.

\subsubsection{Especificação de contratos}
\vspace{-6mm}

\ldots


\subsubsection{Outros padrões }
\vspace{-6mm}

\ldots

\section{Design-by-Contract}
\label{Design-by-Contract}
\vspace{-6mm}

\ldots

\subsection{Origem}
\vspace{-6mm}

- Bertrand Meyer (eifel)

\subsection{Implementações de DbC}

 - Eifel
 - JML
 - Spec\#



% 3. NeoIDL - Objetivos / Resultados / Avaliação empírica / Extensão da
% Linguagem  / Estudo de caso
\chapter{NEOIDL: LINGUAGEM PARA ESPECIFICAÇÃO DE CONTRATOS REST	}
\vspace{-6mm}


\section{APRESENTAÇÃO}
\label{apresentacaoNeoIDL}

A \neoidl{} é uma linguagem específica de domínio (\textit{Domain Specific
Language - DSL}) desenvolvida com o objetivo de possibilitar e simplificar o
processo de elaboração de contratos para serviços REST. Em seu projeto, foram
consideradas os requisitos de concisão, facilidade de compreensão humana,
extensibilidade e suporte à herança simples dos tipos de dados definidos pelo
usuário.

Além de ser um linguagem, a \neoidl{} é também um \textit{framework} de geração
de código, que permite, a partir de contratos especificados na linguagem, a geração
da implementação do serviço em várias linguagens e tecnologias, por meio de seus
\textit{pluggins}. As próximas subseções apresentam o histórico da \neoidl{},
sua sintaxe e \textit{framework}.


\subsection{Histórico e motivação}
\label{histMotivNeoIDL}
\vspace{-6mm}

A \neoidl{} surgiu no contexto de um projeto de colaboração entre a Universidade
de Brasília e o Exército Brasileiro. O projeto tinha os requisitos de modularidade,
com a lógica distribuída inclusive geograficamente, e de execução em plataformas
diversas. Diante dessa necessidade, o Exército desenvolveu um \textit{framework}
proprietário, voltado para arquitetura orientada a serviço e com suporte
a implantação de serviços REST em vários linguagens, chamado \neocortex{}.

A característica do \neocortex{} de se utilizar serviços implementados em vários
linguagens motivou o desenvolvimento um programa gerador de serviços poliglota
-- que produz código de várias linguagens de progração -- a partir da descrição
do contrato do serviço. Daí nasceu a \neoidl{}.

Entretanto, as linguagens de programação disponíveis para especificação de
contratos REST, como Swagger, WADL e RAML, tinham (e ainda tem) limitações
importantes para a abordagem desejada de se escrever primeiramente o contrato e,
a partir dele, gerar a implementação. Todas elas utilizam linguangens de
propósito geral (XML, JSON, YAML), tornando os contratos extensos e de difícil
compreensão por humanos. Além disso, não possuem mecanismos semânticos de
extensibilidade e modularidade.

Partiu-se então para o desenvolvimento de uma nova linguagem, com sintaxe
inspirada em linguagens mais claras e concisas -- CORBA IDL \cite{corba} e
Apache Thrift \cite{thrift} --, e que permitisse a declaração de tipos de
dados definidos pelo usuário e extensibilidade. Ambas,
CORBA e Apache Thrift, possuem limitações nesses últimos aspectos. A sintaxe e
as características da linguagem \neoidl{} são discutidas na subseção \ref{linguagemNeoIDL}.

Em relação à geração de código, dado o requisito de geração de código para
linguagens distintas, a \neoidl{} foi projetada para possui uma arquitetura
modular, de modo que novas linguagens ou características de implementaçao
pudessem se incorporadas por meio de \textit{plugins} da \neoidl{}. Assim, é
possível desenvolver um novo \textit{plugin} para geração de serviços em outras
linguagens, por exemplo PHP, sem alterar qualquer outro componente, conforme
apresentado na subseção \ref{frameNeoIDL}.

A primeira versão da \neoidl{}, ponto de onde partiu este trabalho, dava suporte
a geração de código em Java, Python e Swagger com as características necessárias
para execução no \neocortex{}. Foram desenvolvidos no decorrer do projeto nove
serviços do domínio de comando e controle \cite{david:commandControl},
compreendendo aproximadamente cinquenta módulos e geração de três mil linhas de
código Python a partir dos contrato especificados em \neoidl{} (contratos são
denominados \texttt{módulos} na \neoidl{}). Outros serviços foram implementados em Java. 



\subsection{\textit{Framework}}
\label{frameNeoIDL}
\vspace{-6mm}

A parte da \neoidl{} responsável pela geração de código de
serviço para as várias linguagens é chamada de \framework{} \neoidl{}. O núcleo do \framework{} é
composto de módulos responsáveis por fazer o \textit{parse} do contrato escrito
em \neoidl{}, por processar a especificação e pelo gerenciamento dos
\textit{plugins}. Já os \textit{plugins} extendem, cada um, a \neoidl{} para as
linguagens de destino. A figura \ref{fig:neoidl-architecture} ilustra a
arquitetura do \framework{}.

\begin{figure}[h]
\begin{center}
\includegraphics[scale=0.6]{neoidl.pdf}
\vspace{-.5cm}
\end{center}
\caption{Arquitetura do framework de geração de código da \neoidl{}.}
\label{fig:neoidl-architecture} 
\end{figure}


O núcleo do \framework{} possui um uma pequena aplicação que carrega as
definições dos \textit{plugins} e processa os argumentos da chamada do gerador
(o arquivo de contrato que terá o código gerado, o diretório de destino e a
linguagem a ser utilizada). O modo de funcionamento geral do \framework{} é
ilustrado na figura \ref{fig:programGenerator}. O \textit{parser} da \neoidl{}
foi construído utilizando \bnfc{} \cite{ranta-bnfc:2012} com a linguagem
funcional Haskell.

\begin{figure*}[bt]
\begin{center}
\includegraphics[scale=0.55,trim=0cm 1.5cm 0cm 0cm]{programgenerator.pdf}
\vspace{-.5cm}
\end{center}
\caption{Gerador de código da \neoidl{}}
\label{fig:programGenerator}
\end{figure*}

As próximas subseções detalham o funcionamento de dois módulos onde estão
contidas os principais trechos da lógica implementada na \neoidl{}: O
\textit{PluginDef} e \textit{PluginLoader}.


\subsubsection{Componente PluginDef}{\label{sec:plugindef}}

O desenvolvimento de \textit{plugins} na \neoidl{} devem obedecer algumas regras
de projeto estabelecidas no componente \texttt{PluginDef}, que é um módulo
escrito em Haskell. No \texttt{PluginDef} declara dois tipo de dados
(\texttt{Plugin} and \texttt{GeneratedFile}) e uma assinatura de tipo
(\texttt{Transform = Module -> [GeneratedFile]}). Eles definem uma família de
tipos que mapeiam um módulo \neoidl{} em uma lista de arquivos com o código
gerado ao final do processo.

De acordo com estas regras de projeto, cada \textit{plugin} precisa declarar uma
instância do tipo \texttt{Plugin} e implementar uma função de acordo com a
assinatura definida em \texttt{Transform}. Além disso, cada instância de  
\texttt{Plugin} precisa ter o nome \texttt{plugin} de forma que o componenente
\texttt{PluginLoader} possa obter os dados necessários para o seu processamento.
Assim, a execução de um \textit{plugin} consiste em aplicar sua função de
transformação correspondente a um módulo \neoidl{} e produzir uma lista de
arquivos de código fonte. 

\vspace{6mm}

\begin{figure}[h]
\begin{small}
\lstinputlisting[language=HaskellSimple,firstnumber=1]{pluginDefStructure.tex}
\vspace{-.5cm}
\end{small}
\caption{Assinatura do \texttt{PluginDef}}
\label{lst:pluginDef}
\end{figure}

\subsubsection{Componente PluginLoader}

O carregamento e validação dos \textit{plugins} são competências do componente
\texttt{PlutinLoader}. O primeito passo é listar todos os arquivos do diretório
do \textit{plugin} a ser carregado, filtrar os arquivos Haskell (extensão
\texttt{``.hs''}), criar um nome para um nome reservado (\textit{qualified
name}) para cada arquivo de \textit{plugin} e aplicar a função de compilação.

Caso todas as regras de definição do \textit{plugin} tenham sido atendidas, o
\textit{plugin} é carregado e estará pronto para ser acionado. Caso contrário,
algumas exceções podem ocorrer, como, por exemplo, não haver definição de
nenhum \textit{plugin} no arquivo:

\begin{tabbing}\tt
~\char36{}\char46{}\char47{}neoIDL\\
\tt ~neoIDL\char58{}~panic\char33{}~\char40{}the~\char39{}impossible\char39{}~happened\char41{}\\
\tt ~~\char40{}GHC~version~7\char46{}6\char46{}3~for~x86\char95{}64\char45{}darwin\char41{}\char58{}\\
\tt ~~~~Not~in~scope\char58{}~\char96{}Plugins\char46{}Python\char46{}plugin\char39{}
\end{tabbing}

Ou ainda em razão de o \textit{plugin} não ser uma instância do tipo
\texttt{Plugin}:

\begin{tabbing}\tt
~\char36{}\char46{}\char47{}neoIDL\\
\tt ~neoIDL\char58{}~panic\char33{}~\char40{}the~\char39{}impossible\char39{}~happened\char41{}\\
\tt ~\char40{}GHC~version~7\char46{}6\char46{}3~for~x86\char95{}64\char45{}darwin\char41{}\char58{}\\
\tt ~~Couldn\char39{}t~match~expected~type~\char96{}Plugin\char39{}\\
\tt ~~~with~actual~type~\char96{}\char91{}GHC\char46{}Types\char46{}Char\char93{}\char39{}
\end{tabbing}



\subsection{Linguagem}
\label{linguagemNeoIDL}
\vspace{-6mm}

A \neoidl{} simplifica a especificação de contratos REST pois possui uma sintaxe
concisa própria de linguagens de especificação de interfaces (\emph{interface
description languages}). Ademais, a \neoidl{} provê mecanismos de
modularização e herança, de forma que os contratos possam ser separados em
módulos \footnote{Os arquivos utilizados para definição de contratos
são denominados módulos na \neoidl{}}, facilitando a herança e manutenção dos
contratos.

Para demonstrar como os módulos são estruturados na
\neoidl{}, a seguir são apresentados alguns trechos de um serviço hipotético
de envio de mensagens. Na primeira parte, um módulo faz uma definição de tipo
de dado; em seguida, um segundo módulo, orientado a serviço, importa as
definições do primeiro para então declarar as especificações das operações do
serviço. Por fim, o módulo de serviço é acrescentado de uma anotação como
forma de extender as características da operação.

\vspace{6mm}

\begin{figure}[h]
\begin{small}
\lstinputlisting[language=NeoIDL,firstnumber=1]{mensagemData.tex}
\vspace{-.5cm}
\end{small} 
\caption{Tipos de dados definidos na \neoidl}
\label{lst:messagedata-neo}
\end{figure}

O trecho ilustrado na figura \ref{lst:messagedata-neo} faz a definição de dois
tipos de dados. \emph{MessageType}, declarado no linha 2, é uma estrutura
simples do tipo enumeração. O outro tipo é \emph{Message}, declarado entre as
linhas 4 e 11, composto de seis atributos. O atributo \emph{type} de
\emph{Message} é do tipo \emph{MessageType} recém declarado.

Na \neoidl{} é utiliza a abordagem convenção sobre configuração, de modo que
todos os atributos declarados são obrigatórios, a menos que explicitamente
seja declarado diferente. O atributo \emph{subject} do tipo \emph{Message} é um
exemplo de atributo opcional (\texttt{<Type> <Ident> = 0;}).

O módulo seguinte (figura \ref{lst:sentmessage-neo}), na linha proposta pela
\neoidl{} para suporte a herança e reuso, importa o conjunto de definições de \emph{MessageData} e declara o
serviço \emph{sendbox} (linha 4), o qual possui duas capacidades. A
capacidade \emph{sendMessage} (linha 6) utiliza a operação \method{post} para submeter uma
mensagem (tipo \emph{Message}). A outra capacidade (linha 7) tem a finalidade de
listas as mensagens com um determinado sequecial, por meio da operação
\method{get}.

Por fim, a instrução \emph{path} indica o caminho (URI) onde as
operações serão disponibilizadas. Esse atribuito é importante para se definir
como as requisições serão roteadas entre os serviços.

\vspace{6mm}

\begin{figure}[htb]
\begin{small}
\lstinputlisting[language=NeoIDL,firstnumber=1]{mensagem.tex}
\end{small}
\caption{Sent message service specification in \neoidl}
\label{lst:sentmessage-neo}
\end{figure}

Ainda na filosofia de conversão sobre configuração, a \neoidl{} assume que os
argumentos das operações \method{POST} e \method{PUT} são enviadas no corpo da
requisição. Nas operações \method{GET} e \method{DELETE}, por outro lado,
presume-se que os agumentos estão contidos no \emph{path} da requisição ou ainda
como \textit{query string}.

A especificação dos contratos na \neoidl{} pode ser enriquecida com
anotações, por meio das quais se possibilita extender a semântica de uma
especificação sem que seja necessário alterar a sintaxe da \neoidl{}. Esse
recurso da \neoidl{} simplifica a inclusão de novas características aos
serviços, pois a alteração da sintaxe da própria \neoidl{} envolve um esfoço não
trivial de compatibilizar todos os \textit{plugins} já construídos.

O módulo apresentado na figura \ref{lst:annotationNeoIDL} contém, além das
informações contidas no módulo da figura \ref{lst:sentmessage-neo}, uma anotação chamada
\emph{Security Policy} (linhas 4 a 6) aplicada ao serviço \emph{sentbox}. A
declaração da anotação é feita no final do módulo (linhas 14 a 18).

\vspace{6mm}
 
\begin{figure}
\begin{small}
\lstinputlisting[language=NeoIDL,firstnumber=1]{messageSendAnnotation.tex}
\vspace{-.5cm}
\end{small}
\caption{Especificação de anotação na \neoidl{}}
\label{lst:annotationNeoIDL}
\end{figure}

Além de aplicáveis a \texttt{resources}, as anotações também podem ser aplicadas
a outros construtores da linguagem: \texttt{module}, \texttt{enum},
\texttt{entity}. Qualquer anotação na \neoidl{} possui a mesma estrutura: um
nome, um elemento alvo e uma lista propriedades. Todas estas informações ficam
disponíveis para utilização pelos \textit{plugins}.

As próxima subseções apresentam a estrutura sintática e léxica \neoidl{} e
refletem a situação de Agosto de 2015, antes da incorporação das construções
relativas a \designbycontract{} (estas informações foram geradas 
automaticamente pelo BNF-Converter \cite{forsberg-bnfc:2004} \textit{parser
generator}). 

\subsubsection{A estrutura léxica da \neoidl{}}\label{sub:lexico}

\begin{enumerate}
  \item Identificadores
  
  Identificadores \nonterminal{Ident} são literais (\textit{strings}) não
  delimitadas que começam com uma letra seguida por letras, números e os caracteres {\tt \_ '},
  exceto palavras reservadas.
  
  \item Literais
  
  Literais de texto são cadeias de caracteres  \nonterminal{String}\ com a forma
  \terminal{``}$x$\terminal{``}, onde $x$ é qualquer sequencia de caracter,
  exceto \terminal{``}, a menos que precedido por \text{\tt \char92{}}.
  
  Literais numéricos \nonterminal{Int}\ são sequências não vazias de números.
  
  Literais de ponto flutuantes \nonterminal{Double}\ tem a estrutura
  definida pela seguinte expressão regular: $\nonterminal{digit}+ \mbox{{\it
  `.'}} \nonterminal{digit}+ (\mbox{{\it `e'}} \mbox{{\it `-'}}?
  \nonterminal{digit}+)?$, ou seja, duas sequências de números separadas por um
  ponto, opcionalmente precedida de um símbolo de negativo.

  \item Palavras reservadas e símbolos

  O conjunto de palavras reservadas são os terminais da gramática da linguagem
  \neoidl{}. As palavras reservadas não compostas de letras são chamados
  símbolos, que são tratados de forma diferente dos identificadores. A
  sintaxe analisador léxico segue regras típicas de linguagens como Haskell, C e
  Java, incluindo correspondência mais longa e convenções para o espaço.
  
  As palavras reservadas utilizadas na \neoidl{} são as seguintes: \\

\begin{tabular}{lll}
{\reserved{annotation}} &{\reserved{call}} &{\reserved{entity}} \\
{\reserved{enum}} &{\reserved{extends}} &{\reserved{float}} \\
{\reserved{for}} &{\reserved{import}} &{\reserved{int}} \\
{\reserved{module}} &{\reserved{path}} &{\reserved{resource}} \\
{\reserved{string}} & & \\
\end{tabular}\\
  
Os símbolos utilizados na \neoidl{} são os seguintes: \\

\begin{tabular}{lll}
{\symb{\{}} &{\symb{\}}} &{\symb{;}} \\
{\symb{{$=$}}} &{\symb{.}} &{\symb{@}} \\
{\symb{(}} &{\symb{)}} &{\symb{0}} \\
{\symb{{$=$}{$=$}}} &{\symb{{$<$}{$>$}}} &{\symb{{$>$}}} \\
{\symb{{$>$}{$=$}}} &{\symb{{$<$}}} &{\symb{{$<$}{$=$}}} \\
{\symb{[}} &{\symb{]}} &{\symb{@get}} \\
{\symb{@post}} &{\symb{@put}} &{\symb{@delete}} \\
{\symb{/@require}} &{\symb{/@ensure}} &{\symb{/@invariant}} \\
{\symb{/@otherwise}} &{\symb{/**}} &{\symb{*/}} \\
{\symb{*}} &{\symb{@desc}} &{\symb{@param}} \\
{\symb{@consume}} &{\symb{,}} & \\
\end{tabular}\\

\end{enumerate}

\subsubsection{A estrutura sintática da NeoIDL}\label{sub:sintatico}

Não-terminais são delimitados entre $\langle$ e $\rangle$. O símbolo {\arrow}
(produto), {\delimit} (união) e {\emptyP} (regra vazia) advêm da notação BNF.
Todos os demais símbolos são terminais.\\

\begin{small}
\begin{tabular}{lll}
\label{lst:BNFnot}
{\nonterminal{Modulo}} {\arrow} {\terminal{module}} {\nonterminal{Ident}} {\terminal{\{}} \\ 
 \quad {\nonterminal{ListImport}} \\ 
 \quad {\nonterminal{MPath}} \\ 
 \quad {\nonterminal{ListEnum}} \\ 
 \quad {\nonterminal{ListEntity}} \\ 
 \quad {\nonterminal{ListResource}} \\ 
 \quad {\nonterminal{ListDecAnnotation}} \\ 
{\terminal{\}}}  \\
\end{tabular}

\begin{tabular}{lll}
{\nonterminal{Import}} & {\arrow}  &{\terminal{import}} {\nonterminal{NImport}} {\terminal{;}}  \\
\end{tabular}

\begin{tabular}{lll}
{\nonterminal{MPath}} & {\arrow}  &{\emptyP} \\
 & {\delimit}  &{\terminal{path}} {\terminal{{$=$}}} {\nonterminal{String}} {\terminal{;}}  \\
\end{tabular}

\begin{tabular}{lll}
{\nonterminal{NImport}} & {\arrow}  &{\nonterminal{Ident}}  \\
 & {\delimit}  &{\nonterminal{Ident}} {\terminal{.}} {\nonterminal{NImport}}  \\
\end{tabular}

\begin{tabular}{lll}
{\nonterminal{Entity}} & {\arrow}  &{\nonterminal{ListDefAnnotation}} {\terminal{entity}} {\nonterminal{Ident}} {\terminal{\{}} {\nonterminal{ListProperty}} {\terminal{\}}} {\terminal{;}}  \\
 & {\delimit}  &{\nonterminal{ListDefAnnotation}} {\terminal{entity}} {\nonterminal{Ident}} {\terminal{extends}} {\nonterminal{Ident}} {\terminal{\{}} {\nonterminal{ListProperty}} {\terminal{\}}} {\terminal{;}}  \\
\end{tabular}

\begin{tabular}{lll}
{\nonterminal{Enum}} & {\arrow}  &{\terminal{enum}} {\nonterminal{Ident}} {\terminal{\{}} {\nonterminal{ListValue}} {\terminal{\}}} {\terminal{;}}  \\
\end{tabular}

\begin{tabular}{lll}
{\nonterminal{DecAnnotation}} & {\arrow}  &{\terminal{annotation}} {\nonterminal{Ident}} {\terminal{for}} {\nonterminal{AnnotationType}} {\terminal{\{}} {\nonterminal{ListProperty}} {\terminal{\}}} {\terminal{;}}  \\
\end{tabular}

\begin{tabular}{lll}
{\nonterminal{DefAnnotation}} & {\arrow}  &{\terminal{@}} {\nonterminal{Ident}} {\terminal{(}} {\nonterminal{ListAssignment}} {\terminal{)}} {\terminal{;}}  \\
\end{tabular}

\begin{tabular}{lll}
{\nonterminal{Parameter}} & {\arrow}  &{\nonterminal{Type}} {\nonterminal{Ident}} {\nonterminal{Modifier}}  \\
\end{tabular}

\begin{tabular}{lll}
{\nonterminal{Assignment}} & {\arrow}  &{\nonterminal{Ident}} {\terminal{{$=$}}} {\nonterminal{Value}}  \\
\end{tabular}

\begin{tabular}{lll}
{\nonterminal{Modifier}} & {\arrow}  &{\emptyP} \\
 & {\delimit}  &{\terminal{{$=$}}} {\terminal{0}}  \\
\end{tabular}

\begin{tabular}{lllllllll}
{\nonterminal{AnnotationType}} & {\arrow}  &{\terminal{resource}}  
 & {\delimit}  &{\terminal{enum}}  
 & {\delimit}  &{\terminal{entity}}  
 & {\delimit}  &{\terminal{module}} 
\end{tabular}

\begin{tabular}{lll}
{\nonterminal{Resource}} & {\arrow}  &{\nonterminal{ListDefAnnotation}} {\terminal{resource}} {\nonterminal{Ident}} {\terminal{\{}} {\terminal{path}} {\terminal{{$=$}}} {\nonterminal{String}} {\terminal{;}} {\nonterminal{ListCapacity}} {\terminal{\}}} {\terminal{;}}  \\
\end{tabular}

\begin{tabular}{lll}
{\nonterminal{Capacity}} & {\arrow}  &{\nonterminal{NeoDoc}} {\nonterminal{ListDefNAnnotation}} {\nonterminal{Method}} {\nonterminal{Type}} {\nonterminal{Ident}} {\terminal{(}} {\nonterminal{ListParameter}} {\terminal{)}} {\terminal{;}}  \\
\end{tabular}

\begin{tabular}{lllllllll}
{\nonterminal{Method}} & {\arrow}  &{\terminal{@get}} 
 & {\delimit}  &{\terminal{@post}} 
 & {\delimit}  &{\terminal{@put}}  
 & {\delimit}  &{\terminal{@delete}} 
\end{tabular}\\
\end{small}    



\section{AVALIAÇÃO EMPÍRICA}
\vspace{-6mm}

A primeira versão da \neoidl{} foi submetida a uma avaliação empírica de sua
expressividade e reuso em um contexto real. As próximas subseções apresentam 
o resultado da análise comparativa da representação de 44 (quarenta e quatro)
contratos escritos em Swagger em relação à mesma especificação em \neoidl{}.
Esse estudo foi realizado no transcurso do trabalho de mestrado e 
publicado no periódico IJSeke \cite{lima2015neoidl}.

\subsection{Expressividade}
\vspace{-6mm}

A \neoidl{} é uma DSL, conforme apresentado na seção \ref{apresentacaoNeoIDL},
e como tal se destina a atender a um propósito específico, nem mais, nem menos
\cite{hudak1998modular}. A \neoidl{} foi projetada para permitir a
especificação de contratos de serviços REST de forma mais expressiva e concisa,
facilitando-se a escrita e leitura por humanos (mais detalhes na subseção
\ref{histMotivNeoIDL}).

Programas escritos em DSLs costumam ser mais fáceis de escrever e,
consequentemente, mais fáceis de se manter comparavelmente a programas escritos
em linguagens de propósito geral \cite{hudak1998modular}. Isso se deve
justamente ao fato de a DSL tratar apenas um conjunto reduzido de situações e
problemas, fazendo com ela seja, muitas vezes, mais acessível ao público geral
\cite{taha2008domain}.

A expressividade é um dos principais critérios para se escolher uma
linguagem. Entretanto a linguagem que não expressa todas as situações
necessárias ao seu contexto de uso, por óbvio, não podem ser usadas
\cite{mackinlay1985expressiveness}. Nesse sentido, o primeiro teste que a
\neoidl{} passou foi a produção de contratos e serviços reais no início do
projeto com o Exército Brasileiro (vide subseção \ref{histMotivNeoIDL}).

Assim, tendo a \neoidl{} demonstrado sua capacidade de representar contratos
REST, foi realizada uma segunda análise: quão expressiva seria a \neoidl{}
em comparação com outra linguagem com o mesmo objetivo. Foi escolhida
Swagger \cite{swaggerSite}, uma linguagem cujo uso tem crescido pela indústria.
Em Swagger, os contratos são escritos em JSON\cite{JSon} ou YAML\cite{YAML},
ambos com uma estrutura geral de chave-valor.

Tendo sido a \neoidl{} desenvolvida para ser mais fácil de compreender e
manipular por humanos, foi adotada a estratégia de comparar a expressividade em
termos de quantidade de linhas de código (SLOC - do inglês \textit{Source Lines of Code}), uma vez que
muitas linhas significam maior esforço para escrita, sobretudo na abordagem
\CtFirst{}. 









In a collaborative work with the Brazilian Army,
we obtained a portion of their contracts' specifications in Swagger (44 in total), specified with version 1.2.
Our first step was to rewrite these specifications in \neoidl{} and thus compare the total
number of lines of code (which might serve as a metric of expressiveness).
The 44 contracts in Swagger amount to 13921 lines of specification, while the same set of contracts in NeoIDL comprises 5140 lines of specification. Thus, the average reduction was about 63\%. In others words, it means that 10 lines of structured Swagger specification require
about 4 lines of \neoidl{} specification. In this analysis we only considered \emph{physical lines of code}, ignoring 
blank lines and lines consisting of delimiters only. The Appendix A shows a sample contract we analised.

The reduction in number of lines is not the same in all contracts. For instance, a given service\footnote{For confidentiality reasons, the real names of contracts were omitted.} required 367 lines of Swagger specification and 112 lines of \neoidl{} specification. This case  
represents a reduction of about 69\%. On the other hand, another service contract required 
81 lines of specification in Swagger and 42 lines of \neoidl{} specification. In this case, the SLOC decrease was slightly less than 50\%.

The size of the original contract has only a small influence in the observed expressiveness. 
Therefore, we cannot assume that \emph{the bigger the contract is in Swagger 
the bigger is the improvement (with respect to the smaller specification size) of \neoidl}. We 
also realized that the use of a more descritive documentation, the number of entities, and the number of capacities do not correlate to the advantageous reduction of lines of code during a 
transformation of Swagger specifications into \neoidl{} specifications. Therefore, it seems that 
the benefits do not relate to the size of the original specifications.  
Table~\ref{tab:size-corr} presents the correlation between the improvement of 
expressiveness (measured as the percentage of reduction obtained 
after transforming Swagger specifications into \neoidl{} specifications) and 
some metrics related to the size of the original Swagger specifications.

\begin{table}[htb]
\caption{Correlation of the Expressiveness Improvement with the size of the Swagger specifications}
\begin{center}
\begin{tabular}{lrr} 
\toprule
Metric & Pearson's correlation & \emph{p-value} \\ \hline \hline 
LOC of Swagger specification & 0.19 &  0.20 \\ 
Number of services & 0.14 & 0.35 \\ 
Number of capacities & 0.14 & 0.34 \\
Number of entities & 0.20 & 0.18 \\ \bottomrule 
\end{tabular} 
\end{center}
\label{tab:size-corr}
\end{table}





\subsection{Potencial de reuso}



Similar to \neoidl{}, Swagger presents 
some mechanisms to reuse user defined structures. Nevertheless, 
this feature is almost ignored in the set of contracts we analysed, which 
leads to the duplication of entities' definition across different 
Swagger specifications. This might have occurred either due to the 
nonintuitive construct for reusing definitions in Swagger (based on 
references to JSON files) and the difficulties to identify 
that one entity had already been specified in another contract.
After analysing the 44 Swagger specifications, we realized that 40 entities 
have been specified in  more than one contract. Actually, one specific 
entity is present in 12 distinct Swagger contracts. 




 
\section{EXTENSÃO DA NEOIDL PARA DESIGN BY CONTRACT}

Influência de Eiffel, JML e Spec\#

Eiffel assertions are Boolean expressions, with a few extensions such as the old
notation. Since the whole power of Boolean expressions is available, they may
include function calls. Because the full power of the language is available to
write these functions, the conditions they express can be quite sophisticated.
\cite{meyer1992applying}

 
 
\subsection{Proposta: Serviços com Desing-by-Contract}
\vspace{-6mm}

Os benefícios esperados pela adoção da arquitetura orientada a serviços
somente serão auferidos com a concepção adequada de cada serviço. 
Por essa razão, é necessário planejar o projeto dos serviços criteriosamente
antes de lançar mão do desenvolvimento, com preocupação especial em garantir
um nível aceitável de estabilidade aos consumidores de cada serviço.
Nessa etapa do projeto de desenho da solução, a especificação do contrato do
serviço (Web API) exerce uma função fundamental. 

Na sociedade civil, contratos são meios de se formalizar acordo entre partes a
fim de definir os direitos e deveres de cada parte e buscar atingir o
objetivo esperado dentro de determinadas regras. Cada parte espera que as outras
cumpram com suas obrigações.
Por outro lado, sabe-se que o descumprimeto das obrigações costuma implicar de
penalizações até o desfazimento do contrato. 

Contratos entre serviços Web seguem em uma linha análoga. O desenho das
capacidades (operações) e dos dados das mensagens correspondem aos
termos do contrato no sentido do que o consumidor deve esperar do serviço
provedor. Porém identificou-se, após ampla pesquisa realizada sobre o tema, que
as linguagens disponíves para especificação de contratos atingem apenas esse
nível de garantias. No contexto de webservices em REST, conforme descrito na
seção \ref{secaoREST}, há ainda a ausência de padrão para especificação
contratos, tal como ocorre com o WSDL adotado em SOAP.

A proposta deste trabalho é extender os níveis de garantias, de modo a promover
um patamar adicional com obrigações mutuas entre os serviços (consumidor e
provedor). Isso se dá para adoção do conceito de Design-by-Contract (debatido
na seção \ref{Design-by-Contract}) em que a execução da
capacidade do serviço garantirá a execução, desde que satisfeitas as condições
prévias. O detalhamento do processo é exposto nas seções que se seguem.

\vspace{-6mm}

\subsubsection{Modelo de operação}
\vspace{-6mm}

As garantias para execução dos serviços são estabelecidas em duas etapas: pré- e
pós-condições. Nas pré-condições o provedor do serviço estabelece os requisitos
para que o serviço possa ser executado. A etapa de pós-condições tem o papel de
validar se a mensagem de retorno do serviço possui resultados válidos.

O diagrama da Figura \ref{FigServiceDbC} descreve como ocorre a operação das
pré- e pós-condições. O processo se inicia com a chamada à capacidade do serviço e a
identificação da existência de uma pré-condição. Caso tenham sido estabelecidas 
pré-condições, essas são avaliadas. Caso alguma delas não tenham sido
satisfeitas, o serviço principal não é processado e o provedor do serviço
retornar o código de falha definido no contrato correspondente.


\begin{figure}[!htb]
\centering
\includegraphics[width=\textwidth,trim = 0mm 5mm 0mm 0mm,clip]{ServiceDbC.png}
\caption{Digrama de atividades com verificação de pré e pós condições}
\label{FigServiceDbC}
\end{figure}

Caso tenham sido definidas pós-condições, essas são acionadas após o
processamento da capacidade, porém antes do retorno ao consumidor do serviço.
Assim, conforme Figura \ref{FigServiceDbC}, visando não entregar ao cliente uma
mensagem ou situação incoerente, as pós-condições são validadas. Caso todas as
pós-condições tenham sido satisfeitas, a mensagem de retorno é encaminhada ao
cliente. Caso contrário, será retornado o código de falha.


\vspace{-6mm}


\subsubsection{Verificação das pré-condições}
\vspace{-6mm}

As pré-condições podem ser do tipo baseado nos parâmetros da requisição ou do
tipo baseado na chamada a outro serviço. Denominamos, para o contexto desta
dissertação, de básica a pré-condição baseada apenas nos parâmetros da
requisição (atributos da chamada ao serviço). Nessa validação é direta,
comparando os valores passados com os valores admitidos. 

No caso das pré-condições baseadas em serviços, é realizada chamada a outro
serviço para verificar se uma determinada condição é satisfeita. Este modo de
funcionamento, que se assemelha a uma composição de serviço, é mais versátil, pois permite
validações de condições complexas sem que a lógica associada seja conhecida pelo
cliente. Assim, os contratos que estabelecem esse tipo de
pré-condição se mantem simples.

A Figura \ref{FigServicePrecondition} detalha as etapas de verificação de cada
pré-condição. Nota-se que a saída para as situações de desatendimento às
pré-condições, independentemente do tipo, é o mesmo. O objetivo desta abordagem
é simplificar o tratametno de exceção no consumidor.

\begin{figure}[!htb]
\centering
\includegraphics[width=\textwidth,trim = 0mm 5mm 0mm 0mm,clip]{PreconditionValidation.png}
\caption{Diagrama de atividades do processamento da pré-condição}
\label{FigServicePrecondition}
\end{figure}


\subsubsection{Verificação das pós-condições}
\vspace{-6mm}

\begin{figure}[!htb]
\centering
\includegraphics[width=\textwidth,trim = 0mm 5mm 0mm
0mm,clip]{PostconditionValidation.png} 
\vspace{-6mm}
\caption{Diagrama de atividades do processamento da pós-condição}
\label{FigServicePostcondition}
\end{figure}

A verificação das pós-condições acontece de modo muito similar a das
pré-condições. Há também os dois tipos, baseado em valores e em chamadas a
outros serviços. O diferencial está em que a validação dos valores passa a
ocorrer a partir dos valores contidos na mensagem de retorno. A Figura
\ref{FigServicePostcondition} descreve as etapas necessárias para validação de
cada pré-condição.

	
	
	
	

\subsection{Extensão da linguagem}

\subsubsection{Precondição básica}

\begin{figure}[htb]
\begin{small}
\lstinputlisting[language=NeoIDL,firstnumber=1]{DBCsimple.neo}
\end{small}
\caption{Exemplo da notação DBC básica na \neoidl{}}
\label{lst:DBCService}
\end{figure} 

\subsubsection{Pós-condição básica}

\ldots



\subsubsection{Precondição com chamada a serviço}

\begin{figure}[htb]
\begin{small}
\lstinputlisting[language=NeoIDL,firstnumber=1]{DBCservice.neo}
\end{small}
\caption{Exemplo da notação DBC na \neoidl{} com chamada a serviço}
\label{lst:DBCService}
\end{figure} 


\subsubsection{Pós-condição com chamada a serviço}

\ldots

\subsection{Estudo de caso: plugin twisted}

\ldots

\subsubsection{Arquitetura}

% Diagrama da estrutura do código gerado


\subsubsection{Geração de código}
% 4. Análise da aceitação - Método / GQM / Resultados
\chapter{CONTRATOS REST COM DESIGN-BY-CONTRACT}


\section{PROPOSTA: SERVIÇOS COM DESIGN-BY-CONTRACT}
\label{PropostaServicoDbC}
\vspace{-6mm}

Os benefícios esperados pela adoção da arquitetura orientada a serviços
somente serão auferidos com a concepção adequada de cada serviço. 
Por essa razão, é necessário planejar o projeto dos serviços criteriosamente
antes de lançar mão do desenvolvimento, com preocupação especial em garantir
um nível aceitável de estabilidade aos consumidores de cada serviço.
Nessa etapa do projeto de desenho da solução, a especificação do contrato do
serviço (Web API) exerce uma função fundamental. 

Na sociedade civil, contratos são meios de se formalizar acordo entre partes a
fim de definir os direitos e deveres de cada parte e buscar atingir o
objetivo esperado dentro de determinadas regras. Cada parte espera que as outras
cumpram com suas obrigações.
Por outro lado, sabe-se que o descumprimeto das obrigações costuma implicar de
penalizações até o desfazimento do contrato. 

Contratos entre serviços Web seguem em uma linha análoga. O desenho das
capacidades (operações) e dos dados das mensagens correspondem aos
termos do contrato no sentido do que o consumidor deve esperar do serviço
provedor. Porém identificou-se, após ampla pesquisa realizada sobre o tema, que
as linguagens disponíves para especificação de contratos atingem apenas esse
nível de garantias. No contexto de webservices em REST, conforme descrito na
seção \ref{secaoREST}, há ainda a ausência de padrão para especificação
contratos.

A proposta deste trabalho é extender os níveis de garantias, de modo a promover
um patamar adicional com obrigações mutuas entre os serviços (consumidor e
provedor). Isso se dá para adoção do conceito de \designbycontract{} (debatido
na seção \ref{Design-by-Contract}) em que a execução da
capacidade do serviço garantirá a execução, desde que satisfeitas as condições
prévias. As próximas subseções detalham o modo de operação dos serviços com as
construções de \designbycontract{}.

\vspace{-6mm}

\subsection{Modelo de operação}
\vspace{-6mm}

As garantias para execução dos serviços são estabelecidas em duas etapas: pré e
póscondições. Nas precondições o provedor do serviço estabelece os requisitos
para que o serviço possa ser executado pelo cliente. A etapa de pós-condições
tem o papel de validar se a mensagem de retorno do serviço possui os resultados
esperados.

O diagrama da Figura \ref{FigServiceDbC} descreve como ocorre a operação das
pré e pós-condições. O processo se inicia com a chamada à capacidade do serviço e a
identificação da existência de uma precondição. Caso tenham sido estabelecidas 
precondições, essas são avaliadas. Caso alguma delas não tenham sido
satisfeitas, o serviço principal não é processado e o provedor do serviço
retornar o código de falha definido no contrato correspondente.


\begin{figure}[!htb]
\centering
\includegraphics[width=\textwidth,trim = 0mm 5mm 0mm 0mm,clip]{ServiceDbC.png}
\caption{Digrama de atividades com verificação de pré e pós condições}
\label{FigServiceDbC}
\end{figure}

Caso tenham sido definidas pós-condições, essas são acionadas após o
processamento da capacidade, porém antes do retorno ao consumidor do serviço.
Assim, conforme Figura \ref{FigServiceDbC}, visando não entregar ao cliente uma
mensagem ou situação incoerente, as pós-condições são validadas. Caso todas as
pós-condições tenham sido satisfeitas, a mensagem de retorno é encaminhada ao
cliente. Caso contrário, será retornado o código de falha definido para a
pós-condição violada.

\subsubsection{Observação sobre invariantes}
\vspace{-6mm}

Em \designbycontract{}, além dos conceitos de pré e pós-condições,
há também a ideia de invariantes\cite{meyer1997object}. Quando aplicadas a uma classe na
orientação a objetos, as invariantes estabelecem restrições sobre o estado
armazenado nos objetos instanciados dessa classe. No contexto de orientação a
serviços, tem-se por princípio a ausência de estados dos serviços, descrito na
seção \ref{PrincipiosSOA}. Por essa razão, no estudo sobre a incorporação de
\designbycontract{} em contratos de serviços, as invariantes não foram
consideradas.


\subsection{Verificação das precondições}
\vspace{-6mm}

As precondições podem ser do tipo baseado nos parâmetros da requisição ou do
tipo baseado na chamada a outro serviço. Denomina-se, no contexto desta
dissertação, de básica a precondição baseada apenas nos parâmetros da
requisição (atributos da chamada ao serviço). Essa validação é direta,
comparando os valores passados com os valores admitidos. 

No caso das precondições baseadas em serviços, é realizada chamada a outro
serviço para verificar se uma determinada condição é satisfeita. Este modo de
funcionamento, que se assemelha a uma composição de serviço, é mais versátil, pois permite
validações de condições complexas sem que a lógica associada seja conhecida pelo
cliente. Assim, os contratos que estabelecem esse tipo de
precondição se mantem simples.

A Figura \ref{FigServicePrecondition} detalha as etapas de verificação de cada
precondição. Nota-se que a saída para as situações de desatendimento às
precondições, independentemente do tipo, é o mesmo. O objetivo desta abordagem
é simplificar o tratametno de exceção no consumidor.

\begin{figure}[!htb]
\centering
\includegraphics[width=\textwidth,trim = 0mm 5mm 0mm
0mm,clip]{img/PreconditionValidation.png}
\caption{Diagrama de atividades do processamento da precondição}
\label{FigServicePrecondition}
\end{figure}


\subsection{Verificação das pós-condições}
\vspace{-6mm}

\begin{figure}[!htb]
\centering
\includegraphics[width=\textwidth,trim = 0mm 5mm 0mm
0mm,clip]{img/PostconditionValidation.png} 
\vspace{-6mm}
\caption{Diagrama de atividades do processamento da pós-condição}
\label{FigServicePostcondition}
\end{figure}

A verificação das pós-condições acontece de modo muito similar a das
precondições. Há também os dois tipos, baseado em valores e em chamadas a
outros serviços. O diferencial está em que a validação dos valores passa a
ocorrer a partir dos valores contidos na mensagem de retorno. A Figura
\ref{FigServicePostcondition} descreve as etapas necessárias para validação de
cada precondição.


\section{EXTENSÃO DA NEOIDL PARA DESIGN-BY-CONTRACT}
\label{extensaoNeoIDL-DbC}

A sintaxe escolhida para possibilitar a especificação de pré e pós condições
na \neoidl{} foi influenciada por três linguagens e extensões de linguagens de
programação: Eiffel, JML e Spec\# (exemplificadas na subseção
\ref{implementDbC}). 

Em Eiffel, as asserções são expressões booleanas, de modo que uma pré e uma
poscondição podem ter resultado verdadeiro ou falso. As asserções também podem
incluir chamadas a funções, extendendo a validações a lógicas mais sofisticadas
\cite{meyer1992applying}. Essas caracteríticas, por serem simples e versáteis,
foram consideradas adequadas e incorporadas à especificação de
\designbycontract{} em contratos de serviços na \neoidl{}.

A primeira sintaxe de \designbycontract{} na \neoidl{} teve como
base a sintaxe da JML, especialmente em como se associar as pré e poscondições a
cada serviço ou capacidade, assemelhando-se a comentários e iniciados pelo
símbolo de arroba (@). A figura \ref{lst:precondicaoJML-neo} apresenta um
exemplo de especificação de precondição seguindo a linha da JML.

\vspace{6mm}

\begin{figure}[h]
\begin{small}
\lstinputlisting[language=NeoIDL,firstnumber=1]{trechos_codigo/DBCsimple.neo}
\vspace{-.5cm}
\end{small} 
\caption{Forma preliminar de precondição na \neoidl}
\label{lst:precondicaoJML-neo}
\end{figure}

Essa forma foi apresentada no Workshop de Teses e Dissertações do CBSoft em 2015
\cite{lima2015contratos}, ainda nos primeiros estágios do trabalho. Os revisores
apontaram dificuldade de distinguir, na especificação, entre as pré e
poscondições e as capacidades, pois possuiam prefixos muito
semelhantes (ver linhas 6 a 8).
Essas críticas impulsionaram a busca por outra sintaxe mais adequada aos elementos textuais já
existentes na \neoidl{}.

Spec\# possui uma forma de especificação de asserções em que as pré e
poscondições são declaradas logo após a assinatura do método ou classe, apenas
com o uso das palavras reservadas \emph{require} e \emph{ensure}, sem uso de
símbolos. Essa abordagem foi aplicada à \neoidl{} para versão final da sintaxe
com suporte a \designbycontract{}.

As próximas subseções apresentam alguns exemplos de especificação de pre e
poscondições na \neoidl{} e as mudanças introduzidas na sintaxe da linguagem.
Inicialmente as condições de \designbycontract{} são demonstradas separadamente
e, ao final, a subseção \ref{SintaxeGeralDbc} consolida o conjunto
de novos elementos sintáticos e como eles são estruturados.


\subsection{Precondição básica}
\label{precondicaoBasica}

Uma precondição básica é a que valida os valores recebidos na
requisição, comparando-os com os valores estabelecidos na instrução
\emph{require} do contrato. Esse tipo de precondição assemelha-se a validação
dos atributos recebidos por um método no paradigma de orientação a objetos.

A origem das informações, isto é, onde os valores que serão validados se
encontram, depende da operação HTTP utilizada. A subseção \ref{FonteDadosDbc} descreve como esses
dados são obtidos. Para realizar a comparação do valor recebido com o valor
esperado, a \neoidl{} admite seis operadores de comparação (Subseção
\ref{TiposContrDbC}).

A Figura \ref{lst:DBCPreCondBasica} (linhas 7 e 8) apresenta um exemplo de
precondição básica em uma forma simples, em que apenas um valor é testado (id)
e, caso a condição não seja satisfeita, a instrução \emph{otherwise} indica o
valor a ser retornado (código HTTP Not Found). 	

\begin{figure}[htb]
\begin{small}
\lstinputlisting[language=NeoIDL,firstnumber=1]{trechos_codigo/store_pre_basica.neo}
\end{small}
\caption{Exemplo de notação de precondição básica na \neoidl{}}
\label{lst:DBCPreCondBasica}
\end{figure} 



\subsection{Pós-condição básica}

Em termos sintáticos, as pós-condições básicas possuem uma forma muito
semelhante às precondições básicas (\ref{precondicaoBasica}), diferindo-se
exclusivamente pelo uso da instrução \emph{ensure}. A Figura
\ref{lst:DBCPosCondBasica} (linhas 8 e 9) mostra um exemplo de pós-condição
básica em que, após a execução da operação \method{GET}, se o valor do atributo
\emph{quantity} não for maior que zero, então o serviço não foi executado adequadamente e a exceção
(\emph{otherwise}) é retornada.

\begin{figure}[htb]
\begin{small}
\lstinputlisting[language=NeoIDL,firstnumber=1]{trechos_codigo/store_pos_basica.neo}
\end{small}
\caption{Exemplo de notação de pós-condição básica na \neoidl{}}
\label{lst:DBCPosCondBasica}
\end{figure} 


\subsection{Precondição com chamada a serviço}

As precondições baseadas em serviços seguem uma sequência que envolve a chamada
a outro serviço antes do processamento do serviço principal, em um tipo simples de
composição de serviço. Essa abordagem permite que precondições complexas sejam
validadas por serviços especializados, sem que a especificação do contrato
de serviço seja complexa. Essa proposta preserva ainda a ideia original de
Eiffel\cite{meyer1988eiffel}, de que pré e pós-condições sejam expressões
\emph{booleanas}.

A primeira etapa do processo de execução da precondição de serviço consiste em
fazer a chamada a um serviço (ver Figura \ref{FigServicePrecondition}) por meio de uma
operação \method{GET}. Em seguida, o código de \textit{status} retornado pelo
serviço da precondição é comparado com o valor especificado na precondição do contrato.

Após o acionamento do serviço da precondição, o comportamento é mesmo da
precondição básica (ver \ref{precondicaoBasica}). Caso a precondição seja
satisfeita, é retornado o valor indicado pela instrução \emph{otherwise}. As
precondições de serviço na \neoidl{} admitem os mesmos operadores de comparação
que as precondições básicas.

A Figura \ref{lst:DBCPreCondServico} ilustra a especificação de uma precondição
do tipo serviço (linhas 8 e 9). Assim, antes de executar a operação \method{POST}
do serviço principal, o serviço \emph{store.getOrder} é acionado. Caso esse
serviço retorno o código correspondente a \emph{HTTP Not Found}, a operação
\method{POST} é executada. Caso contrário, o serviço principal retorna o valor
correspondente a \emph{HTTP Invalid Precondition}, em razão do estabelecido no
\emph{otherwise}.


\begin{figure}[htb]
\begin{small}
\lstinputlisting[language=NeoIDL,firstnumber=1]{trechos_codigo/store_pre_servico.neo}
\end{small}
\caption{Exemplo de notação de precondição com chamada a serviço na
\neoidl{}} 
\label{lst:DBCPreCondServico}
\end{figure} 



\subsection{Pós-condição com chamada a serviço}
\label{Pos-condicao servico}

A pós-condição com chamada a serviço seguem a sequência de eventos indicada na
Figura \ref{FigServicePostcondition}. No caso da pós-condição, a execução do
serviço principal já ocoreu e a função do serviço na pós-condição é validar se a
execução do serviço principal ocorreu com sucesso. Algumas pós-condições são
naturais, como as que verificam se um objeto foi inserido após a operação de
inclusão (método \method{POST}). Ou ainda, a que verifica se o objeto foi excluído
após uma operação \method{DELETE}.

No exemplo da Figura \ref{lst:DBCPosCondServico}, o serviço principal faz a
exclusão de um objeto \textit{Order}. A pós-condição (linha 8) verifica, após o
processamento do \method{DELETE}, se o objeto foi efetivamente apagado por meio do
serviço \emph{store.getOrder}. Se o serviço da pós-condição retornar o valor
\emph{HTTP Not Found}, o objeto foi adequadamente excluído. Caso contrário, o
serviço principal retornará o valor \emph{HTTP Not Modified}, o qual foi
estabelecido na instrução \textit{otherwise}.

\begin{figure}[htb]
\begin{small}
\lstinputlisting[language=NeoIDL,firstnumber=1]{trechos_codigo/store_pos_servico.neo}
\end{small}
\caption{Exemplo de notação de pós-condição com chamada a serviço na
\neoidl{}}
\label{lst:DBCPosCondServico}
\end{figure} 


\subsection{Sintaxe geral de pré e pós-condições}
\label{SintaxeGeralDbc}

Entre as subseções \ref{precondicaoBasica} e \ref{Pos-condicao servico} foram
apresentados separadametne exemplos simples de especificação de pre e condições
na \neoidl{} de modo a facilitar a compreensão. Esta subseção demonstra
a estruturação sintática das construções de \designbycontract{} agregadas
\neoidl{} por este trabalho.

\subsubsection{Listas de pré e pós-condições}

Um módulo \neoidl{} possui uma seção para declaração dos serviços (ver Figura
\ref{fig:moduloNeoIDL}). Cada serviço declarado na \neoidl{} pode ter um mais
capacidades, as quais correspondem às operações HTTP utilizadas na arquitetura
REST. Sintaticamente, um serviço possui uma lista de capacidades\footnote{Os
símbolos ``[`` e ``]'' identificam uma lista.}:

\begin{center}
\hilight{\{\texttt{Serviço [Capacidade]}\}}
\end{center}

A sintaxe da \neoidl{} foi extendida para admitir a vinculação de pré e
pós-condições às capacidades. Essas contruções de \designbycontract{} são
opcionais, ou seja, uma capacidade pode não ter nenhuma pré ou pós-condição. Por
outro lado, pode-se incluir nas capacidades mais de uma precondição e mais de
uma pós-condição ou ainda qualquer combinação delas, simultâneamente.

\begin{center}
\hilight{\{\texttt{Capacidade \textcolor{red}{[Condição DbC]}}\}}
\end{center}

É possível ainda, caso uma precondição ou pós-condição se aplique a todas as
capacidades de um serviço, é possível declará-la para o serviço como um todo,
logo antes da declaração das capacidades:

\begin{center}
\hilight{\{\texttt{Serviço \textcolor{red}{[Condição DbC]} [Capacidade]}\}}
\end{center}

 Assim, generalizando, a \neoidl{} passou a suportar a seguinte sintaxe:
 
\begin{center}
\hilight{\{\texttt{Serviço \textcolor{red}{[Condição DbC]} [Capacidade
\textcolor{red}{[Condição DbC]} ]}\}}
\end{center}
 

\subsubsection{Tipos de construções de \designbycontract{}}
\label{TiposContrDbC}

Conforme exemplificado entre as subseções \ref{precondicaoBasica} e
\ref{Pos-condicao servico}, as construções de \designbycontract{} possuem as
seguintes estruturas:

\begin{center}
\hilight{Condição básica: \textbf{TipoCondição} \{\texttt{[Argumento Comparação
Valor]}\}}
\hilight{Condição serviço: \textbf{TipoCondição} \{\texttt{Serviço (Parâmetro)
Comparação ValorSrv}\}}
\hilight{Exceção: \textbf{Otherwise} \{\texttt{Valor de Retorno	}\}}

\end{center}

Onde:
\vspace{-6mm}
\begin{description}
\item [TipoCondição] indica se é uma precondição (\emph{require}) ou uma
pós-condição (\emph{ensure});

\item [Argumento] indica o nome do atributo que será testado;

\item [Comparação] indica o operação de comparação que a ser utilizada. A
\neoidl{}  admite seis operadores de comparação: igualdade (\literal{==}), 
diferença (\literal{<>}), maior (\literal{>}), maior ou igual (\literal{>=}), 
menor (\literal{<}) e menor ou igual (\literal{<=}). Eles podem ser aplicados  a
qualquer tipo de pré ou poscondição.

Mais de um atributo pode ser testado em uma pré ou pós-condição. Na expressão
acima, essa característica é simbolizada com a indicação de uma lista.

\item [Valor] corresponde ao valor que será comparado com o \textbf{Argumento}.
As pré e pós-condições básicas admitem algumas combinações com os operadores
\emph{not}, \emph{and} e \emph{or}, formando expressões booleanas e,
assim, permitindo estabelecer regras mais abrangentes.

Por exemplo, a precondição apresentada na Figura \ref{lst:DBCPreCondBasica} poderia
ser escrita como \emph{\textbf{require} (\textbf{not} id \literal{<=} 0)}. Os
operadores \emph{and} e \emph{or} são infixos (ex.: \emph{\textbf{require}
((id \literal{>} 0 \textbf{or} id \literal{<=} -1000))}). O uso do operador
\emph{and} produz o mesmo efeito de se declarar duas precondições.

\item [Serviço] indica o nome do serviço que será acionado. As informações para
indicação concreta da localização do serviço, ou seja, a URL de acionamento, não
são indicados nesse atributo de pre e pós-condição. Essas informações dependem
do local de implantação do serviço e não convém que estejam declaradas no
contrato, sob pena de que mudanças no ambiente de implantação do serviço gerem
impacto ao contrato.
Entretanto, caso se queira fazer essa vinculação, pode ser criada uma anotação
\neoidl{}, com essa finalidade, para o serviço.

\item [Parâmetro] tem a função de indicar os valores que serão passados para a
chamada ao serviço da pré ou pós-condição.

 \item [ValorSrv] corresponde a um valor que será comparado com o
 \textit{Status Code HTTP} retornado pelo serviço. Na \neoidl{}, esses valores
 são convencionados como o nome do corresponde ao código HTTP, com palavras
 justapostas (Ex.: "NotFound" tem o valor do código 404, de HTTP \textit{Not
 Found})

\item [Valor de Retorno] é utilizado na cláusula \emph{otherwise} para indicar o
valor que o serviço principal deve retornar caso uma precondiçaõ ou pós-condição
não seja atendida.

\end{description}


\subsubsection{Um último exemplo de módulo com \designbycontract{}}

A Figura \ref{lst:ModuloNeoVariosDbC} apresenta um exemplo módulo \neoidl{} de
um serviço completo, utilizando alguns dos recursos apresentados na subseção
\ref{TiposContrDbC}. Esse serviço possui três operações, cada uma com duas
condições de \designbycontract{}.

A operação \method{GET} possui uma precondição em que o valor do atributo \emph{id} deve ser maior que zero (linha 8), caso contrário a operação deve retornar o valor
correspondente a "NotFound". Uma pós-condição assegura que, se o serviço for
executado adequadamente, o valor de \emph{quantity} será maior que zero (linha
9). Se for violada a pós-condição, a operação \method{GET} retorna o valor
"NoContent".

A operação \method{POST} possui duas precondições (linhas 13 e 14). A primeira,
básica, valida o valor do atributo \emph{quantity}. A segunda, aciona o serviço
\emph{store.getOrder} com o parâmetro \emph{id}. Nessa operação, foi
definido um valor de exceção único para as duas precondições. A \neoidl{}
associa essa instrução de \emph{otherwise} geral às instruções anteriores que
não possuem condição e exceção específica (caso da linha 8).

Na operação \method{DELETE} foi utilizado o operador lógico \emph{and} na
precondição (linha 18). A pós-condição faz a chamada ao serviço
\emph{store.getOrder}. Tanto a pré como a pós-condição possuem o mesmo valor de
exceção (linha 20). 

\begin{figure}[htb]
\begin{small}
\lstinputlisting[language=NeoIDL,firstnumber=1]{trechos_codigo/store_dbc_complete.neo}
\end{small}
\caption{Exemplo de módulo \neoidl{} com várias instruções de \designbycontract{}}
\label{lst:ModuloNeoVariosDbC}
\end{figure} 	



\subsection{Fontes de dados para pré e pós-condições}
\label{FonteDadosDbc}

Os serviços REST na \neoidl{} seguem uma convenção em relação ao conteúdo
presente na requisição e na resposta para cada tipo de operação \emph{HTTP},
conforme descrito na subseção \ref{linguagemNeoIDL}. Por exemplo, o método \method{GET}
submete os argumentos pelo \emph{path} ou como \textit{query string} da
requisição e não em seu corpo. Esses aspectos foram considerados para se definir
a fonte das informações para os argumentos e parâmetros utilizados nas pré e
pós-condições.

A Figura \ref{Fig:FonteDadosDbcNeoIDL} resume a origem para cada operação. Há
diferenças também entre os tipos básico e baseado em serviços das construções de
\designbycontract{}. As operações \method{GET} e \method{DELETE} não possuem
argumentos encaminhados no corpo da requisição. Para essas operações, os
argumentos são retirados do \emph{path} ou \emph{query string} para validação
das precondições, tanto no tipo básico como no tipo baseado em composição. Na
figura, essa origem é identificada como \textit{Request arguments}.

\begin{figure}[!htb]
\centering
\includegraphics[width=120mm,trim = 0mm 0mm 0mm
0mm,clip]{img/FonteDadosDbcNeoIDL.pdf}
\caption{Diagrama da fonte de dados para acionamento de pré e pós-condições}
\label{Fig:FonteDadosDbcNeoIDL}
\end{figure}


Por outro lado, as operações \method{POST} e \method{PUT} submetem os dados a
serem inseridos ou alterados pelo corpo da requisição (\textit{Request body}),
local de onde as precondições básicas e baseadas em serviços devem extrair os
parâmetros de validação.
A \neoidl{} utiliza a notação JSON\cite{JSon} no corpo das requisições. A
representação dos argumentos utiliza o padrão separado por pontos para indicar
elementos mais internos (ex. "Pessoa.Nome").

No caso das pós-condições, a origem das informações diverge entre o tipo
básicos e o tipo por serviços. A única operação que admite pós-condição básica é
a operação \method{GET}, em que os argumentos de validação são extraídos do
corpo da resposta (\textit{Response body}). As demais operações não retornam dados
no corpo da requisição, pois estão voltadas para alteração de informações e não
para consultas. Além disso, não há utilidade em se validar dados de requisição
em uma pós-condição.

Já as pós-condições baseadas em serviços não suportam a operação \method{GET}.
Embora seja tecnicamente possível acionar um serviço com base nos argumentos da
requisição, como não se tem modificação nos dados por essa operação, a
validação dessas informações pode meio de serviço deve ser feita nas
precondições. Ademais, as pós-condições básicas da operação \method{GET} cumprem
com o objetivo que validar as informações de saída.

A pós-condição da operação \method{DELETE} pode acionar serviços com base nos
argumentos da requisição (\textit{Request arguments}). Um caso típico é de
acionar o serviço de consulta para verificar se o dado foi efetivamente
excluído. As operações \method{POST} e \method{PUT} podem acionam serviços em
pós-condições utilizando as informações do corpo da requisição (\textit{Request
body}), uma vez que essas operações não possuem dados no corpo da resposta. 



\section{ESTUDO DE CASO: PLUGIN TWISTED}
\label{pluginTwisted}

\ldots

\subsection{Arquitetura}

\begin{figure}[!htb]
\centering
\includegraphics[width=120mm,trim = 18mm 4mm 0mm
0mm,clip]{img/TwistedServer.pdf}
\caption{Arquitetura do serviço Twisted gerado pela \neoidl{}}
\label{Fig:TwistedArchtecture}
\end{figure}


\subsection{Modo de funcionamento}

\begin{figure}[!htb]
\centering
\includegraphics[width=80mm,trim = 5mm 6mm 0mm 
3mm,clip]{img/TwistedFilters.pdf}
\caption{Modo de operação das pré e pós-condições no serviço Twisted}
\label{Fig:TwistedFiltes}
\end{figure}


\subsection{Geração de código}

\begin{figure}[!htb]
\centering
\includegraphics[width=110mm,trim = 0mm 0mm 0mm 
0mm,clip]{img/PluginTwisted.pdf}
\caption{Plugin para geração de código Twisted com suporte a \designbycontract{}}
\label{PluginTwisted}
\end{figure}


\section{ESTUDO EMPÍRICO DA ANÁLISE SUBJETIVA} 
\vspace{-6mm}

\label{analiseSubjetiva}
\vspace{-6mm}

%%%%
A language's expressiveness is the major criterion for choosing a language to
state a given set of facts: a language that cannot express the facts should not
be used. However, additional criteria are needed to choose among languages that
are sufficiently expressive  for a set of facts. Two of these criteria are how
ease it is to state the facts in the language and how easy is to perceive the
facts once they are stated.

expressividade de uma linguagem é o principal critério para a escolha de uma
linguagem para indicar um determinado conjunto de fatos: uma linguagem que não
podem expressar os fatos não devem ser usados. No entanto, os critérios
adicionais são necessários para escolher entre os idiomas que são
suficientemente expressivo para um conjunto de fatos. Dois desses critérios são
quão fácil é expor os fatos na linguagem e como é fácil de perceber os fatos,
uma vez que são demonstrados.

\cite{mackinlay1985expressiveness}

%%%%


%%%
Instead of aiming to be the best for solving any kind of
computing problem, DSLs aim to be particularly good for
solving a specific class of problems, and in doing so they
are often much more accessible to the general public than tra-
ditional programming languages.
\cite{taha2008domain}
%%%% 

%%%%
They offer substantial gains in expressiveness and ease of use compared
with GPLs in their domain of application’. [2] describes the typical costs of a
DSL, noting that a small extra initial investment in a DSL implementation typ-
ically leads to long term savings, in comparison to alternative routes.
\cite{tratt2008evolving}
%%%%

%%%%
A domain specific language (DSL) is a program-
ming language tailored for a particular application do-
main. Characteristic of an effective DSL is the ability
to develop complete application programs for a do-
main quickly and effectively. A DSL is not (neces-
sarily) “general purpose.” Rather, it should capture
precisely the semantics of an application domain, no
more and no less.

There are lots of advantages to using DSLs, start-
ing with the fact that programs are generally easier to
write, reason about, and modify compared to equivalent 
programs written in general purpose languages.
Indeed, these are the same advantages gained from using any high-level
programming language.
\cite{hudak1998modular}
%%%%




\subsection{Método}
\vspace{-6mm}

\subsection{GQM}
\vspace{-6mm}

\subsection{Questionário}
\vspace{-6mm}

\subsection{Análise dos Resultados}
\vspace{-6mm}


* Questionário montado para avaliar a utilidade de DbC com NeoIDL

* Inicialmente motivado pelo estudo do Alessandro Garcia

* GQM e Avaliação TAM

* Montagem do questionário

1. Perfil técnico-profissional do respondente
1.1 Para qual órgão ou empresa você presta serviços atualmente?
1.2 A quanto tempo você trabalha com desenvolvimento Web
1.3 A quanto tempo você desenvolve com uso de APIs Web (Web Service)
1.4 Qual o seu nível de experiência com especificação de API REST
1.5 Qual o seu nível de experiência com especificação de contratos com Swagger

3 Questões sobre especificação e implementação de APIs Web
3.1 A especificação do contrato formalmente, seja em Swagger ou NeoIDL, em
relação a descrição textual, aumentará meu nível de acerto na implementação (efetividade).
3.2 Identificar e compreender as operações e atributos na especificação Swagger
é simples para mim.
3.3 Identificar e compreender as operações e atributos na especificação NeoIDL é
simples para mim.

4. DbC
4.1 Conhecer previamente e explicitamente as precondições será útil para mim.
(Useful)
4.2 Aprender a identificar as precondições na NeoIDL parece ser simples pra mim
(Easy to learn)
4.3 Parece ser fácil para mim declarar uma precondição na NeoIDL
(Clear and understandable)
4.4 Me lembrar da sintaxe da precondição na NeoIDL é fácil  (Remember)

5. Geração de código
5.1 É claro e compreensível para mim o efeito da precondição sobre o código
gerado  (Controllable)
5.2 A geração do código de pré e pós-condições aumentará minha produtividade na
implementação do serviço (Job performance)
5.3 Assumindo ter a disposição a NeoIDL no meu trabalho, para especificação de
contratos e geração de código, eu presumo que a utilizarei regularmente no futuro.
5.4 Nesse mesmo contexto, eu vou preferir utilizar contratos escritos em NeoIDL
do que descritos de outra forma



* Distribuição do questionário

% \subsubsection{Avaliação dos resultados}
% 
% \begin{figure*}[h]
% \begin{center}
% \includegraphics[scale=0.55,trim=0cm 1.5cm 0cm
% 0cm]{img/ResultQuestions06a09.pdf}  
% \end{center}
% \caption{Módulo \neoidl{}}
% \label{fig:moduloNeoIDL}
% \end{figure*}

* Ameaças
- não foi fornecido nenhum material sobre a NeoIDL, apresentando somente o uma
descrição de serviço
- O questionário foi aplicado uma única vez, sem melhorias a partir do primeiro
conjunto de respostas




* Questionários futuros

A principal questão de pesquisa a ser avaliada com o uso do questionário é a utilidade em se agregar ao design das especificações de serviços REST as garantias de pré e
pós-condições. Em segundo momento, pressupondo a utilidade, avaliar se a NeoIDL
cumpre satisfatoriamente com este propósito, agregando à sintaxe da linguagem
a possibilidade de se expressar pré e pós-condições.

* Separar os respondentes em faixas de experiência. Verificar se as respostas
dos menos experientes precisam ser descartadas pela pouca capacidade crítica.
Separar a análise entre os respondentes que conhecem e os que não conhecem
Swagger.

* Perspectivas de comparação
a) Experiência com desenvolvimento com uso de REST
b) Experiência com Swagger
c) Utilidade da especificação formal de contratos
d) Percepção da NeoIDL sem DbC






% 5. Conclusão e trabalhos futuros
\chapter{CONCLUSÕES E TRABALHOS FUTUROS}
\vspace{-6mm}

\section{CONCLUSÕES}
\vspace{-6mm}

A necessidade de estratégias de integração de sistemas e soluções adaptáveis às
constantes necessidade des mudanças tem levado às empresas a, cada vez mais,
adotarem o modelo de computação orientada a serviços -- SOC
\cite{papazoglou2008service} \cite{erl2009web}. A qualidade da especificação do
serviço por meio de seu contrato é um dos fatores determinantes para o sucesso
do uso de SOC.

A \neoidl{} foi criada para ser uma alternativa às linguagens de especificação
de \wss REST, uma vez que estas possuem uma sintaxe pouco expressiva para humanos,
além de não disporem de mecanismos com suporte a extensibilidade e modularização. O estudo
empírico da comparação entre especificações Swagger e \neoidl{} demonstrou a capacidade
expressiva e potencial de reuso da \neoidl{}. 
Entretanto, nem a \neoidl{}, nem as demais linguagens possibilitavam especificar 
contratos robustos, como os existentes em linguagens com suporte a \designbycontract{}.

Esse trabalho apresentou uma extensão da \neoidl{} para possibilitar a especificação
de contratos REST com suporte a pré e pós-condições e, a partir do contrato, permitir
a geração de código de serviços REST com essas garantias, seguindo a abordagem
\CtFirst{}. A proposta se baseou no paradgima de orientação a objetos, uma das 
principais influências da orientação a serviços.

Essa proposta foi submetida ao \textit{feedback} de profissionais experientes e também ao 
\textit{Workshop} de teses de dissertações do WTDSoft 2015. Os elementos sintáticos
de linguagens com suporte a \designbycontract{} como Eiffel, JML e Spec\# foram avaliados
e proporcionaram a criação de novas construções sintáticas para a \neoidl{} mantendo a harmonia com 
a linguagem pre-existente, sem que se perdesse o potencial para criação de regras de validação flexíveis e abrangentes.

A \neoidl{} passou a permitir a validação dos parâmetros de entrada e saída de uma requisição (por meio
de pré e pós-condições básicas) assim como permitir também fazer requisição a outros serviços para realizar validações mais
complexas, por meio de pequenas composições de serviços. A construção de um \textit{Plugin Twisted} demonstrou ser
viável produzir código que realize, em tempo de execução, a validação das regras estabelecidas no contrato, sem que o 
desenvolvedor tenha que se preocupar com elas e direcione seu esforço para a implementação das regras
de negócio.

A avaliação subjetiva, realizada por meio de questionário baseado nas técnicas GQM e TAM, apresentou
resultados satisfatórios à hipótese de pesquisa, em que grande parte dos respondentes indicou que a \neoidl{} com suporte
a \designbycontract{} é uma ferramenta com potencial de adoção, demonstrando ser uma linguagem e um 
\framework{} úteis e fáceis de serem utilizados sob a ótica de quem escreve contratos e implementa
serviços. 

Ficou demonstrado, no contexto de avaliação desse trabalho, que os conceitos de \designbycontract, quais sejam,
expressar direitos e obrigações entre os clientes e fornecedores de recursos, proporcionando qualidade na
análise, projeto, implementação e comunicação, são também aplicáveis no paradigma de orientação
a serviços. Ademais, que serviços baseados em \wss REST podem ser simples e leves sob a ótica
da especificação mas também serem robustos sobre a ótica da estabilidade e comportamento.


\section{TRABALHOS FUTUROS}
\vspace{-6mm}



- Trabalhos relacionados com hipermedia




\clearpage
\newpage
%%%%%%%%%%%%%%%%%%%%%%%% Bibliografia %%%%%%%%%%%%%%%%%%%%%%%

%Espa�amento simples
%\baselineskip=12pt

% espa�amento 1,5
\linespread{1.3}

 \bibliographystyle{plainbr}

%\bibliographystyle{ieee}

\addcontentsline{toc}{chapter}{REFERÊNCIAS BIBLIOGRÁFICAS}

\bibliography{bibdata}

%%%%%%%%%%%%%%
% \input{biblio}

\baselineskip=18pt
%%%%%%%%%%%%%%%%%%%%%%%%% Ap�ndices
\appendix



\clearpage

\addcontentsline{toc}{chapter}{APÊNDICES}
\hspace{1mm}

\vfill

\begin{large}

\begin{center}
\begin{bf}
APÊNDICES

% T�TULO DO PRIMEIRO CAP�TULO DO AP�NDICE

\end{bf}
\end{center}
\end{large}

\vfill

\clearpage
 \chapter{TRANFORMAÇÃO SWAGGER PARA NEOIDL}
\label{scriptSwagger2NeoIDL}

\begin{small}
\scriptsize
\lstinputlisting[language=Perl,firstnumber=1]{trechos_codigo/swagger2NeoIDL.tex}
\end{small}
 \chapter{ESTRUTURA DA LINGUAGEM NEOIDL}
\label{apend:estruturaLexicaNeoIDL}

Estas informações foram geradas automaticamente pelo BNF-Converter \cite{forsberg-bnfc:2004}
\textit{parser generator}) a partir da gramática da \neoidl{}.

\section{ESTRUTURA LÉXICA DA NEOIDL}

\begin{enumerate}
  \item Identificadores
  
  Identificadores \nonterminal{Ident} são literais (\textit{strings}) não
  delimitadas que começam com uma letra seguida por letras, números e os caracteres {\tt \_ '},
  exceto palavras reservadas.
  
  \item Literais
  
  Literais de texto são cadeias de caracteres  \nonterminal{String}\ com a forma
  \terminal{``}$x$\terminal{``}, onde $x$ é qualquer sequencia de caracter,
  exceto \terminal{``}, a menos que precedido por \text{\tt \char92{}}.
  
  Literais numéricos \nonterminal{Int}\ são sequências não vazias de números.
  
  Literais de ponto flutuantes \nonterminal{Double}\ tem a estrutura
  definida pela seguinte expressão regular: $\nonterminal{digit}+ \mbox{{\it
  `.'}} \nonterminal{digit}+ (\mbox{{\it `e'}} \mbox{{\it `-'}}?
  \nonterminal{digit}+)?$, ou seja, duas sequências de números separadas por um
  ponto, opcionalmente precedida de um símbolo de negativo.

  \item Palavras reservadas e símbolos

  O conjunto de palavras reservadas são os terminais da gramática da linguagem
  \neoidl{}. As palavras reservadas não compostas de letras são chamados
  símbolos, que são tratados de forma diferente dos identificadores. A
  sintaxe analisador léxico segue regras típicas de linguagens como Haskell, C e
  Java, incluindo correspondência mais longa e convenções para o espaço.
  
  As palavras reservadas utilizadas na \neoidl{} são as seguintes: \\

\begin{tabular}{lll}
{\reserved{annotation}} &{\reserved{call}} &{\reserved{entity}} \\
{\reserved{enum}} &{\reserved{extends}} &{\reserved{float}} \\
{\reserved{for}} &{\reserved{import}} &{\reserved{int}} \\
{\reserved{module}} &{\reserved{path}} &{\reserved{resource}} \\
{\reserved{string}} & & \\
\end{tabular}\\
  
Os símbolos utilizados na \neoidl{} são os seguintes: \\

\begin{tabular}{lll}
{\symb{\{}} &{\symb{\}}} &{\symb{;}} \\
{\symb{{$=$}}} &{\symb{.}} &{\symb{@}} \\
{\symb{(}} &{\symb{)}} &{\symb{0}} \\
{\symb{{$=$}{$=$}}} &{\symb{{$<$}{$>$}}} &{\symb{{$>$}}} \\
{\symb{{$>$}{$=$}}} &{\symb{{$<$}}} &{\symb{{$<$}{$=$}}} \\
{\symb{[}} &{\symb{]}} &{\symb{@get}} \\
{\symb{@post}} &{\symb{@put}} &{\symb{@delete}} \\
{\symb{/@require}} &{\symb{/@ensure}} &{\symb{/@invariant}} \\
{\symb{/@otherwise}} &{\symb{/**}} &{\symb{*/}} \\
{\symb{*}} &{\symb{@desc}} &{\symb{@param}} \\
{\symb{@consume}} &{\symb{,}} & \\
\end{tabular}\\

\end{enumerate}

\section{ESTRUTURA SINTÁTICA DA NEOIDL}\label{sub:sintatico}

Não-terminais são delimitados entre $\langle$ e $\rangle$. O símbolo {\arrow}
(produto), {\delimit} (união) e {\emptyP} (regra vazia) advêm da notação BNF.
Todos os demais símbolos são terminais.\\


\begin{small}

\begin{tabular}{lll}
\label{lst:BNFnot}
{\nonterminal{Modulo}} {\arrow} {\terminal{module}} {\nonterminal{Ident}} {\terminal{\{}} \\ 
 \quad {\nonterminal{ListImport}} \\ 
 \quad {\nonterminal{MPath}} \\ 
 \quad {\nonterminal{ListEnum}} \\ 
 \quad {\nonterminal{ListEntity}} \\ 
 \quad {\nonterminal{ListResource}} \\ 
 \quad {\nonterminal{ListDecAnnotation}} \\ 
{\terminal{\}}}  \\
\end{tabular}


\begin{tabular}{lll}
{\nonterminal{Import}} & {\arrow}  &{\terminal{import}} {\nonterminal{NImport}} {\terminal{;}}  \\
\end{tabular}

\begin{tabular}{lll}
{\nonterminal{MPath}} & {\arrow}  &{\emptyP} \\
 & {\delimit}  &{\terminal{path}} {\terminal{{$=$}}} {\nonterminal{String}} {\terminal{;}}  \\
\end{tabular}

\begin{tabular}{lll}
{\nonterminal{NImport}} & {\arrow}  &{\nonterminal{Ident}}  \\
 & {\delimit}  &{\nonterminal{Ident}} {\terminal{.}} {\nonterminal{NImport}}  \\
\end{tabular}

\begin{tabular}{lll}
{\nonterminal{Entity}} & {\arrow}  &{\nonterminal{ListDefAnnotation}} {\terminal{entity}} {\nonterminal{Ident}} {\terminal{\{}} {\nonterminal{ListProperty}} {\terminal{\}}} {\terminal{;}}  \\
 & {\delimit}  &{\nonterminal{ListDefAnnotation}} {\terminal{entity}} {\nonterminal{Ident}} {\terminal{extends}} {\nonterminal{Ident}} {\terminal{\{}} {\nonterminal{ListProperty}} {\terminal{\}}} {\terminal{;}}  \\
\end{tabular}

\begin{tabular}{lll}
{\nonterminal{Enum}} & {\arrow}  &{\terminal{enum}} {\nonterminal{Ident}} {\terminal{\{}} {\nonterminal{ListValue}} {\terminal{\}}} {\terminal{;}}  \\
\end{tabular}

\begin{tabular}{lll}
{\nonterminal{DecAnnotation}} & {\arrow}  &{\terminal{annotation}} {\nonterminal{Ident}} {\terminal{for}} {\nonterminal{AnnotationType}} {\terminal{\{}} {\nonterminal{ListProperty}} {\terminal{\}}} {\terminal{;}}  \\
\end{tabular}

\begin{tabular}{lll}
{\nonterminal{DefAnnotation}} & {\arrow}  &{\terminal{@}} {\nonterminal{Ident}} {\terminal{(}} {\nonterminal{ListAssignment}} {\terminal{)}} {\terminal{;}}  \\
\end{tabular}

\begin{tabular}{lll}
{\nonterminal{Parameter}} & {\arrow}  &{\nonterminal{Type}} {\nonterminal{Ident}} {\nonterminal{Modifier}}  \\
\end{tabular}

\begin{tabular}{lll}
{\nonterminal{Assignment}} & {\arrow}  &{\nonterminal{Ident}} {\terminal{{$=$}}} {\nonterminal{Value}}  \\
\end{tabular}

\begin{tabular}{lll}
{\nonterminal{Modifier}} & {\arrow}  &{\emptyP} \\
 & {\delimit}  &{\terminal{{$=$}}} {\terminal{0}}  \\
\end{tabular}

\begin{tabular}{lllllllll}
{\nonterminal{AnnotationType}} & {\arrow}  &{\terminal{resource}}  
 & {\delimit}  &{\terminal{enum}}  
 & {\delimit}  &{\terminal{entity}}  
 & {\delimit}  &{\terminal{module}} 
\end{tabular}

\begin{tabular}{lll}
{\nonterminal{Resource}} & {\arrow}  &{\nonterminal{ListDefAnnotation}} {\terminal{resource}} {\nonterminal{Ident}} {\terminal{\{}} {\terminal{path}} {\terminal{{$=$}}} {\nonterminal{String}} {\terminal{;}} {\nonterminal{ListCapacity}} {\terminal{\}}} {\terminal{;}}  \\
\end{tabular}

\begin{tabular}{lll}
{\nonterminal{Capacity}} & {\arrow}  &{\nonterminal{NeoDoc}} {\nonterminal{ListDefNAnnotation}} {\nonterminal{Method}} {\nonterminal{Type}} {\nonterminal{Ident}} {\terminal{(}} {\nonterminal{ListParameter}} {\terminal{)}} {\terminal{;}}  \\
\end{tabular}

\begin{tabular}{lllllllll}
{\nonterminal{Method}} & {\arrow}  &{\terminal{@get}} 
 & {\delimit}  &{\terminal{@post}} 
 & {\delimit}  &{\terminal{@put}}  
 & {\delimit}  &{\terminal{@delete}} 
\end{tabular}\\
\end{small}    
 \chapter{CLASSES DO PACOTE DBCCONDITIONS}
\label{classesDbcCondition} 
\vspace{-6mm}

\begin{small}
\scriptsize
\lstinputlisting[language=PythonTwisted,firstnumber=1]{trechos_codigo/dbcCondition.py}
\end{small}

% fim do texto


\end{document}
