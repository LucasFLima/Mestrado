\chapter{NEOIDL: LINGUAGEM PARA ESPECIFICAÇÃO DE CONTRATOS REST	}
\vspace{-6mm}


\section{APRESENTAÇÃO}
\label{apresentacaoNeoIDL}

A \neoidl{} é uma linguagem específica de domínio (\textit{Domain Specific
Language - DSL}) desenvolvida com o objetivo de possibilitar e simplificar o
processo de elaboração de contratos para serviços REST. Em seu projeto, foram
consideradas os requisitos de concisão, facilidade de compreensão humana,
extensibilidade e suporte à herança simples dos tipos de dados definidos pelo
usuário.

Além de ser um linguagem, a \neoidl{} é também um \textit{framework} de geração
de código, que permite, a partir de contratos especificados na linguagem, a geração
da implementação do serviço em várias linguagens e tecnologias, por meio de seus
\textit{pluggins}. As próximas subseções apresentam o histórico da \neoidl{},
sua sintaxe e \textit{framework}.


\subsection{Histórico e motivação}
\label{histMotivNeoIDL}
\vspace{-6mm}

A \neoidl{} surgiu no contexto de um projeto de colaboração entre a Universidade
de Brasília e o Exército Brasileiro. O projeto tinha os requisitos de modularidade,
com a lógica distribuída inclusive geograficamente, e de execução em plataformas
diversas. Diante dessa necessidade, o Exército desenvolveu um \textit{framework}
proprietário, voltado para arquitetura orientada a serviço e com suporte
a implantação de serviços REST em vários linguagens, chamado \neocortex{}.

A característica do \neocortex{} de se utilizar serviços implementados em vários
linguagens motivou o desenvolvimento um programa gerador de serviços poliglota
-- que produz código de várias linguagens de progração -- a partir da descrição
do contrato do serviço. Daí nasceu a \neoidl{}.

Entretanto, as linguagens de programação disponíveis para especificação de
contratos REST, como Swagger, WADL e RAML, tinham (e ainda tem) limitações
importantes para a abordagem desejada de se escrever primeiramente o contrato e,
a partir dele, gerar a implementação. Todas elas utilizam linguangens de
propósito geral (XML, JSON, YAML), tornando os contratos extensos e de difícil
compreensão por humanos. Além disso, não possuem mecanismos semânticos de
extensibilidade e modularidade.

Partiu-se então para o desenvolvimento de uma nova linguagem, com sintaxe
inspirada em linguagens mais claras e concisas -- CORBA IDL \cite{corba} e
Apache Thrift \cite{thrift} --, e que permitisse a declaração de tipos de
dados definidos pelo usuário e extensibilidade. Ambas,
CORBA e Apache Thrift, possuem limitações nesses últimos aspectos. A sintaxe e
as características da linguagem \neoidl{} são discutidas na subseção \ref{linguagemNeoIDL}.

Em relação à geração de código, dado o requisito de geração de código para
linguagens distintas, a \neoidl{} foi projetada para possui uma arquitetura
modular, de modo que novas linguagens ou características de implementaçao
pudessem se incorporadas por meio de \textit{plugins} da \neoidl{}. Assim, é
possível desenvolver um novo \textit{plugin} para geração de serviços em outras
linguagens, por exemplo PHP, sem alterar qualquer outro componente, conforme
apresentado na subseção \ref{frameNeoIDL}.

A primeira versão da \neoidl{}, ponto de onde partiu este trabalho, dava suporte
a geração de código em Java, Python e Swagger com as características necessárias
para execução no \neocortex{}. Foram desenvolvidos no decorrer do projeto nove
serviços do domínio de comando e controle \cite{david:commandControl},
compreendendo aproximadamente cinquenta módulos e geração de três mil linhas de
código Python a partir dos contrato especificados em \neoidl{} (contratos são
denominados \texttt{módulos} na \neoidl{}). Outros serviços foram implementados em Java. 



\subsection{\textit{Framework}}
\label{frameNeoIDL}
\vspace{-6mm}

A parte da \neoidl{} responsável pela geração de código de
serviço para as várias linguagens é chamada de \framework{} \neoidl{}. O núcleo do \framework{} é
composto de módulos responsáveis por fazer o \textit{parse} do contrato escrito
em \neoidl{}, por processar a especificação e pelo gerenciamento dos
\textit{plugins}. Já os \textit{plugins} extendem, cada um, a \neoidl{} para as
linguagens de destino. A figura \ref{fig:neoidl-architecture} ilustra a
arquitetura do \framework{}.

\begin{figure}[h]
\begin{center}
\includegraphics[scale=0.6]{neoidl.pdf}
\vspace{-.5cm}
\end{center}
\caption{Arquitetura do framework de geração de código da \neoidl{}.}
\label{fig:neoidl-architecture} 
\end{figure}


O núcleo do \framework{} possui um uma pequena aplicação que carrega as
definições dos \textit{plugins} e processa os argumentos da chamada do gerador
(o arquivo de contrato que terá o código gerado, o diretório de destino e a
linguagem a ser utilizada). O modo de funcionamento geral do \framework{} é
ilustrado na figura \ref{fig:programGenerator}. O \textit{parser} da \neoidl{}
foi construído utilizando \bnfc{} \cite{ranta-bnfc:2012} com a linguagem
funcional Haskell.

\begin{figure*}[bt]
\begin{center}
\includegraphics[scale=0.55,trim=0cm 1.5cm 0cm 0cm]{programgenerator.pdf}
\vspace{-.5cm}
\end{center}
\caption{Gerador de código da \neoidl{}}
\label{fig:programGenerator}
\end{figure*}

As próximas subseções detalham o funcionamento de dois módulos onde estão
contidas os principais trechos da lógica implementada na \neoidl{}: O
\textit{PluginDef} e \textit{PluginLoader}.


\subsubsection{Componente PluginDef}{\label{sec:plugindef}}

O desenvolvimento de \textit{plugins} na \neoidl{} devem obedecer algumas regras
de projeto estabelecidas no componente \texttt{PluginDef}, que é um módulo
escrito em Haskell. No \texttt{PluginDef} declara-se dois tipo de dados
(\texttt{Plugin} and \texttt{GeneratedFile}) e uma assinatura de tipo
(\texttt{Transform = Module -> [GeneratedFile]}). Eles definem uma família de
tipos que mapeiam um módulo \neoidl{} em uma lista de arquivos com o código
gerado ao final do processo.

De acordo com estas regras de projeto, cada \textit{plugin} precisa declarar uma
instância do tipo \texttt{Plugin} e implementar uma função de acordo com a
assinatura definida em \texttt{Transform}. Além disso, cada instância de  
\texttt{Plugin} precisa ter o nome \texttt{plugin} de forma que o componenente
\texttt{PluginLoader} possa obter os dados necessários para o seu processamento.
Assim, a execução de um \textit{plugin} consiste em aplicar sua função de
transformação correspondente a um módulo \neoidl{} e produzir uma lista de
arquivos de código fonte. 

\vspace{6mm}

\begin{figure}[h]
\begin{small}
\lstinputlisting[language=HaskellSimple,firstnumber=1]{pluginDefStructure.tex}
\vspace{-.5cm}
\end{small}
\caption{Assinatura do \texttt{PluginDef}}
\label{lst:pluginDef}
\end{figure}

\subsubsection{Componente PluginLoader}

O carregamento e validação dos \textit{plugins} são competências do componente
\texttt{PlutinLoader}. O primeito passo é listar todos os arquivos do diretório
do \textit{plugin} a ser carregado, filtrar os arquivos Haskell (extensão
\texttt{``.hs''}), criar um nome para um nome reservado (\textit{qualified
name}) para cada arquivo de \textit{plugin} e aplicar a função de compilação.

Caso todas as regras de definição do \textit{plugin} tenham sido atendidas, o
\textit{plugin} é carregado e estará pronto para ser acionado. Caso contrário,
algumas exceções podem ocorrer, como, por exemplo, não haver definição de
nenhum \textit{plugin} no arquivo:

\begin{tabbing}\tt
~\char36{}\char46{}\char47{}neoIDL\\
\tt ~neoIDL\char58{}~panic\char33{}~\char40{}the~\char39{}impossible\char39{}~happened\char41{}\\
\tt ~~\char40{}GHC~version~7\char46{}6\char46{}3~for~x86\char95{}64\char45{}darwin\char41{}\char58{}\\
\tt ~~~~Not~in~scope\char58{}~\char96{}Plugins\char46{}Python\char46{}plugin\char39{}
\end{tabbing}

Ou ainda em razão de o \textit{plugin} não ser uma instância do tipo
\texttt{Plugin}:

\begin{tabbing}\tt
~\char36{}\char46{}\char47{}neoIDL\\
\tt ~neoIDL\char58{}~panic\char33{}~\char40{}the~\char39{}impossible\char39{}~happened\char41{}\\
\tt ~\char40{}GHC~version~7\char46{}6\char46{}3~for~x86\char95{}64\char45{}darwin\char41{}\char58{}\\
\tt ~~Couldn\char39{}t~match~expected~type~\char96{}Plugin\char39{}\\
\tt ~~~with~actual~type~\char96{}\char91{}GHC\char46{}Types\char46{}Char\char93{}\char39{}
\end{tabbing}



\subsection{Linguagem}
\label{linguagemNeoIDL}
\vspace{-6mm}

A \neoidl{} simplifica a especificação de contratos REST pois possui uma sintaxe
concisa própria de linguagens de especificação de interfaces (\emph{interface
description languages}). Ademais, a \neoidl{} provê mecanismos de
modularização e herança, de forma que os contratos possam ser separados em
módulos \footnote{Os arquivos utilizados para definição de contratos
são denominados módulos na \neoidl{}}, facilitando a herança e manutenção dos
contratos.

Para demonstrar como os módulos são estruturados na
\neoidl{}, a seguir são apresentados alguns trechos de um serviço hipotético
de envio de mensagens. Na primeira parte, um módulo faz uma definição de tipo
de dado; em seguida, um segundo módulo, voltado para especificação do serviço em
si, importa as definições do primeiro para então declarar as operações do
serviço. Por fim, ao módulo de serviço é acrescentada uma anotação como
forma de extender as características da operação.

\vspace{6mm}

\begin{figure}[h]
\begin{small}
\lstinputlisting[language=NeoIDL,firstnumber=1]{mensagemData.tex}
\vspace{-.5cm}
\end{small} 
\caption{Tipos de dados definidos na \neoidl}
\label{lst:messagedata-neo}
\end{figure}

O trecho ilustrado na figura \ref{lst:messagedata-neo} faz a definição de dois
tipos de dados. \emph{MessageType}, declarado no linha 2, é uma estrutura
simples do tipo enumeração. O outro tipo é \emph{Message}, declarado entre as
linhas 4 e 11, composto de seis atributos. O atributo \emph{type} de
\emph{Message} é do tipo \emph{MessageType} recém declarado.

Na \neoidl{} é utiliza a abordagem convenção sobre configuração, de modo que
todos os atributos declarados são obrigatórios, a menos que explicitamente
seja declarado diferente. O atributo \emph{subject} do tipo \emph{Message} é um
exemplo de atributo opcional (\texttt{<Type> <Ident> = 0;}).

O módulo seguinte (figura \ref{lst:sentmessage-neo}), na linha proposta pela
\neoidl{} para suporte a herança e reuso, importa o conjunto de definições de \emph{MessageData} e declara o
serviço \emph{sendbox} (linha 4), o qual possui duas capacidades. A
capacidade \emph{sendMessage} (linha 6) utiliza a operação \method{post} para submeter uma
mensagem (tipo \emph{Message}). A outra capacidade (linha 7) tem a finalidade de
listas as mensagens com um determinado sequecial, por meio da operação
\method{get}.

Por fim, a instrução \emph{path} indica o caminho (URI) onde as
operações serão disponibilizadas. Esse atribuito é importante para se definir
como as requisições serão roteadas entre os serviços.

\vspace{6mm}

\begin{figure}[htb]
\begin{small}
\lstinputlisting[language=NeoIDL,firstnumber=1]{mensagem.tex}
\end{small}
\caption{Sent message service specification in \neoidl}
\label{lst:sentmessage-neo}
\end{figure}

Ainda na filosofia de conversão sobre configuração, a \neoidl{} assume que os
argumentos das operações \method{POST} e \method{PUT} são enviadas no corpo da
requisição. Nas operações \method{GET} e \method{DELETE}, por outro lado,
presume-se que os agumentos estão contidos no \emph{path} da requisição ou ainda
como \textit{query string}.

A especificação dos contratos na \neoidl{} pode ser enriquecida com
anotações, por meio das quais se possibilita extender a semântica de uma
especificação sem que seja necessário alterar a sintaxe da \neoidl{}. Esse
recurso da \neoidl{} simplifica a inclusão de novas características aos
serviços, pois a alteração da sintaxe da própria \neoidl{} envolve um esfoço não
trivial de compatibilizar todos os \textit{plugins} já construídos.

O módulo apresentado na figura \ref{lst:annotationNeoIDL} contém, além das
informações contidas no módulo da figura \ref{lst:sentmessage-neo}, uma anotação chamada
\emph{Security Policy} (linhas 4 a 6) aplicada ao serviço \emph{sentbox}. A
declaração da anotação é feita no final do módulo (linhas 14 a 18).

\vspace{6mm}
 
\begin{figure}
\begin{small}
\lstinputlisting[language=NeoIDL,firstnumber=1]{messageSendAnnotation.tex}
\vspace{-.5cm}
\end{small}
\caption{Especificação de anotação na \neoidl{}}
\label{lst:annotationNeoIDL}
\end{figure}

Além de aplicáveis a \texttt{resources}, as anotações também podem ser aplicadas
a outros construtores da linguagem: \texttt{module}, \texttt{enum},
\texttt{entity}. Qualquer anotação na \neoidl{} possui a mesma estrutura: um
nome, um elemento alvo e uma lista propriedades. Todas estas informações ficam
disponíveis para utilização pelos \textit{plugins}.

A estrutura de um módulo \neoidl{}, como pode ser verificado nos exemplos
acima, é composta pelas seguintes seções: importação, caminho do serviço,
enumerações, entidades, serviços e declaração de anotações. O apêndice 
\ref{apend:estruturaLexicaNeoIDL} apresenta a estrutura sintática e léxica da
\neoidl{} e refletem a situação de Agosto de 2015, antes da incorporação das 
construções relativas a \designbycontract{}.



\section{AVALIAÇÃO EMPÍRICA}
\vspace{-6mm}

A primeira versão da \neoidl{} foi submetida a um estudo empírico de sua expressividade 
e reuso em um contexto real. As próximas subseções apresentam 
o resultado da análise comparativa da representação de 44 (quarenta e quatro)
contratos escritos em Swagger em relação à mesma especificação em \neoidl{}.
Esse estudo foi realizado no transcurso do trabalho de mestrado e 
publicado no periódico IJSeke \cite{lima2015neoidl}.

\subsection{Expressividade}
\label{estudoExpressividadeNeoIDL}
\vspace{-6mm}

A \neoidl{} é uma DSL, conforme apresentado na seção \ref{apresentacaoNeoIDL},
e como tal se destina a atender a um propósito específico, nem mais, nem menos
\cite{hudak1998modular}. A \neoidl{} foi projetada para permitir a
especificação de contratos de serviços REST de forma mais expressiva e concisa,
facilitando-se a escrita e leitura por humanos (mais detalhes na subseção
\ref{histMotivNeoIDL}).

Programas escritos em DSLs costumam ser mais fáceis de escrever e,
consequentemente, mais fáceis de se manter comparavelmente a programas escritos
em linguagens de propósito geral \cite{hudak1998modular}. Isso se deve
justamente ao fato de a DSL tratar apenas um conjunto reduzido de situações e
problemas, fazendo com ela seja, muitas vezes, mais acessível ao público geral
\cite{taha2008domain}.

A expressividade é um dos principais critérios para se escolher uma
linguagem. Entretanto a linguagem que não expressa todas as situações
necessárias ao seu contexto de uso, por óbvio, não pode ser usada
\cite{mackinlay1985expressiveness}. Nesse sentido, o primeiro teste a que a
\neoidl{} foi submetida constituiu-se na produção de contratos e serviços reais
no início do projeto com o Exército Brasileiro (vide subseção \ref{histMotivNeoIDL}).

Assim, tendo a \neoidl{} demonstrado sua capacidade de representar contratos
REST, foi realizada uma segunda análise: quão expressiva seria a \neoidl{}
em comparação com outra linguagem com o mesmo objetivo. Foi escolhida
Swagger \cite{swaggerSite}, uma linguagem cujo uso tem crescido pela indústria.
Em Swagger, os contratos são escritos em JSON\cite{JSon} ou YAML\cite{YAML},
ambos com uma estrutura geral de chave-valor.

Tendo sido a facilidade de compreensão e manipulação por humanos um dos pilares
de desenvolvimento da \neoidl{}, foi adotada a estratégia de comparar a
expressividade em termos de quantidade de linhas de código (SLOC - do inglês
\textit{Source Lines of Code}), uma vez que muitas linhas significam maior
esforço para escrita, sobretudo na abordagem \CtFirst{}.

Com este propósito, foi obtido um conjunto de 44 contratos do Exército
Brasileiro especificados em Swagger. A primeira etapa foi reescrever esses
contratos em \neoidl{} e então comparar a quantidade de linhas de código
produzida com a quantidade de linhas de código dos contratos originais. Para a
transcrição dos contratos, em razão da \neoidl{} não possuir recursos para
transcrição bidirecional de Swagger para \neoidl{}, foi desenvolvido
e utilizado um \textit{script} na linguagem \textit{Perl}, 
o qual consta do Apêndice desta dissertação
(Apêndice \ref{scriptSwagger2NeoIDL}).

Os quarenta e quatro contratos em Swagger contabilizaram 13.921 linhas de
especificação. Os mesmos contratos especificados em \neoidl{} somaram 5.140
linhas de especificação, correspondendo a uma redução média de 63\%. Assim, para
cada 10 linhas de especificação em Swagger são requeridas 4 linhas de
especificação \neoidl{}. Nessa análise foram consideradas linhas físicas de
código, ignorando-se linhas em branco ou compostas apenas de delimitadores.

A proporção de redução não se deu de forma igual em todos os contratos. Por
exemplo, o contrato de um determinado serviço \footnote{Os nomes reais dos
contratos foram omitidos em razão de acordo de confidencialidade.} requereu 367
linhas na especificação Swagger e 112 linhas na especificação \neoidl{} --
redução da ordem de 69\%.
Em contraponto, outro serviço especificado em Swagger possuia 81 linhas e o
correspondente em \neoidl{} 42 linhas -- redução de linhas de código pouco inferior a 50\%.

Estatisticamente, o tamanho original dos contratos tem apenas uma pequena
influência na expressividade avaliada. Dessa forma, não é possível assumir que
contratos Swagger maiores terão um correspondente proporcionalmente menor em
\neoidl{}.
Outros atributos como documentação mais descritiva, quantidade de entidades e o
número de capacidades de cada serviço também não possuem correlação com redução
de linha de código maior ou menor após o processo de tranformação para
\neoidl{}.

A tabela \ref{tab:size-corr} apresenta a correlação entre a
melhoria na expressividade observada (medida como percentual de redução após a
transformação da especificação Swagger em especificação \neoidl{}) e algumas
métricas relacionadas ao tamanho da especificação original em Swagger.

% 
% 
% Table~\ref{tab:size-corr} presents the correlation between the improvement of
% expressiveness (measured as the percentage of reduction obtained after
% transforming Swagger specifications into \neoidl{} specifications) and some
% metrics related to the size of the original Swagger specifications.

\begin{table}[htb]
\caption{Correlação da melhoria de expressividade com o tamanho da especificação
em Swagger}
\begin{center}
\begin{tabular}{lrr} 
\toprule
Métrica & Correlação Pearson's & \emph{p-value} \\ \hline \hline 
LOC da especificação Swagger & 0.19 &  0.20 \\ 
Número de serviços & 0.14 & 0.35 \\ 
Numero de capacidades & 0.14 & 0.34 \\
Número de entidades & 0.20 & 0.18 \\ \bottomrule 
\end{tabular} 
\end{center}
\label{tab:size-corr}
\end{table}





\subsection{Potencial de reuso}

De forma similar à \neoidl{}, Swagger também possui recursos para reuso de
estruturas, por meio de referências (marcação \$ref). Entretanto, esse recurso
praticamente não foi explorado no conjunto de contratos analisados, o que
ocasinou na duplicação da declaração de estruturas entre os diferentes arquivos
de especificação Swagger. Isso se deve, provavelmente, ao modo não intuitivo de
se fazer referência em Swagger (baseado na referência a outro arquivo JSON) e à
dificuldade de se identificar que uma determinada entidade foi declarada em
outro contrato.

No conjunto dos 44 contratos Swagger analisados foram identificadas 40 entidades
especificadas em pelo menos dois contratos. Uma entidade específica, de
identificação da posição geográfica, muito utilizada no domínio de Comando e Controle, aparece
declarada 12 vezes com contratos distintos.

