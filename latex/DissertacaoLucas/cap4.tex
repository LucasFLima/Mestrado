\chapter{CONTRATOS REST COM DESIGN-BY-CONTRACT}


\section{PROPOSTA: SERVIÇOS COM DESIGN-BY-CONTRACT}
\label{PropostaServicoDbC}
\vspace{-6mm}

Os benefícios esperados pela adoção da arquitetura orientada a serviços
somente serão auferidos com a concepção adequada de cada serviço. 
Por essa razão, é necessário planejar o projeto dos serviços criteriosamente
antes de lançar mão do desenvolvimento, com preocupação especial em garantir
um nível aceitável de estabilidade aos consumidores de cada serviço.
Nessa etapa do projeto de desenho da solução, a especificação do contrato do
serviço (Web API) exerce uma função fundamental. 

Na sociedade civil, contratos são meios de se formalizar acordo entre partes a
fim de definir os direitos e deveres de cada parte e buscar atingir o
objetivo esperado dentro de determinadas regras. Cada parte espera que as outras
cumpram com suas obrigações.
Por outro lado, sabe-se que o descumprimeto das obrigações costuma implicar de
penalizações até o desfazimento do contrato. 

Contratos entre serviços Web seguem em uma linha análoga. O desenho das
capacidades (operações) e dos dados das mensagens correspondem aos
termos do contrato no sentido do que o consumidor deve esperar do serviço
provedor. Porém identificou-se, após ampla pesquisa realizada sobre o tema, que
as linguagens disponíves para especificação de contratos atingem apenas esse
nível de garantias. No contexto de webservices em REST, conforme descrito na
seção \ref{secaoREST}, há ainda a ausência de padrão para especificação
contratos.

A proposta deste trabalho é extender os níveis de garantias, de modo a promover
um patamar adicional com obrigações mútuas entre os serviços (consumidor e
provedor). Isso se dá pela adoção do conceito de \designbycontract{} (Seção 
\ref{Design-by-Contract}) em que a execução da capacidade do serviço garantirá 
a execução, desde que satisfeitas as condições prévias. As próximas subseções 
detalham o modo de operação dos serviços com as construções de \designbycontract{}.

\vspace{-6mm}

\subsection{Modelo de operação}
\vspace{-6mm}

As garantias para execução dos serviços são estabelecidas em duas etapas: pré e
pós-condições. Nas precondições o provedor do serviço estabelece os requisitos
para que o serviço possa ser executado pelo cliente. A etapa de pós-condições
tem o papel de validar se a mensagem de retorno do serviço possui os resultados
esperados.

O diagrama apresentado na Figura \ref{FigServiceDbC} descreve como ocorre a
operação das pré e pós-condições. O processo se inicia com a chamada à capacidade do serviço e a
identificação da existência de uma precondição. Caso tenham sido estabelecidas 
precondições, essas são avaliadas. Caso alguma delas não tenham sido
satisfeitas, o serviço principal não é processado e o provedor do serviço
retornar o código de falha definido no contrato correspondente.


\begin{figure}[!htb]
\centering
\includegraphics[width=140mm,trim = 0mm 0mm 0mm
0mm,clip]{img/FluxoDbcCondicoes.pdf}
\caption{Digrama de atividades com verificação de pré e pós condições}
\label{FigServiceDbC}
\end{figure}

Caso tenham sido definidas pós-condições, essas são acionadas após o
processamento da capacidade, porém antes do retorno ao consumidor do serviço.
Assim, conforme Figura \ref{FigServiceDbC}, visando não entregar ao cliente uma
mensagem ou situação incoerente, as pós-condições são validadas. Caso todas as
pós-condições tenham sido satisfeitas, a mensagem de retorno é encaminhada ao
cliente. Caso contrário, será retornado o código de falha definido para a
pós-condição violada.

\subsubsection{Observação sobre invariantes}
\vspace{-6mm}

Em \designbycontract{}, além dos conceitos de pré e pós-condições,
há também a ideia de invariantes\cite{meyer1997object}. Quando aplicadas a uma classe na
orientação a objetos, as invariantes estabelecem restrições sobre o estado
armazenado nos objetos instanciados dessa classe. No contexto de orientação a
serviços, tem-se por princípio a ausência de estados dos serviços, descrito na
seção \ref{PrincipiosSOA}. Por essa razão, no estudo sobre a incorporação de
\designbycontract{} em contratos de serviços, as invariantes não foram
consideradas.


\subsection{Verificação das precondições}
\vspace{-6mm}

As precondições podem ser do tipo baseado nos parâmetros da requisição ou do
tipo baseado na chamada a outro serviço. Denomina-se, no contexto desta
dissertação, de básica a precondição baseada apenas nos parâmetros da
requisição (atributos da chamada ao serviço). Essa validação é direta,
comparando os valores passados com os valores admitidos. 

No caso das precondições baseadas em serviços, é realizada chamada a outro
serviço para verificar se uma determinada condição é satisfeita. Este modo de
funcionamento, que se assemelha a uma composição de serviço, é mais versátil, pois permite
validações de condições complexas sem que a lógica associada seja conhecida pelo
cliente. Assim, os contratos que estabelecem esse tipo de
precondição se mantem simples.

A Figura \ref{FigServicePrecondition} apresenta as etapas de verificação de cada
precondição. Nota-se que a saída para as situações de desatendimento às
precondições, independentemente do tipo, é o mesmo. O objetivo desta abordagem
é simplificar o tratametno de exceção no consumidor.

\begin{figure}[!htb]
\centering
\includegraphics[width=140mm,trim = 0mm 0mm 0mm
0mm,clip]{img/FluxoPrecondicoes.pdf}
\caption{Diagrama de atividades do processamento da precondição}
\label{FigServicePrecondition}
\end{figure}


\subsection{Verificação das pós-condições}
\vspace{-6mm}

\begin{figure}[!htb]
\centering
\includegraphics[width=140mm,trim = 0mm 0mm 0mm
0mm,clip]{img/FluxoPostcondicoes.pdf}
\vspace{-6mm}
\caption{Diagrama de atividades do processamento da pós-condição}
\label{FigServicePostcondition}
\end{figure}

A verificação das pós-condições acontece de modo muito similar a das
precondições. Há também os dois tipos, baseado em valores e em chamadas a
outros serviços. O diferencial está em que a validação dos valores passa a
ocorrer a partir dos valores contidos na mensagem de retorno. A Figura
\ref{FigServicePostcondition} descreve as etapas necessárias para validação de
cada precondição.


\section{EXTENSÃO DA NEOIDL PARA DESIGN-BY-CONTRACT}
\label{extensaoNeoIDL-DbC}

A sintaxe escolhida para possibilitar a especificação de pré e pós condições
na \neoidl{} foi influenciada por três linguagens e extensões de linguagens de
programação: Eiffel, JML e Spec\# (exemplificadas na subseção
\ref{implementDbC}). 

Em Eiffel, as asserções são expressões booleanas, de modo que uma pré e uma
pós-condição podem ter resultado verdadeiro ou falso. As asserções também podem
incluir chamadas a funções, extendendo a validações a lógicas mais sofisticadas
\cite{meyer1992applying}. Essas caracteríticas, por serem simples e versáteis,
foram consideradas adequadas e incorporadas à especificação de
\designbycontract{} em contratos de serviços na \neoidl{}.

A primeira sintaxe de \designbycontract{} na \neoidl{} teve como
base a sintaxe da JML, especialmente em como se associar as pré e
pós-condições a cada serviço ou capacidade, assemelhando-se a comentários e iniciados pelo
símbolo de arroba (@). A figura \ref{lst:precondicaoJML-neo} apresenta um
exemplo de especificação de precondição seguindo a linha da JML.

\vspace{6mm}

\begin{figure}[h]
\begin{small}
\lstinputlisting[language=NeoIDL,firstnumber=1]{trechos_codigo/DBCsimple.neo}
\vspace{-.5cm}
\end{small} 
\caption{Forma preliminar de precondição na \neoidl}
\label{lst:precondicaoJML-neo}
\end{figure}

Essa forma foi apresentada no Workshop de Teses e Dissertações do CBSoft em 2015
\cite{lima2015contratos}, ainda nos primeiros estágios do trabalho. Os revisores
apontaram dificuldade de distinguir, na especificação, entre as pré e
pós-condições e as capacidades, pois possuiam prefixos muito
semelhantes (ver linhas 6 a 8).
Essas críticas impulsionaram a busca por outra sintaxe mais adequada aos elementos textuais já
existentes na \neoidl{}.

Spec\# possui uma forma de especificação de asserções em que as pré e
pós-condições são declaradas logo após a assinatura do método ou classe, apenas
com o uso das palavras reservadas \emph{require} e \emph{ensure}, sem uso de
símbolos. Essa abordagem foi aplicada à \neoidl{} para versão final da sintaxe
com suporte a \designbycontract{}.

As próximas subseções apresentam alguns exemplos de especificação de pre e
pós-condições na \neoidl{} e as mudanças introduzidas na sintaxe da linguagem.
Inicialmente as condições de \designbycontract{} são demonstradas separadamente
e, ao final, a subseção \ref{SintaxeGeralDbc} consolida o conjunto
de novos elementos sintáticos e como eles são estruturados.


\subsection{Precondição básica}
\label{precondicaoBasica}

Uma precondição básica é a que valida os valores recebidos na
requisição, comparando-os com os valores estabelecidos na instrução
\emph{require} do contrato. Esse tipo de precondição assemelha-se a validação
dos atributos recebidos por um método no paradigma de orientação a objetos.

A origem das informações, isto é, onde os valores que serão validados se
encontram, depende da operação HTTP utilizada. A subseção \ref{FonteDadosDbc} descreve como esses
dados são obtidos. Para realizar a comparação do valor recebido com o valor
esperado, a \neoidl{} admite seis operadores de comparação (Subseção
\ref{TiposContrDbC}).

A Figura \ref{lst:DBCPreCondBasica} (linhas 7 e 8) apresenta um exemplo de
precondição básica em uma forma simples, em que apenas um valor é testado (id)
e, caso a condição não seja satisfeita, a instrução \emph{otherwise} indica o
valor a ser retornado (código HTTP Not Found). 	

\begin{figure}[htb]
\begin{small}
\lstinputlisting[language=NeoIDL,firstnumber=1]{trechos_codigo/store_pre_basica.neo}
\end{small}
\caption{Exemplo de notação de precondição básica na \neoidl{}}
\label{lst:DBCPreCondBasica}
\end{figure} 



\subsection{Pós-condição básica}

Em termos sintáticos, as pós-condições básicas possuem uma forma muito
semelhante às precondições básicas (\ref{precondicaoBasica}), diferindo-se
exclusivamente pelo uso da instrução \emph{ensure}. A Figura
\ref{lst:DBCPosCondBasica} (linhas 8 e 9) mostra um exemplo de pós-condição
básica em que, após a execução da operação \method{GET}, se o valor do atributo
\emph{quantity} não for maior que zero, então o serviço não foi executado adequadamente e a exceção
(\emph{otherwise}) é retornada.

\begin{figure}[htb]
\begin{small}
\lstinputlisting[language=NeoIDL,firstnumber=1]{trechos_codigo/store_pos_basica.neo}
\end{small}
\caption{Exemplo de notação de pós-condição básica na \neoidl{}}
\label{lst:DBCPosCondBasica}
\end{figure} 


\subsection{Precondição com chamada a serviço}

As precondições baseadas em serviços seguem uma sequência que envolve a chamada
a outro serviço antes do processamento do serviço principal, em um tipo simples de
composição de serviço. Essa abordagem permite que precondições complexas sejam
validadas por serviços especializados, sem que a especificação do contrato
de serviço seja complexa. Essa proposta preserva ainda a ideia original de
Eiffel\cite{meyer1988eiffel}, de que pré e pós-condições sejam expressões
\emph{booleanas}.

A primeira etapa do processo de execução da precondição de serviço consiste em
fazer a chamada a um serviço (ver Figura \ref{FigServicePrecondition}) por meio de uma
operação \method{GET}. Em seguida, o código de \textit{status} retornado pelo
serviço da precondição é comparado com o valor especificado na precondição do contrato.

Após o acionamento do serviço da precondição, o comportamento é o mesmo da
precondição básica (ver \ref{precondicaoBasica}). Caso a precondição seja
satisfeita, é retornado o valor indicado pela instrução \emph{otherwise}. As
precondições de serviço na \neoidl{} admitem os mesmos operadores de comparação
que as precondições básicas.

A Figura \ref{lst:DBCPreCondServico} ilustra a especificação de uma precondição
do tipo serviço (linhas 8 e 9). Assim, antes de executar a operação \method{POST}
do serviço principal, o serviço \emph{store.getOrder} é acionado. Caso esse
serviço retorno o código correspondente a \emph{HTTP Not Found}, a operação
\method{POST} é executada. Caso contrário, o serviço principal retorna o valor
correspondente a \emph{HTTP Invalid Precondition}, em razão do estabelecido no
\emph{otherwise}.


\begin{figure}[htb]
\begin{small}
\lstinputlisting[language=NeoIDL,firstnumber=1]{trechos_codigo/store_pre_servico.neo}
\end{small}
\caption{Exemplo de notação de precondição com chamada a serviço na
\neoidl{}} 
\label{lst:DBCPreCondServico}
\end{figure} 



\subsection{Pós-condição com chamada a serviço}
\label{Pos-condicao servico}

A pós-condição com chamada a serviço seguem a sequência de eventos indicada na
Figura \ref{FigServicePostcondition}. No caso da pós-condição, a execução do
serviço principal já ocoreu e a função do serviço na pós-condição é validar se a
execução do serviço principal ocorreu com sucesso. Algumas pós-condições são
naturais, como as que verificam se um objeto foi inserido após a operação de
inclusão (método \method{POST}). Ou ainda, a que verifica se o objeto foi excluído
após uma operação \method{DELETE}.

No exemplo da Figura \ref{lst:DBCPosCondServico}, o serviço principal faz a
exclusão de um objeto \textit{Order}. A pós-condição (linha 8) verifica, após o
processamento do \method{DELETE}, se o objeto foi efetivamente apagado por meio do
serviço \emph{store.getOrder}. Se o serviço da pós-condição retornar o valor
\emph{HTTP Not Found}, o objeto foi adequadamente excluído. Caso contrário, o
serviço principal retornará o valor \emph{HTTP Not Modified}, o qual foi
estabelecido na instrução \textit{otherwise}.

\begin{figure}[htb]
\begin{small}
\lstinputlisting[language=NeoIDL,firstnumber=1]{trechos_codigo/store_pos_servico.neo}
\end{small}
\caption{Exemplo de notação de pós-condição com chamada a serviço na
\neoidl{}}
\label{lst:DBCPosCondServico}
\end{figure} 


\subsection{Sintaxe geral de pré e pós-condições}
\label{SintaxeGeralDbc}

Entre as subseções \ref{precondicaoBasica} e \ref{Pos-condicao servico} foram
apresentados separadametne exemplos simples de especificação de pre e condições
na \neoidl{} de modo a facilitar a compreensão. Esta subseção demonstra
a estruturação sintática das construções de \designbycontract{} agregadas
\neoidl{} por este trabalho.

\subsubsection{Listas de pré e pós-condições}

Um módulo \neoidl{} possui uma seção para declaração dos serviços (ver Figura
\ref{fig:moduloNeoIDL}). Cada serviço declarado na \neoidl{} pode ter um mais
capacidades, as quais correspondem às operações HTTP utilizadas na arquitetura
REST. Sintaticamente, um serviço possui uma lista de capacidades\footnote{Os
símbolos ``[`` e ``]'' identificam uma lista.}:

\begin{center}
\hilight{\{\texttt{Serviço [Capacidade]}\}}
\end{center}

A sintaxe da \neoidl{} foi extendida para admitir a vinculação de pré e
pós-condições às capacidades. Essas contruções de \designbycontract{} são
opcionais, ou seja, uma capacidade pode não ter nenhuma pré ou pós-condição. Por
outro lado, pode-se incluir nas capacidades mais de uma precondição e mais de
uma pós-condição ou ainda qualquer combinação delas, simultâneamente.

\begin{center}
\hilight{\{\texttt{Capacidade \textcolor{red}{[Condição DbC]}}\}}
\end{center}

É possível ainda, caso uma precondição ou pós-condição se aplique a todas as
capacidades de um serviço, é possível declará-la para o serviço como um todo,
logo antes da declaração das capacidades:

\begin{center}
\hilight{\{\texttt{Serviço \textcolor{red}{[Condição DbC]} [Capacidade]}\}}
\end{center}

 Assim, generalizando, a \neoidl{} passou a suportar a seguinte sintaxe:
 
\begin{center}
\hilight{\{\texttt{Serviço \textcolor{red}{[Condição DbC]} [Capacidade
\textcolor{red}{[Condição DbC]} ]}\}}
\end{center}
 

\subsubsection{Tipos de construções de \designbycontract{}}
\label{TiposContrDbC}

Conforme exemplificado entre as subseções \ref{precondicaoBasica} e
\ref{Pos-condicao servico}, as construções de \designbycontract{} possuem as
seguintes estruturas:

\begin{center}
\hilight{Condição básica: \textbf{TipoCondição} \{\texttt{[Argumento Comparação
Valor]}\}}
\hilight{Condição serviço: \textbf{TipoCondição} \{\texttt{Serviço (Parâmetro)
Comparação ValorSrv}\}}
\hilight{Exceção: \textbf{Otherwise} \{\texttt{Valor de Retorno	}\}}

\end{center}

Onde:
\vspace{-6mm}
\begin{description}
\item [TipoCondição] indica se é uma precondição (\emph{require}) ou uma
pós-condição (\emph{ensure});

\item [Argumento] indica o nome do atributo que será testado;

\item [Comparação] indica a operação de comparação que será utilizada. A
\neoidl{}  admite seis operadores de comparação: igualdade (\literal{==}), 
diferença (\literal{<>}), maior (\literal{>}), maior ou igual (\literal{>=}), 
menor (\literal{<}) e menor ou igual (\literal{<=}). Eles podem ser aplicados  a
qualquer tipo de pré ou pós-condição.

Mais de um atributo pode ser testado em uma pré ou pós-condição. Na expressão
acima, essa característica é simbolizada com a indicação de uma lista.

\item [Valor] corresponde ao valor que será comparado com o \textbf{Argumento}.
As pré e pós-condições básicas admitem algumas combinações com os operadores
\emph{not}, \emph{and} e \emph{or}, formando expressões booleanas e,
assim, permitindo estabelecer regras mais abrangentes.

Por exemplo, a precondição apresentada na Figura \ref{lst:DBCPreCondBasica} poderia
ser escrita como \emph{\textbf{require} (\textbf{not} id \literal{<=} 0)}. Os
operadores \emph{and} e \emph{or} são infixos (ex.: \emph{\textbf{require}
((id \literal{>} 0 \textbf{or} id \literal{<=} -1000))}). O uso do operador
\emph{and} produz o mesmo efeito de se declarar duas precondições.

\item [Serviço] indica o nome do serviço que será acionado. As informações para
indicação concreta da localização do serviço, ou seja, a URL de acionamento, não
são indicados nesse atributo de pre e pós-condição. Essas informações dependem
do local de implantação do serviço e não convém que estejam declaradas no
contrato, sob pena de que mudanças no ambiente de implantação do serviço gerem
impacto ao contrato.
Entretanto, caso se queira fazer essa vinculação, pode ser criada uma anotação
\neoidl{}, com essa finalidade, para o serviço.

\item [Parâmetro] tem a função de indicar os valores que serão passados para a
chamada ao serviço da pré ou pós-condição.

 \item [ValorSrv] corresponde a um valor que será comparado com o
 \textit{Status Code HTTP} retornado pelo serviço. Na \neoidl{}, esses valores
 são convencionados como o nome do corresponde ao código HTTP, com palavras
 justapostas (Ex.: "NotFound" tem o valor do código 404, de HTTP \textit{Not
 Found})

\item [Valor de Retorno] é utilizado na cláusula \emph{otherwise} para indicar o
valor que o serviço principal deve retornar caso uma precondiçaõ ou pós-condição
não seja atendida.

\end{description}


\subsubsection{Um exemplo completo de módulo com \designbycontract{}}

A Figura \ref{lst:ModuloNeoVariosDbC} apresenta um exemplo de módulo \neoidl{}
de um serviço completo, utilizando alguns dos recursos apresentados na subseção
\ref{TiposContrDbC}. Esse serviço possui três operações, cada uma com duas
condições de \designbycontract{}.

A operação \method{GET} possui uma precondição em que o valor do atributo \emph{id} deve ser maior que zero (linha 8), caso contrário a operação deve retornar o valor
correspondente a "NotFound". Uma pós-condição assegura que, se o serviço for
executado adequadamente, o valor de \emph{quantity} será maior que zero (linha
9). Se for violada a pós-condição, a operação \method{GET} retorna o valor
"NoContent".

A operação \method{POST} possui duas precondições (linhas 13 e 14). A primeira,
básica, valida o valor do atributo \emph{quantity}. A segunda, aciona o serviço
\emph{store.getOrder} com o parâmetro \emph{id}. Nessa operação, foi
definido um valor de exceção único para as duas precondições. A \neoidl{}
associa essa instrução de \emph{otherwise} geral às instruções anteriores que
não possuem condição e exceção específica (caso da linha 8).

Na operação \method{DELETE} foi utilizado o operador lógico \emph{and} na
precondição (linha 18). A pós-condição faz a chamada ao serviço
\emph{store.getOrder}. Tanto a pré como a pós-condição possuem o mesmo valor de
exceção (linha 20). 

\begin{figure}[htb]
\begin{small}
\lstinputlisting[language=NeoIDL,firstnumber=1]{trechos_codigo/store_dbc_complete.neo}
\end{small}
\caption{Exemplo de módulo \neoidl{} com várias instruções de \designbycontract{}}
\label{lst:ModuloNeoVariosDbC}
\end{figure} 	



\subsection{Fontes de dados para pré e pós-condições}
\label{FonteDadosDbc}

Os serviços REST na \neoidl{} seguem uma convenção em relação ao conteúdo
presente na requisição e na resposta para cada tipo de operação \emph{HTTP},
conforme descrito na subseção \ref{linguagemNeoIDL}. Por exemplo, o método \method{GET}
submete os argumentos pelo \emph{path} ou como \textit{query string} da
requisição e não em seu corpo. Esses aspectos foram considerados para se definir
a fonte das informações para os argumentos e parâmetros utilizados nas pré e
pós-condições.

A Figura \ref{Fig:FonteDadosDbcNeoIDL} resume a origem para cada operação. Há
diferenças também entre os tipos básico e baseado em serviços das construções de
\designbycontract{}. As operações \method{GET} e \method{DELETE} não possuem
argumentos encaminhados no corpo da requisição. Para essas operações, os
argumentos são retirados do \emph{path} ou \emph{query string} para validação
das precondições, tanto no tipo básico como no tipo baseado em composição. Na
figura, essa origem é identificada como \textit{Request arguments}.

\begin{figure}[!htb]
\centering
\includegraphics[width=120mm,trim = 0mm 0mm 0mm
0mm,clip]{img/FonteDadosDbcNeoIDLIngles.pdf}
\caption{Diagrama da fonte de dados para acionamento de pré e pós-condições}
\label{Fig:FonteDadosDbcNeoIDL}
\end{figure}


Por outro lado, as operações \method{POST} e \method{PUT} submetem os dados a
serem inseridos ou alterados pelo corpo da requisição (\textit{Request body}),
local de onde as precondições básicas e baseadas em serviços devem extrair os
parâmetros de validação.
A \neoidl{} utiliza a notação JSON\cite{JSon} no corpo das requisições. A
representação dos argumentos utiliza o padrão separado por pontos para indicar
elementos mais internos (ex. "Pessoa.Nome").

No caso das pós-condições, a origem das informações diverge entre o tipo
básicos e o tipo por serviços. A única operação que admite pós-condição básica é
a operação \method{GET}, em que os argumentos de validação são extraídos do
corpo da resposta (\textit{Response body}). As demais operações não retornam dados
no corpo da requisição, pois estão voltadas para alteração de informações e não
para consultas. Além disso, não há utilidade em se validar dados de requisição
em uma pós-condição.

Já as pós-condições baseadas em serviços não suportam a operação \method{GET}.
Embora seja tecnicamente possível acionar um serviço com base nos argumentos da
requisição, como não se tem modificação nos dados por essa operação, a
validação dessas informações pode meio de serviço deve ser feita nas
precondições. Ademais, as pós-condições básicas da operação \method{GET} cumprem
com o objetivo que validar as informações de saída.

A pós-condição da operação \method{DELETE} pode acionar serviços com base nos
argumentos da requisição (\textit{Request arguments}). Um caso típico é de
acionar o serviço de consulta para verificar se o dado foi efetivamente
excluído. As operações \method{POST} e \method{PUT} podem acionam serviços em
pós-condições utilizando as informações do corpo da requisição (\textit{Request
body}), uma vez que essas operações não possuem dados no corpo da resposta. 



\section{ESTUDO DE CASO: PLUGIN TWISTED}
\label{pluginTwisted}

A incorporação de regras de \designbycontract{} aos contratos para
serviços REST escritos em \neoidl{} elevam a um novo patamar os níveis de
garantias com a estabilidade comportamental dos serviços. Nesse sentido, a
preocupação em garantir ao cliente que o serviço proverá as informações de que
ele necessita aumenta, reforçando os benefícios observados com a prática
\CtFirst{}. Ou seja, o desenho do serviço considera ainda mais a perspectiva do
consumidor do serviço.

A seção \ref{extensaoNeoIDL-DbC} tratou do elemento \emph{Contrato}, sobre como
ele pode ser escrito em \neoidl{} com suporte a construções de
\designbycontract{}. O contrato, porém, é apenas uma especificação, no
sentido de descrever regras e não de torná-las executáveis em si. Todavia, a
\neoidl{} é, além de uma linguagem formal, um \framework{} de geração de
código poliglota (subseção \ref{histMotivNeoIDL}) por meio de plugins.

Para se comprovar a viabilidade de se conceber serviços com suporte a pré e
pós-condições, foi desenvolvido, no decorrer da pesquisa de mestrado, um plugin
da \neoidl{} que cumpre com tal finalidade. As próximas subseções detalham como
é estruturada a arquitetura do serviço gerado e seu comportamento em relação a
\designbycontract{}. Ao final, alguns aspectos sobre a implementação do
próprio plugin são descritos.

\subsection{Visão geral do Python \twisted{}}
\label{PythonTwisted}

Adotou-se o \framework{} \emph{Python Twisted}\cite{fettig2005twisted} como tecnologia para
construção dos serviços com pré e pós-condições. A escolha se deu em razão de
\twisted{} possuir uma arquitetura voltada para processamento de requisições de vários
tipos de protocolos de rede sobre uma infraestrutura simples e autônoma. 

O \twisted{} é baseado em eventos e adota a estratégia de tratamentos de
requisições de forma assíncrona, em detrimento ao uso de \textit{threads}. O
núcleo do \twisted{} é o \textit{loop} do objeto \emph{reactor} responsável por
aguardar e direcionar o processamento dos eventos (Figura
\ref{Fig:TwistedReactor}). Uma requisição HTTP, como as das chamadas a serviços REST, são tratados como
eventos.

\begin{figure}[!htb]
\centering
\includegraphics[width=80mm,trim = 0mm 0mm 0mm
0mm,clip]{img/TwistedReactor.pdf}
\caption{Arquitetura assíncrona do \twisted{}}
\label{Fig:TwistedReactor}
\end{figure}

O \emph{reactor} entrega os eventos para serem tratados para processamentos
especializados, indicados no lado direito da Figura \ref{Fig:TwistedReactor}.
Caso o processamento de um evento seja lento, deve-se disparar um
processamento assíncrono, e registrar no \emph{reactor} uma chamada para quando
o processamento assíncrono se encerrar, o qual será tratado como um outro
evento.
Esse controle é feito por um objeto denominado \emph{Deferred}, que contém uma lista
de \emph{Callbacks} \cite{fettig2005twisted}.


\subsection{Arquitetura dos serviços \twisted{}}
\label{ArquiteturaTwisted}

Os serviços \twisted{} gerados pela \neoidl{} são estruturados em uma
arquitetura que extende a arquitetura base do \twisted{} \emph{Reactor},
incorporando serviços ao processamento das requisições, como ilustrado na Figura
\ref{Fig:TwistedReactorServices}. Os serviços são autônomos entre si, e
processam as requisições de acordo com o roteamento realizado pelo servidor.

\begin{figure}[!htb]
\centering
\includegraphics[width=80mm,trim = 0mm 0mm 0mm
0mm,clip]{img/TwistedReactorServices.pdf}
\caption{Operação dos serviços na arquitetura \twisted{}}
\label{Fig:TwistedReactorServices}
\end{figure}

Em termos de classes, a \neoidl{} gera um conjunto de tres módulos base,
que constituem o pacote do \twisted{} \textit{server}, representados na
parte superior da Figura \ref{Fig:TwistedArchtecture}. Essas classes são fixas,
ou seja, não dependem da quantidade nem do conteúdo dos serviços declarados no
módulo \neoidl{}.

A classe \emph{Server} é o núcleo do \twisted{} \textit{server}. Ela é
responsável por subir o servidor HTTP, receber as requisições, identificar
as operações da requisição (\method{GET}, \method{POST}, \method{PUT} ou \method{DELETE}), e
direcionar a requisição para o serviço específico. A identificação do serviço é
feita por meio de um arquivo de rota (routes.json), o qual possui uma tradução
entre os \emph{paths} das requisições e os serviços responsáveis por cada uma
delas.

A classe \emph{Utils} contém um conjunto de funções utilitárias, como a que
realiza o \textit{parse} da requisição para extrair os argumentos repassados
na URI ou \textit{query string}. Ela também define o objeto que trafega a
resposta dos serviços entre os serviço e o servidor.

Essas duas classes são suficientes para o servidor quando não se utiliza
\designbycontract{} nos serviços. O pacote
\emph{DbcConditions} consolida o conjunto de classes responsáveis por processar
as pré e pós-condições. A principal função de \emph{DbcConditions} é realizar a
comparação entre o valor real e o valor esperado, efetivando toda a
lógica corresponde às construções de \designbycontract{} descritas na subseção
\ref{TiposContrDbC}.

\begin{figure}[htb]
\centering
\includegraphics[width=90mm,trim = 18mm 4mm 0mm
0mm,clip]{img/TwistedServer.pdf}
\caption{Arquitetura do serviço Twisted gerado pela \neoidl{}}
\label{Fig:TwistedArchtecture}
\end{figure}

Em \emph{DbcConditions} estão as funções que carregam a lista de pré e
pós-condições para cada serviço. A chamada para as pré e pós-condições baseadas
em serviços também é construída nesse pacote. Essa pacote é fundamental para o
funcionamento das construções de \designbycontract{} e consta transcrita no
Apêndice \ref{classesDbcCondition}.

Para cada serviço declarado no módulo \neoidl{} são geradas duas classes: os
filtros do serviço e o serviço em si. A classe \emph{ServiceFilter} recebe a
requisição; carrega e processa as precondições; aciona o serviço e; ao final,
carrega e processa as pós-condições. Esse modo de operação dos filtros é
ilustrado na Figura \ref{Fig:TwistedFiltes}. A classe do serviço, por
fim, contém apenas a estrutura para implementação da lógica interna do serviço.

\begin{figure}[!htb]
\centering
\includegraphics[width=80mm,trim = 5mm 6mm 0mm 
3mm,clip]{img/TwistedFilters.pdf}
\caption{Modo de operação das pré e pós-condições no serviço Twisted}
\label{Fig:TwistedFiltes}
\end{figure}



\subsection{Geração de código}

O plugin \twisted{} é um conjunto de seis módulos Haskell. Para cada tipo de
arquivo gerado, há um módulo no plugin, conforme Figura \ref{PluginTwisted}. Os
módulos \emph{PPService}, \emph{PPUtils} e P\emph{PDbcConditions} apenas
imprimem um código Python fixo, sem qualquer interferência do módulo \neoidl{}
processado. \emph{PPRoute} é um pequeno módulo (42 linhas) que processa as
informações contidas nas instruções \emph{path} dos serviços declarados no
módulo \neoidl{} e mapeia a correspondência entre as URIs e os serviços.

\begin{figure}[!htb]
\centering
\includegraphics[width=100mm,trim = 0mm 0mm 0mm 
0mm,clip]{img/PluginTwisted.pdf}
\caption{Plugin para geração de código Twisted com suporte a \designbycontract{}}
\label{PluginTwisted}
\end{figure}

O módulo \emph{PPService} gera uma classe com um método para cada operação do
serviço. Na versão atual do plugin, os métodos são implementados com uma lógica
de gravação de objetos em um banco em memória, apenas para simplificar o teste
dos código gerado. Essa implementação não é relevante, uma vez que a lógica
especializada do serviço real a substituirá.

A parte correspondente ao acionamento das pré e pós-condições é produzido pelo
módulo \emph{PPServiceFilter}. O código produzido por este módulo é composto de
duas seções: (i) a declaração das condições de \designbycontract{} e (ii) o
carregamento dessas condições. A primeira seção, de declaração das condições
\designbycontract{} contém a lista de pré e pós-condições do serviço. A Figura 
\ref{Fig:LhsRhsDbcConditions} apresenta um exemplo de condição especificada em
\neoidl{} transformada em código \emph{Python Twisted}.

O lado esquerda dessa
figura contém a especificação de uma pós-condição (\emph{\textbf{ensure}}, na
linha 2) associada à operação \method{GET} (linha 1). Constitui-se de uma
pós-condição básica, que testa se o valor de resposta \emph{quantity} é maior que zero
(linhas 3 a 5). Caso a pós-condição não seja satisfeita, o serviço retorna o
código \emph{HTTP No Content}.


\setbox0=\hbox{%
%\begin{minipage}{1.9in}
\begin{minipage}{6cm}
\begin{lstlisting}[
basicstyle={\tiny\ttfamily},
identifierstyle={\color{black}},
tabsize=2,
language={NeoIDL},
numbersep=8pt,
numbers=left,
xleftmargin=0.5cm,frame=tlbr,framesep=2pt,framerule=0pt,
morekeywords ={class,run}
]
@get    Order getOrder (int id)
		ensure (
				quantity
				>
				"0"
		),
		otherwise "NoContent";
\end{lstlisting}
\end{minipage}
}
\savestack{\listingA}{\box0}

\setbox0=\hbox{%
\begin{minipage}{6cm}
\begin{lstlisting}[
basicstyle={\tiny\ttfamily},
identifierstyle={\color{black}},
tabsize=2,
language={PythonTwisted},
numbersep=8pt,
numbers=left,
xleftmargin=0.5cm,frame=tlbr,framesep=2pt,framerule=0pt,
morekeywords ={class,run}
]
self.list.append(
	DbcCheckBasic(
		'quantity',
		'>',
		'0',
		204,
		ValuesSource.responseBody,
		DbcConditionType.PostCondition,
		OperationType.GET
	)
)
\end{lstlisting}
\end{minipage}
}
\savestack{\listingB}{\box0}

\begin{figure}[h]
\begin{center}
\begin{tabular}{|c|c|c|}
\hline
%\stackinset{l}{-5pt}{t}{13\llength}{$\bullet$}{\listingA} &
%\stackinset{l}{-5pt}{t}{ 7\llength}{$\bullet$}{\listingB} \\
\stackinset{l}{-5pt}{t}{}{}{\listingA} &
 &
\stackinset{l}{-5pt}{t}{}{}{\listingB} \\
\hline
\end{tabular}

\caption{Transformação de pós-condição \neoidl{} (lado esquerdo) em código
Python \twisted{} (lado direito)}
\label{Fig:LhsRhsDbcConditions}
\end{center}
\end{figure}

Do lado direito da Figura \ref{Fig:LhsRhsDbcConditions} está o código
\emph{Python Twisted} correspondente. O motor de transformação identifica que a
condição especificada é uma pós-condição básica (classe \emph{DbcCheckBasic} na linha 2
e tipo na linha 8) associada uma operação \method{GET} (linha 9). Conforme
convenção adotada sobre a fonte de informações (Subseção \ref{FonteDadosDbc}), a
pós-condição é carregada com a indicação de que os argumentos devem ser lidos do
corpo da resposta do serviço (linha 7).

Os parâmetros da pós-condição possuem uma correspondência direta, em que as
linhas 3 a 5 do lado esquerdo correspondem também às linhas 3 a 5 do lado
direito. O valor de exceção da especificação \neoidl{} é traduzido no código
HTTP correspondente (linha 6). Ambos os códigos apresentados na Figura
\ref{Fig:LhsRhsDbcConditions} são, na realidade, escritos em uma única linha.
O uso de múltiplas linhas foi adotado aqui apenas para efeitos didáticos.

A segunda seção da classe \emph{ServiceFilter} é genérica para
qualquer serviço. A Figura \ref{lst:filtrosServicosTwisted} contém um trecho
dessa seção. Cada método se inicia com o carregamento das precondições (linha
10), seguindo pelo acionamento do serviço principal (linha 12). Por fim,
as pós-condições são carregadas (linha 14).

\begin{figure}[h]
\begin{small}
\lstinputlisting[language=PythonTwisted,firstnumber=1]{trechos_codigo/serviceFilterLoad.py.tex}
\vspace{-.5cm}
\end{small} 
\caption{Seção de código do filtro de serviços}
\label{lst:filtrosServicosTwisted} 
\end{figure} 


\section{ESTUDO EMPÍRICO POR MEIO DE ANÁLISE SUBJETIVA} 
\label{analiseSubjetiva}
\vspace{-6mm}

A construção de \textit{softwares} mais confiáveis é um dos grandes objetivos da
engenharia de \textit{softwares}. O conceito de confiança, porém, não é uma
característica geral, mas uma noção relativa: um \textit{software} é correto em
relação a uma determinada especificação \cite{arnout2001net}. Nesse cenário, a
qualidade da especificação é fundamental. Em particular, no que tange a relação
de confiança entre elementos de \textit{software}, a qualidade do
contrato é essencial.

\textit{Design-by-Contract} surgiu para atender a essa necessidade e tem se
mostrado como uma abordagem efetiva no contexto de orientação a objetos desde
suas primeiras experimentações \cite{jazequel1997design} ainda na década de
90. A proposta é particularmente útil quando se lida com sistemas distribuídos
\cite{arnout2001net}. 

O objetivo dessa seção é investigar empiricamente o uso de construções de
\designbycontract{} no contexto de computação orientada a serviços, um
subconjunto da vertente de sistemas distribuídos, verificando a viabilidade e
utilidade de sua adoção na especificação de contratos e implementação de
serviços REST. Em outras palavras: submeter a uma avaliação subjetiva o objetivo
geral desse trabalho de pesquisa.

\subsection{Planejamento do estudo empírico}

Há diversas formas e técnicas para se conduzir um estudo empírico, cada uma com
suas vantagens e desvantagens \cite{shull2008guide}. Nas técnicas diretas, o pesquisador
atua de forma explícita e perceptível pela equipe do projeto de \textit{software}, interagindo
com eles em maior ou menos grau. As técnicas indiretas são caracterizadas pelo
acesso indireto do pesquisador aos participantes, por meio da captura de
informações do ambiente de trabalho deles. Há ainda a abordagem independente,
em que se analisa apenas o produtos de trabalho, como documentação e código
fonte.

As técnicas diretas são divididas em inquisitivas e de observação.
\textit{Brainstorming}, grupo focais, entrevistas, questionários e modelagem
conceitual estão entre as técnicas inquisitivas. Essas são, muitas vezes, a
única forma de obter o engajamento dos envolvidos na realização das atividades.
Entretanto, elas tem o risco de serem excessivamente subjetivas e de medição de
resultados pouco precisa.

A abordagem observatória está ligada a técnicas como seções de
\textit{think-aloud}\footnote{Momentos em que os participantes são
convidados a expressar seus pensamentos em voz alta ('pensar alto'), a partir de
questões levantadas pelo investigador}, sombreamento\footnote{Consiste
em acompanhar o participante na realização de todas as suas atividades} e
participante-observador (em que o pesquisar participa realizando 
atividades chave para o projeto).
As técnicas observatórias permitem um estudo em tempo real dos fenômenos. Porém, elas levam a uma fase de análise
difícil, pois geram muitas informações e requerem um elevado conhecimento para
se interpretar corretamente os dados.

No decorrer do trabalho de pesquisa, foram cogitadas algumas estratégias para
condução da investigação sobre adoção de \designbycontract{} em serviços REST.
Entre as alternativas, estava a realização de experimento com alunos de
graduação, apresentando especificações textuais com características de
\designbycontract{} e solicitando a eles a elaboração de especificação
correspondente em \neoidl{}.

Outra possibilidade de estudo seria, a partir da análise do
código fonte de \textit{softwares} de código aberto que utilizam serviços REST escritos
em \textit{Swagger}, investigar a existência de comentários ou documentação que
caracterizassem situações típicas de pré e pós-condições. 

Essas ações foram modeladas conforme a técnica GQM\cite{basili1992software} e,
após avaliação dos benefícios e riscos, elas foram descartadas por
demandarem tempo excessivo em suas etapas de preparação e execução. Dada essa
conclusão, adotou-se uma terceira alternativa: aplicação de questionário, que, embora exija cuidado e reflexão sobre a modelagem das perguntas e respostas, é mais rápido de se preparar e permite a participação de pessoas em tempos e locais distintos.

O modelo GQM -- \textit{Goals-Questions-Metrics} -- é uma abordagem
estruturada e documentada na qual se estebelece uma meta para medição de
processos e produtos de \textit{software}. A Tabela \ref{TabelaMetasGQM}
apresenta como as metas são descritas em um modelo GQM, com a finalidade de
caracterizá-las objetivamente em termos de objeto, finalidade, foco, ponto de
vista e ambiente da análise.

\begin{table}[h]
\centering
\vspace{0.5cm}
\begin{tabular}{r|lr}
\multicolumn{2}{c}{GQM}\\
\hline    
\textbf{Objeto} & O processo ou estudo. \\
\textbf{Finalidade}  & Motivação por trás da meta (por que). \\
\textbf{Foco} & A qualidade do objeto em estudo (o quê).\\
\textbf{Ponto de vista} & Perspectivas da meta (de quem). \\
\textbf{Ambiente} & Contexto da aplicação da medição.           
\end{tabular}
\caption{Estruturação de metas em GQM}
\label{TabelaMetasGQM}
\end{table}

A meta de investigar a utilidade do uso de construções de \designbycontract{} na
\neoidl{} estruturada conforme o método GQM é apresentada na Tabela
\ref{TabelaMetasGQM}.

\begin{table}[h]
\centering
\vspace{0.5cm}
\begin{tabular}{r|lr}
\multicolumn{2}{c}{GQM}\\
\hline    
Analisar & contratos especificados em NeoIDL \\
Com o propósito de  & avaliá-los \\
Com relação a & facilidade de uso e utilidade das construções \\
& de \designbycontract{} \\
Do ponto de vista de & arquitetos e desenvolvedores experientes \\
No contexto de & um serviço real.           
\end{tabular}
\caption{Meta de investigar o uso de Dbc na \neoidl{} estruturado conforme
método GQM}
\label{TabelaMetasGQM}
\end{table}

Em termos de processo, a técnica GQM é organizada em seis etapas: (i)
desenvolver o conjunto de metas de medição associadas; (ii) gerar perguntas,
baseadas em modelos, que definam os objetivos tão bem quanto possível e de modo
quantificável; (iii) especificar as respostas necessárias para essas questões;
(iv) desenvolver os mecanismos para coleta de dados; (v) coletar, validar, e analisar a necessidade de ação corretiva; (vi) analisar
os dados para avaliar a conformidades com as metas correspondentes.

O modelo escolhido para subsidiar a elaboração das perguntas foi o
\textit{Technology Acceptance Model} -- TAM, proposto por Davis em 1989
\cite{davis1989perceived} e que possui uma sólida base teórica e ampla
utilização.
O principal objetivo do modelo TAM é identificar os motivos que levam à aceitação
ou rejeição de uma tecnologia. O modelo está estruturado em dois conceitos:
percepção da utilidade (\textit{perceived usefullness} - PU) e percepção sobre a
facilidade de uso (\textit{perceived ease of use} - PEU)
\cite{hernandes2010avaliaccao}.

Percepção de utilidade é medido com o grau em que uma pessoa acredita que
utilizar uma determinada tecnologia vai melhorar sua eficiência em suas
atividades. A percepção de utilidade está relacionada ao grau em que uma pessoa
acredita que determinada tecnologia será livre de esforço. Adicionamente, alguns
estudos incluem um outro aspecto, que é a predisposição ao uso
(\textit{self-predicted future usage}), que está relacionado à indicação sobre a
preferência de uma determinada tecnologia em detrimento de outras \cite{laitenberger1998evaluating}.

Os questionários baseados em TAM apresentam afirmações indicando que uma
determinada tecnologia é fácil de se usar e útil. As respostas normalmente são
escalas que vão desde a discordância total até a concordância total quanto a
afirmação apresentada. Essa escala é denominada \textit{Likert Scale}
\cite{allen2007likert}. As respostas são agrupadas para se avaliar se, no
conjunto, as respostas mais favoráveis indicam uma tendência de aceitação. O
processo de elaboração do questionário é apresentado na subseção \ref{ModelagemQuestionario}.

A distribuição do questionário foi feita por formulário eletrônico publicado na
Internet, utilizando a ferramenta \textit{Google Forms}. Essa ferramenta
simplifica o processo de escrita do questionário e também a consolidação das
informações. O \textit{link} para acesso ao questionário foi encaminhado a
grupos de arquitetos e desenvolvedores cujo nível técnico elevado era de
conhecimento dos pesquisadores. O questionário ficou disponível para respostas
por um período de três semanas, até que atingisse um volume de respostas satisfatório.

Os resultados obtidos foram consolidados em uma planilha eletrônica e
manipulados por meio do ferramental estatístico da linguagem R. A análise dos
dados obtidos é apresentada na subseção \ref{AnaliseQuestionario}.


\subsection{Modelagem do questionário}
\label{ModelagemQuestionario}

Uma vez que a
\neoidl{} tem como principal objetivo a modelagem de contratos de serviços REST,
a avaliação da aceitabilidade da \neoidl{} relativamente a outra linguagem com o
mesmo propósito pode indicar se o projeto da linguagem está alinhado às
expectativas dos desenvolvedores e arquitetos. Entretanto, é necessário
avaliar especificamente os efeitos da inclusão de construções de
\designbycontract{} na \neoidl{}, já que não há esse suporte nas outras linguagens para especificação
de contratos REST (subseção \ref{PropostaServicoDbC}).

O questionário foi organizado em três seções: a primeira seção têm cinco
perguntas para traçar o perfil do respondente, principalmente em relação a sua
experiência técnica; a segunda seção, com três perguntas, faz uma avaliação
sobre especificação formal de contratos e a sintaxe da
\neoidl{} e de Swagger \cite{swaggerSite}; 
Oito questões formam a última seção, a qual dá enfoque ao principal ponto, que é
a avaliação das utilidade do uso de \designbycontract{} na \neoidl{}.


\subsubsection{Primeira seção do questionário}

A Tabela \ref{Secao1Questionario} apresenta a lista de perguntas da primeira
seção.
A pergunta sobre o local de prestação de serviços tem a finalidade de 
identificar e, na fase de análise, eliminar respostas oriundas de locais para os quais o questionário
não foi submetido, pois este não exigia autenticação. A segunda e terceira
questões visam mapear se o respondente possui experiência e
conhecimento suficiente para responder adequadamente o questionário. A questões
quatro e cinco pretendem indicar o nível de especialização em \wss{} REST e
na linguagem Swagger.

\begin{table}[h]
\scriptsize
\centering
\vspace{0.5cm}
\begin{tabular}{r|p{10cm}|p{3cm}}
\multicolumn{3}{c}{{\normalsize Perfil técnico-profissional}}\\

\hline    
\textbf{no.} & \textbf{Questão} & \textbf{Respostas} \\
\hline    
1 & Para qual órgão ou empresa você presta serviços atualmente? & Questão aberta
\\
2 & A quanto tempo você trabalha com desenvolvimento Web? & Experiência, em
anos.\\
3 & A quanto tempo você desenvolve com uso de APIs Web (Web Service)? & Experiência, em
anos.\\
4 & Qual o seu nível de experiência com especificação de API REST? & Lista A
(Tabela \ref{respostasQ4e5}) \\
5 & Qual o seu nível de experiência com especificação de contratos com Swagger?
& Lista B (Tabela \ref{respostasQ4e5})\\

\end{tabular}
\caption{Questões da primeira seção do questionário que traçam o perfil
técnico-profissional}
\label{Secao1Questionario}
\end{table}

As questões 2 e 3 possuiam uma escala de respostas em que a primeira era
não ter experiência, as intermediárias agrupadas a cada dois
anos, e a última indicava experiência superior a dez anos. As respostas às
questões 4 e 5 estão listadas na Tabela \ref{respostasQ4e5}.

\begin{table}[!bth] 
\centering
\vspace{0.5cm}

\scriptsize
\begin{tabular}{p{1.5cm}|p{13cm}}
\hline   

\multirow{4}{*}{\textbf{Lista A}} & Nunca desenvolvi com a tecnologia
\\
& Já desenvolvi serviços algumas APIs REST, mas não conheço a
especificação \\
& Tenho experiência superior a dois anos em desenvolvimento REST, mas não conheço
a especificação \\
& Tenho experiência superior a dois anos em desenvolvimento REST e
conheço a especificação \\
\hline
\multirow{4}{*}{\textbf{Lista B}} & Não sei do que se trata ou apenas ouvi a
respeito\\
 & Já tive contato com especificações em Swagger \\
 & Já especifiquei APIs REST em Swagger\\
 & Tenho experiência superior a um ano em Swagger \\
\end{tabular}

\caption{Lista de respostas para as questões 4 e 5}
\label{respostasQ4e5}

\end{table}


\subsubsection{Segunda seção do questionário}

A partir da segunda seção do questionário, o
método TAM foi utilizado para nortear a elaboração das
perguntas.
O questionário TAM foi adaptado ao cenário de avaliação da aceitação de uma DSL,
pois, durante a pesquisa, não foram identificados estudos que utilizam TAM para
fazer estudo idêntico. Entretanto, a aplicação de TAM abrange um campo vasto
e as questões pode ser adaptadas ao objeto de estudo \cite{babar2007evaluating},
desde que os aspectos base do método sejam mantidos.

A base da segunda seção do questionário é um conjunto de especificações de
serviços. Inicialmente, é apresentada em texto
descritivo, a especificação de um serviço que deve recuperar uma lista de
informações, desde que atendida uma condição. Em seguida, o mesmo serviço é especificado em Swagger. Na sequência, a
especificação é feita em \neoidl{}. Ao final, um trecho do código Java que
implementa o serviço real é apresentado.

Como a \neoidl{} é uma DSL muito pouco conhecida, foram acrescentadas questões
sobre a sua expressividade, pois essa característica é considerada o principal
critério para a escolha de uma linguagem \cite{mackinlay1985expressiveness}. O
quão fácil é idenficar os elementos da linguagem e compreender o que eles
representam é fundamental para se decidir usar uma DSL. Além disso, espera-se de uma DSL, por
não ter o compromisso de atender a propósitos gerais, que ela seja mais
acessível \cite{taha2008domain}. 

\begin{table}[h]
\scriptsize
\centering
\vspace{0.5cm}
\begin{tabular}{r|p{10cm}|p{3cm}}
\multicolumn{3}{c}{{\normalsize Comparação de especificações}}\\

\hline    
\textbf{no.} &  \textbf{Afirmação} & \textbf{Aspecto TAM} \\
\hline    
6 & A especificação do contrato formalmente, seja em Swagger ou NeoIDL, em
relação a descrição textual, aumentará meu nível de acerto na implementação. &
Efetividade \\
7 & Identificar e compreender as operações e atributos na especificação Swagger
é simples para mim. & Clareza e compreensão \\
8 & Identificar e compreender as operações e atributos na especificação NeoIDL é
simples para mim. & Clareza e compreensão \\
 
\end{tabular}
\caption{Questões da segunda seção do questionário com comparação de
especificações}
\label{Secao2Questionario}
\end{table}

As questões da segunda seção são apresentadas na Tabela
\ref{Secao2Questionario}. Na questão 6, é feita uma afirmação comparando-se a
especificação textual com especificação formal, com o objetivo de avaliar a
efetividade em se adotar uma linguagem formal. As questões 7 e 8 avaliam a
clareza e compreensão das especificações feitas em Swagger e \neoidl{}. O
objetivo dessas duas questões é fazer uma comparação entre as duas linguagens
nesses aspectos. As respostas utilizaram a escala Likert \cite{allen2007likert}
com cinco níveis.


\subsubsection{Terceira seção do questionário}

A terceira seção é a principal, pois enfoca na utilização de construções de
\designbycontract{} em contratos REST. Essa seção possui uma parte introdutória
onde é apresentada uma conceituação de \designbycontract{}. Em seguida, é
apresentada a especificação de serviço da seção anterior em \neoidl{}
acrescentando a ela	 as construções de \designbycontract{}. Após essa parte
introdutória, são feitas quatro questões (9 a 12) sobre a simplicidade e
utilidade do novo contrato em \neoidl{}.

Na sequência, é apresentado um trecho curto sobre geração automática de código a
partir de especificações formais. O propósito dessa parte é tratar também da
utilidade do recurso de geração de código provido pela \neoidl{}. Duas perguntas
são feitas sobre esse ponto (13 e 14). Por fim, são feitas duas afirmações (15
e 16) sobre a predisposição ao uso futuro da \neoidl{}.


\begin{table}[h]
\scriptsize
\centering
\vspace{0.5cm}
\begin{tabular}{r|p{10cm}|p{3cm}}
\multicolumn{3}{c}{{\normalsize Construções de \designbycontract{} na \neoidl{}}}\\

\hline    
\textbf{no.} &  \textbf{Afirmação} & \textbf{Aspecto TAM} \\
\hline    
9 & Conhecer previamente e explicitamente as precondições será útil para mim. &
Util \\
10 & Aprender a identificar as precondições na NeoIDL parece ser simples pra
mim. & Facilidade de aprender \\
11 & Parece ser fácil para mim declarar uma precondição na NeoIDL. & Clareza e compreensão \\
12 & Me lembrar da sintaxe da precondição na NeoIDL é fácil. & Fácil de
lembrar \\
\hline 
13 & É claro e compreensível para mim o efeito da precondição sobre o código
gerado. & Controlável \\
14 & A geração do código de pré e pós-condições aumentará minha produtividade na
implementação do serviço. & Eficiência \\
\hline  
15 & Assumindo ter a disposição a NeoIDL no meu trabalho, para especificação de
contratos e geração de código, eu presumo que a utilizarei regularmente no
futuro. & Adoção \\
16 & Nesse mesmo contexto, eu vou preferir utilizar contratos escritos em NeoIDL
do que descritos de outra forma. & Preferência \\
\end{tabular}
\caption{Questões da terceira seção do questionário sobre \designbycontract{}
na \neoidl{}}
\label{Secao3Questionario}
\end{table}


\subsection{Análise dos Resultados}
\label{AnaliseQuestionario}
\vspace{-6mm}

Esta subseção apresenta e discute os resultados do questionário aplicado. As
respostas são avaliadas em grupos, iniciando pela avaliação do perfil de
respondentes. Em seguida, são avaliadas as questões 6 a 8, sobre a especificação
formal de contratos e a expressividade de Swagger e \neoidl{}. As respostas às
questões 9 a 12, sobre a aplicabilidade de \designbycontract{} na especificação
 de contratos REST, são debatidos na sequência. A penúltimo grupo trata da
 geração de código com \designbycontract{}, com as questões 13 e 14. Por fim,
 são discutidas as duas últimas questões, sobre predisposição ao uso.

\subsubsection{Perfil dos respondentes}

A primeira questão tinha o objetivo de apenas identificar respostas
submetidas por pessoas fora do público alvo. Cinco instituições das quais se
conhecia o nível técnico e com as quais se tinha contato com profissionais
concentraram a maior parte das respostas. As demais respostas são empresas
relacionas a essas instituições, por indicação de técnicos experientes
do primeiro grupo. Nenhuma das respostas foi considerada anormal e, portanto,
todas foram validadas, totalizando 26 respondentes.

A segunda questão levantou o tempo de experiência dos respondentes com
desenvolvimento de sistemas Web. O resultado foi sumarizado na Tabela
\ref{quantidadeRespostas2}. Como o público alvo, definido na meta GQM (Tabela
\ref{TabelaMetasGQM}), é de desenvolvedores experientes, os respondentes que
apontaram experiência inferior a três anos estão fora do perfil esperado e suas
respostas foram descartadas (linhas em cinza). Assim, o questionário teve um
conjunto de 21 respostas válidas.

\begin{table}[!bth] 
\centering
\vspace{0.5cm}
\scriptsize
\begin{tabular}{p{3cm}|p{1cm}}
\hline   
Experiência com desenvolvimento Web& Q2 \\
\hline   
\cellcolor{light-gray}Não tenho experiência &  \cellcolor{light-gray}2  \\
\cellcolor{light-gray}Entre 1 e 2 anos      &  \cellcolor{light-gray}3  \\
Entre 3 e 4 anos      &  1  \\
Entre 5 e 6 anos      &  3  \\
Entre 7 e 8 anos      &  5  \\
Entre 9 e 10 anos     &  1  \\
Há mais de 10 anos    & 11  \\
\end{tabular}
\caption{Respostas sobre a experiência com desenvolvimento Web}
\label{quantidadeRespostas2}
\end{table}

A terceira questão identificou o conhecimento dos respondentes com
desenvolvimento de \wss{}. Conforme apresentado na Tabela
\ref{quantidadeRespostas3}, as respostas estão polarizadas, em que 40\%
(quarenta por cento) tem nenhuma ou pouca experiência (menos de dois anos). Por
outro lado, 23\% (vinte e três por cento) dos respondentes possui experiência
muito alta (mais de 10 anos). Caso o volume de respostas fosse muito grande (mais de
200 pessoas), poderia ser feita uma análise comparativa da \neoidl{} de acordo
com a experiência de \ws{}.


\begin{table}[!bth] 
\centering
\vspace{0.5cm}
\scriptsize
\begin{tabular}{p{3cm}|p{1cm}}
\hline   
Experiência com \ws{} & Q3 \\
\hline   
Não tenho experiência &  3  \\
Entre 1 e 2 anos      &  5  \\
Entre 3 e 4 anos      &  6  \\
Entre 5 e 6 anos      &  2  \\
Entre 7 e 8 anos      &  0  \\
Entre 9 e 10 anos     &  0  \\
Há mais de 10 anos    &  5  \\
\end{tabular}
\caption{Respostas sobre experiência com \wss{}}
\label{quantidadeRespostas3}
\end{table}        

A questão quatro identificou o conhecimento e experiência do respondente com o
modelo arquitetural REST. Apenas dois (menos de 10\%) respondentes não tem
experiência alguma com desenvolvimento de \wss{} REST, como pode ser verificado na Tabela
\ref{quantidadeRespostas4}. Outros 42\% (quarenta de dois por cento) dos
respondentes, porém, tem experiência e conhecimento sobre os padrões arquiteturais que formam
o modelo REST.

\begin{table}[!bth] 
\centering
\vspace{0.5cm}
\scriptsize
\begin{tabular}{p{10cm}|p{1cm}}
\hline   
Experiência com REST & Q4 \\
\hline   
Nunca desenvolvi com a tecnologia &  2  \\
Já desenvolvi serviços algumas APIs REST, mas não conheço a especificação & 7 \\
Tenho experiência superior a dois anos em desenvolvimento REST, mas não conheço
a especificação & 3 \\
Tenho experiência superior a dois anos em desenvolvimento REST e conheço a
especificação &  9  \\

\end{tabular}
\caption{Respostas sobre conhecimento e experiência com REST}
\label{quantidadeRespostas4}
\end{table}

A última questão sobre o perfil do respondente tratou do conhecimento e
experiência com a especificação de contratos em Swagger. O conjunto de respostas
confirmou a tendência pelo mercado pela uso de Swagger. Mais de 70\% (setenta
por cento) dos respondentes já tiveram algum contato com especificação de
contratos em Swagger (Tabela \ref{quantidadeRespostas5}). Esse cenário indica
que o conjunto de profissionais é qualificado e possui capacidade crítica para
responder às demais questões.

\begin{table}[!bth] 
\centering
\vspace{0.5cm}
\scriptsize
\begin{tabular}{p{10cm}|p{1cm}}
\hline   
Experiência com Swagger & Q5 \\
\hline   
Não sei do que se trata ou apenas ouvi a respeito & 6 \\
Já tive contato com especificações em Swagger & 8 \\ 
Já especifiquei APIs REST em Swagger &  5  \\
Tenho experiência superior a um ano em Swagger & 2 \\


\end{tabular}
\caption{Conhecimento e experiência dos respondentes com Swagger}
\label{quantidadeRespostas5}
\end{table}

  
   

\subsubsection{Análise da especificação de contratos formais e expressividade
de Swagger e \neoidl{}}

A partir da questão 6, as respostas foram dadas na escala \textit{Likert} sobre
as afirmações colocadas. Os dados foram processados com utilização do software
estatístico R, com o objetivo de apresentar os resultados por meio de gráficos
que facilitam a interpretação de tendência. A cor vermelha é utilizada para
indicar discordância sobre as afirmações, sendo que o vermelho mais escuro
corresponde a discordância total. Da mesma forma, a cor azul é utilizada para
indicar concordância.

Na questão 6 foi apresentada uma afirmação sobre a efetividade de se especificar
contratos formalmente, em detrimento de especificação textual descritiva.
Conforme apresentado na primeira barra da Figura \ref{Respostas6a8},
praticamente todos os respondentes concordaram com a afirmação e nenhum
discordou. Esse resultado reforça a utilidade de se utilizar contratos para
especificação de serviços.


\begin{figure}[!htb]
\centering
\includegraphics[width=100mm,trim = 6mm 115mm 6mm 
10mm,clip]{img/GraficoResultadoQuestoes6a8.pdf}

\includegraphics[width=100mm,trim = 6mm 0mm 6mm 
170mm,clip]{img/GraficoResultadoQuestoes6a8.pdf}

\caption{Gráfico com o resultado das questões 6 a 8}
\label{Respostas6a8}
\end{figure}

A questão 7 consultou a percepção dos respondentes sobre a facilidade de
identificar as operações e atributos em um contrato especificado em Swagger. A
questão 8 contém a mesma afirmação sobre um contrato especificado em \neoidl{}.
O resultado de ambas questões estão na Figura \ref{Respostas6a8} e indicam que
tanto os contratos em Swagger como em \neoidl{} são claros e fáceis de se compreender. Vale ressaltar que a maior parte dos
respondentes já possuia contato anterior com Swagger (Tabela
\ref{quantidadeRespostas5}), o que reforça o resultado positivo para a
\neoidl{}.


\subsubsection{Análise sobre aceitabilidade de construções de \designbycontract{}}

Entre as questões 9 e 12 foram feitas afirmações sobre a facilidade de se
identificar os elementos e a utilidade de se especificar pré e pós-condições em
contratos de serviços REST com a \neoidl{}. Os resultados são apresentados na
Figura \ref{Respostas9a12}. 

\begin{figure}[!htb]
\centering
\includegraphics[width=100mm,trim = 6mm 115mm 6mm 
10mm,clip]{img/GraficoResultadoQuestoes9a12.pdf}

\includegraphics[width=100mm,trim = 6mm 0mm 6mm 
170mm,clip]{img/GraficoResultadoQuestoes9a12.pdf}

\caption{Gráfico com o resultado das questões 9 a 12}
\label{Respostas9a12}
\end{figure}

A questão 9 tem a afirmação de que o conhecimento explícito de precondição terá
utilidade para o desenvolvedor responsável por implementar o serviço. Quase
todos os respondentes concordaram com a afirmação, em maior e menor grau. Um
pequeno conjunto discordou, indicando que outros elementos podem ser
necessários para esse conjunto identificar a utilidade. Pode haver influência
ainda da experiência do respondente com implementação de serviços.

A afirmação da questão 10 é sobre a facilidade de se identificar pré e
pós-condições em contrato na \neoidl{}. Na questão 11, o enfoque é sobre a
facilidade de se declarar pré e pós-condições na \neoidl{}. Nos dois casos,
nenhum dos respondentes manifestou discordância, sendo que a concordância total foi mais
intensa para a afirmação da questão 10. Apenas uma pequena parcela informou que
nem concorda nem discorda da afirmação.

A questão 12 afirma que é fácil se lembrar da sintaxe de uma precondição na
\neoidl{}. Essa questão está relacionada a como se pode associar mentalmente as
contruções sintáticas ao efeito que elas geram. Muito embora se esperasse um
resultado mais dividido, pois ser uma construção influenciada por linguagens
diversas, os resultados indicam que a sintaxe é fácil de ser lembrada.

\subsubsection{Análise da geração de código para construções de \designbycontract{}}

Duas questões foram inseridas para tratar da utilidade da geração de código com
a \neoidl{} para contratos com suporte a \designbycontract{}. A questão 13
apontava ser claro o efeito da precondição sobre o código gerado. A maior parte
dos respondentes indicou que a geração do código para a precondição é util
em relação a produção de efeitos controláveis. A Figura \ref{Respostas13e14}
apresenta os resultados.

\begin{figure}[!htb]
\centering
\includegraphics[width=100mm,trim = 6mm 115mm 6mm 
10mm,clip]{img/GraficoResultadoQuestoes13e14.pdf}

\includegraphics[width=100mm,trim = 6mm 0mm 6mm 
170mm,clip]{img/GraficoResultadoQuestoes13e14.pdf}

\caption{Gráfico com o resultado das questões 13 e 14}
\label{Respostas13e14}
\end{figure}

A questão 14 afirmou que a geração de código sobre precondições ampliará a
produtividade de implementação do serviço. O resultado apontou que uma parte
expressiva dos respondentes não concordou nem discordou. Para esses, a
especificação de precondições não incluencia a produtividade.
Por outro lado, tem-se que a maior parte concordou com a contribuição positiva
da geração de código das precondições para a produtividade. Uma parcela muito pequena discordou, ou seja,
eles acreditam que a geração de código para precondição prejudicará o
desempenho.

\subsubsection{Análise sobre a predisposição ao uso}

As duas últimas questões tratam de quão inclinado ao uso da \neoidl{} em seu
ambiente de trabalho estaria o respondente.
Esta análise é relativa \cite{laitenberger1998evaluating}, ou seja, depende de
quais os benefícios trazidos pela \neoidl{} em relação aos benefícios trazidos por soluções alternativas, como
especificação textual de contratos ou ainda por meio de Swagger. Os resultados
para essas questões são apresentadas na Figura \ref{Respostas15e16}.

\begin{figure}[!htb]
\centering
\includegraphics[width=100mm,trim = 6mm 115mm 6mm 
10mm,clip]{img/GraficoResultadoQuestoes15e16.pdf}

\includegraphics[width=100mm,trim = 6mm 0mm 6mm 
170mm,clip]{img/GraficoResultadoQuestoes15e16.pdf}

\caption{Gráfico com o resultado das questões 15 e 16}
\label{Respostas15e16}
\end{figure}

A questão 15 apresenta uma afirmação sobre a predisposição ao uso regular da
\neoidl{} no futuro. Pelos resultados da pesquisa, verifica-se que a maior
parcela dos respondentes nem concorda nem discorda. Esse resultado é relevante,
pois, conforme debatido anteriormente (Tabela \ref{quantidadeRespostas5}), boa
parte dos respondentes já tem contato com Swagger, que consiste em uma
alternativa direta à \neoidl{}.
Se, por um lado os respondentes não concordam prontamente em utilizar \neoidl{}, por
outro não descartam.

É necessário investigar que fatores influenciariam positivamente a predisposição
do uso da \neoidl{} e verificar se são fatores internos às empresas em que
trabalham atualmente os respondentes ou quais pontos podem ser melhorados na
\neoidl{} a fim de ampliar o nível de aceitação. Em relação às demais respostas,
nota-se uma tendência mais forte de concordância sobre o uso que regular, que
a de discordância, o que reforça o potencial de adesão ao uso da \neoidl{}.

A última questão abordou diretamente se haveria preferência de uso pela
\neoidl{} em detrimento a outras linguagens. O resultado é o mais equilibrado do
questionário, havendo uma pequena tendência para a preferência pela \neoidl{}.
O resultado da questão 16 levanta ainda outra hipótese sobre a quantidade de
respondentes em dúvida na questão 15: fatores individuais, como não lidar
diretamente com especificação de contratos, podem explicar haver um grande grupo
que não sabe informar se utilizaria regularmente a \neoidl{}.

Assim, essas questões abrem outras perspectivas de estudo, com um enfoque sobre
os fatores institucionais e o que mais poderia ser desenvolvido na \neoidl{}
para atender às necessidades das equipes de desenvolvimento.

\subsection{Ameaças}
\label{AmeacaoEstudoSubjetivo}

Questionários são instrumentos práticos para se realizar estudos empíricos que
visam obter a opniões subjetivas. Uma de suas principais vantagens está em
permitir a participação de várias pessoas, ampliando a abrangência temporal e
geográfica. Há também mais domínio e facilidade de análise sobre os dados
coletados.

Entretanto, algumas ameaças podem influenciar a conclusão dos resultados. Para o
questionário aplicado neste trabalho de pesquisa, podemos destacar alguns pontos
que podem influenciar ou limitar a qualidade dos dados, bem como direcionar para
a realização de outros estudos empíricos sobre o tema.

Um primeiro ponto a observar é a quantidade de respondentes. Embora seja um
número superior a de alguns outros estudos avaliados
\cite{albuquerque2015quantifying}
\cite{hernandes2010avaliaccao} \cite{laitenberger1998evaluating}, uma maior
quantidade de participante ampliaria a qualidade dos dados, pois eliminaria influência de fatores muito
específicos a um conjunto de respondentes, e possibilitaria análise de 
cenários, utilizando visões de dados. O nível técnicos dos respondentes,
contudo, reduz os impactos dessa ameaça.

Outro ponto de melhoria são ações que visem proporcionar um conhecimento mínimo
sobre a \neoidl{}. Todas as respostas foram dadas apenas pelas informações
apresentadas no próprio questionário e os resultados poderiam ser diferentes se
houvesse um treinamento prévio sobre a \neoidl{}. Sob esse aspecto, tem-se que a
compreensão sobre dos elementos da sintaxe básica na \neoidl{} (questão 6) teve
resultado inferior que a compreensão dos elementos de \designbycontract{}
(questão 8). Esse resultado é inesperado, pois pressupõe-se que identificar os
elementos de \designbycontract{} só seja possível após compreender a sintaxe
básica.

Ainda, como o questionário foi aplicado apenas uma vez, não foram incluídos
pontos de melhoria a partir do primeiro conjunto de respostas para uma segunda
avaliação. Conforme debatido na subseção \ref{AnaliseQuestionario}, algumas
questões poderiam ser inseridas em uma nova aplicação do questionário, de
modo produzir informações que permitissem conclusões mais completas.

Em estudos futuros, essas melhorias podem ser aplicadas. Em que pese esses
pontos de atenção, o estudo realizado por meio do questionário atigiu seus
resultados e produziu dados de grande relevancia para o objeto em pesquisa.


