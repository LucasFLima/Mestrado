\chapter{AVALIAÇÃO SUBJETIVA}
\vspace{-6mm}

%%%%
A language's expressiveness is the major criterion for choosing a language to
state a given set of facts: a language that cannot express the facts should not
be used. However, additional criteria are needed to choose among languages that
are sufficiently expressive  for a set of facts. Two of these criteria are how
ease it is to state the facts in the language and how easy is to perceive the
facts once they are stated.

expressividade de uma linguagem é o principal critério para a escolha de uma
linguagem para indicar um determinado conjunto de fatos: uma linguagem que não
podem expressar os fatos não devem ser usados. No entanto, os critérios
adicionais são necessários para escolher entre os idiomas que são
suficientemente expressivo para um conjunto de fatos. Dois desses critérios são
quão fácil é expor os fatos na linguagem e como é fácil de perceber os fatos,
uma vez que são demonstrados.

\cite{mackinlay1985expressiveness}

%%%%


\section{Utilidade da NeoIDL com DbC}
\vspace{-6mm}

\subsection{Método}
\vspace{-6mm}

\subsection{GQM}
\vspace{-6mm}

\subsection{Questionário}
\vspace{-6mm}

\subsection{Análise dos Resultados}
\vspace{-6mm}


* Questionário montado para avaliar a utilidade de DbC com NeoIDL

* Inicialmente motivado pelo estudo do Alessandro Garcia

* GQM e Avaliação TAM

* Montagem do questionário

1. Perfil técnico-profissional do respondente
1.1 Para qual órgão ou empresa você presta serviços atualmente?
1.2 A quanto tempo você trabalha com desenvolvimento Web
1.3 A quanto tempo você desenvolve com uso de APIs Web (Web Service)
1.4 Qual o seu nível de experiência com especificação de API REST
1.5 Qual o seu nível de experiência com especificação de contratos com Swagger

3 Questões sobre especificação e implementação de APIs Web
3.1 A especificação do contrato formalmente, seja em Swagger ou NeoIDL, em
relação a descrição textual, aumentará meu nível de acerto na implementação (efetividade).
3.2 Identificar e compreender as operações e atributos na especificação Swagger
é simples para mim.
3.3 Identificar e compreender as operações e atributos na especificação NeoIDL é
simples para mim.

4. DbC
4.1 Conhecer previamente e explicitamente as pré-condições será útil para mim.
(Useful)
4.2 Aprender a identificar as pré-condições na NeoIDL parece ser simples pra mim
(Easy to learn)
4.3 Parece ser fácil para mim declarar uma pré-condição na NeoIDL
(Clear and understandable)
4.4 Me lembrar da sintaxe da pré-condição na NeoIDL é fácil  (Remember)

5. Geração de código
5.1 É claro e compreensível para mim o efeito da pré-condição sobre o código
gerado  (Controllable)
5.2 A geração do código de pré e pós-condições aumentará minha produtividade na
implementação do serviço (Job performance)
5.3 Assumindo ter a disposição a NeoIDL no meu trabalho, para especificação de
contratos e geração de código, eu presumo que a utilizarei regularmente no futuro.
5.4 Nesse mesmo contexto, eu vou preferir utilizar contratos escritos em NeoIDL
do que descritos de outra forma



* Distribuição do questionário

* Avaliação dos resultados

* Ameaças
- não foi fornecido nenhum material sobre a NeoIDL, apresentando somente o uma
descrição de serviço
- O questionário foi aplicado uma única vez, sem melhorias a partir do primeiro
conjunto de respostas


* Questionários futuros

A principal questão de pesquisa a ser avaliada com o uso do questionário é a utilidade em se agregar ao design das especificações de serviços REST as garantias de pré e
pós-condições. Em segundo momento, pressupondo a utilidade, avaliar se a NeoIDL
cumpre satisfatoriamente com este propósito, agregando à sintaxe da linguagem
a possibilidade de se expressar pré e pós-condições.

* Separar os respondentes em faixas de experiência. Verificar se as respostas
dos menos experientes precisam ser descartadas pela pouca capacidade crítica.
Separar a análise entre os respondentes que conhecem e os que não conhecem
Swagger.

* Perspectivas de comparação
a) Experiência com desenvolvimento com uso de REST
b) Experiência com Swagger
c) Utilidade da especificação formal de contratos
d) Percepção da NeoIDL sem DbC





