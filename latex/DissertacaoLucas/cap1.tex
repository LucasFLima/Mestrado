

%%%%%%%%%%%%%%%%%% in�cio dos cap�tulos
% cap�tulo 1
\chapter{INTRODUÇÃO}
\pagenumbering{arabic}
\vspace{-6mm}

A computação orientada a serviços ( \emph{Service-oriented computing, SOC)} tem
se mostrado uma solução de \textit{design} de \textit{software} que favorece o
alinhamento às mudanças constantes e urgentes nas instituições
\cite{chen2008towards}. Nessa abordagem, os recursos de software são empacotados
como serviços, os quais são módulos bem definidos e auto-contidos, provêem funcionalidades negociais e com estado e
contexto independente \cite{papazoglou2007service}.

Os benefícios de SOC estão diretamente relacionados ao
baixo acoplamento dos serviços que compõem a solução, de forma que as partes
(nesse caso serviços) possam ser substituídas e evoluídas facilmente, ou ainda
rearranjadas em novas composições. Contudo, para que isso seja possível, é
necessário que os serviços possuam contratos bem definidos e independentes da
implementação.

A relação entre quem provê e quem consome o serviço se
dá por meio de um contrato. O contrato de serviço é o documento que descreve os
propósitos e as funcionalidades do serviço, como ocorre a troca de mensagens, condições sobre
como as operações são realizadas e informações sobre as operações \cite{erl2009web}.

Nesse contexto, a qualidade da especificação do contrato é fundamental para o
projeto de software baseado em SOC. Este trabalho de pesquisa aborda um aspecto
importante para a melhoria da robustez de contratos de serviços: a construção de
garantias mútuas por meio da especificação formal de contratos, agregando o
conceito de Desing-by-Contract.

\section{PROBLEMA DE PESQUISA}
\vspace{-6mm}

As linguagens de especificação de contratos para SOC apresentam
algumas limitações. Por exemplo, a linguagem WSDL (\emph{Web-services
description language}) \cite{zur2005developing} é considerada uma solução
verbosa que desestimula a abordagem \textit{Contract First}. Por essa razão,
especificações WSDL são usualmente derivadas a partir de anotações em código
fonte \textit{Code First}.
Além disso, os conceitos descritos em contratos na linguagem WSDL não são
diretamente mapeados aos elementos que compõem as interfaces do estilo
arquitetural REST (\emph{Representational State Transfer}).
Outras alternativas para REST, como Swagger e
RAML\footnote{http://raml.org/spec.html}, usam linguagens de propósito geral (em
particular JSON e YAML) adaptadas para especificação de contratos. Ainda que
façam uso de contratos mais sucintos que WSDL, essas linguagens não se
beneficiam da clareza típica das linguagens específicas para esse fim (como IDLs CORBA) e não oferecem
mecanismos semânticos de extensibilidade e modularidade.

Com o objetivo de mitigar esses problemas, a linguagem \neoidl{} foi proposta
para simplificar a especificação de serviços REST com mecanismos de modularização,
suporte a anotações, herança em tipos de dados definidos pelo desenvolvedor, e
uma sintaxe simples e concisa semelhante às \textit{Interface Description
Languages} -- IDLs -- presentes em \textit{Apache Thrift}\texttrademark e
CORBA\texttrademark. Por outro lado, a \neoidl{}, da mesma forma que WSDL,
Swagger e RAML não oferece construções para especificação de contratos formais de
comportamento como os presentes em linguagens que suportam DBC (\emph{Design by
Contract})~\cite{meyer1992applying}, como JML, Spec\# e Eiffel. Em outras
palavras, a \neoidl{}  admite apenas contratos fracos (\textit{weak contracts}),
sem suporte a construções como pré e pós condições.



\section{OBJETIVO GERAL}
\vspace{-6mm}

O objetivo geral de trabalho é investigar o uso de construções de
\designbycontract{} no contexto de computação orientada a serviços, verificando a
viabilidade e utilidade de sua adoção na especificação de contratos e
implentação de serviços REST.

\subsection{Objetivos específicos}
\vspace{-6mm}

\begin{enumerate}
  \item Realizar análise empírica de expressividade e reuso da especificação de
  contratos em \neoidl{} em comparação com \textit{Swagger}, a partir de contratos
  reais do Exército Brasileiro.
  \item Extender a sintaxe da \neoidl{} para admitir construções de 
  \designbycontract, com pré e pós condições para operações de serviços REST.
  \item Implementar um estudo de caso de geração de código em \textit{Python
  Twisted} com suporte a \designbycontract a partir de contratos especificados
  em \neoidl
  \item Coletar a percepção de desenvolvedores sobre a aceitação da
  especificação de contratos REST com \designbycontract na \neoidl{}
\end{enumerate}


\subsection{Justificativa e relevância}
\vspace{-6mm}

A necessidade de integração entre sistemas de várias origens e tecnologias fez
aumentar a adoção de soluções baseada em computação orienta a serviços. Isso de
deve justamente de busar tornar a interoperabilidade de soluções heterogênias
o menos acopladas possível, de modo a que mudanças nos requisitos de negócio ou
na inclusão de novos serviços seja simples, eficiente e rápida.

O uso de \ws{} é a forma mais comum de se implementar os serviços. O
desenvolvimetno de \ws{}, que era inicialmente construído sobre a abordagem
SOAP, com o tráfego de mensagens codificadas em XML, tem gradativamente se
intensificado no sentido da utilização de REST.

Uma dos principais benefícios do uso de SOC está na possibilidade de reuso de
seus componentes. Porém, reuso requer que serviços bem construídos e precisos em
relação a sua especificação \cite{jazequel1997design}. A qualidade e precisão do
contrato de serviço torna-se assim um elemento fudamental para que auferir os benefícios da
abordagem SOC.

Nesse contexto, REST não dispõe de um meio padrão para especificação de
contratos. Linguangens como Swagger, YAML e WAML cumprem com o propósito de
especificar contratos REST, porém padecem do mesmo problema: são voltados para
computadores e de escrita não trivial para humanos, o que prejudica a prática de
\CtFirst{}. A linguagem \neoidl{} foi concebida com o objetivo de ser mais
expressiva para humanos.

Todas essas linguagens tem, entretanto, um outra limitação em comum: não dão
suporte a contratos robustos, com garantias. A estratégia para superar essa
limitação foi de buscar no paradigma de orientação a objetos, que é uma das
principais influências de orientação a serviços \cite{erl2009web},
o conceito de \designbycontract{}. Ambas as abordagens, orientação a serviços e
a objetos, tem em comum a ênfase no reuso e comunicação entre componentes
(serviços e classes).

A proposta deste trabalho de
pesquisa de mestrado está justamente em incluir garantias na especificação de
contratos REST, extendendo a linguagem \neoidl{} para suportar construções de
\designbycontract{}.



\subsection{Estrutura}
\vspace{-6mm}

Este trabalho está organizado em quatro capítulos\ldots

% Introdução - Estudos empíricos / DbC para NeoIDL
% Fundamentação teórica - SOC / REST / Contratos / DbC - 10 páginas
% NeoIDL - Objetivos / Resultados / Extensão da Linguagem / Avaliação empírica
% Análise da Aceitação da NeoIDL - Método / GQM / Resultados
% Conclusão 

