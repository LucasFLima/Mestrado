

%%%%%%%%%%%%%%%%%% in�cio dos cap�tulos
% cap�tulo 1
\chapter{INTRODUÇÃO}
\pagenumbering{arabic}
\vspace{-6mm}

A computação orientada a serviços ( \emph{Service-oriented computing, SOC)} tem
se mostrado uma solução de \textit{design} de \textit{software} que favorece o
alinhamento às mudanças constantes e urgentes nas instituições
\cite{chen2008towards}. Nessa abordagem, os recursos de software são empacotados
como serviços, os quais são módulos bem definidos e auto-contidos, provêem funcionalidades negociais e com estado e
contexto independente \cite{papazoglou2007service}.

Os benefícios de SOC estão diretamente relacionados ao
baixo acoplamento dos serviços que compõem a solução, de forma que as partes
(nesse caso serviços) possam ser substituídas e evoluídas facilmente, ou ainda
rearranjadas em novas composições. Contudo, para que isso seja possível, é
necessário que os serviços possuam contratos bem definidos e independentes da
implementação.

A relação entre quem provê e quem consome o serviço se
dá por meio de um contrato. O contrato de serviço é o documento que descreve os
propósitos e as funcionalidades do serviço, como ocorre a troca de mensagens, condições sobre
como as operações são realizadas e informações sobre as operações \cite{erl2009web}.

Nesse contexto, a qualidade da especificação do contrato é fundamental para o
projeto de software baseado em SOC. Este trabalho de pesquisa aborda um aspecto
importante para a melhoria da robustez de contratos de serviços: a construção de
garantias mútuas por meio da especificação formal de contratos, agregando o
conceito de Desing-by-Contract.

\section{PROBLEMA DE PESQUISA}
\vspace{-6mm}

As linguagens de especificação de contratos para SOC apresentam
algumas limitações. Por exemplo, a linguagem WSDL (\emph{Web-services
description language}) \cite{zur2005developing} é considerada uma solução
verbosa que desestimula a abordagem \textit{Contract First}. Por essa razão,
especificações WSDL são usualmente derivadas a partir de anotações em código
fonte \textit{Code First}.
Além disso, os conceitos descritos em contratos na linguagem WSDL não são
diretamente mapeados aos elementos que compõem as interfaces do estilo
arquitetural REST (\emph{Representational State Transfer}).
Outras alternativas para REST, como Swagger e
RAML\footnote{http://raml.org/spec.html}, usam linguagens de propósito geral (em
particular JSON e YAML) adaptadas para especificação de contratos. Ainda que
façam uso de contratos mais sucintos que WSDL, essas linguagens não se
beneficiam da clareza típica das linguagens específicas para esse fim (como IDLs CORBA) e não oferecem
mecanismos semânticos de extensibilidade e modularidade.

Com o objetivo de mitigar esses problemas, a linguagem \neoidl foi proposta
para simplificar a especificação de serviços REST com mecanismos de modularização,
suporte a anotações, herança em tipos de dados definidos pelo desenvolvedor, e
uma sintaxe simples e concisa semelhante às \textit{Interface Description
Languages} -- IDLs -- presentes em \textit{Apache Thrift}\texttrademark e
CORBA\texttrademark. Por outro lado, a \neoidl, da mesma forma que WSDL, Swagger
e RAML não oferece construções para especificação de contratos formais de
comportamento como os presentes em linguagens que suportam DBC (\emph{Design by
Contract})~\cite{meyer1992applying}, como JML, Spec\# e Eiffel. Em outras
palavras, a \neoidl  admite apenas contratos fracos (\textit{weak contracts}),
sem suporte a construções como pré e pós condições.



\section{OBJETIVO GERAL}
\vspace{-6mm}

O objetivo geral de trabalho é investigar o uso de construções de
\designbycontract{} no contexto de computação orientada a serviços, verificando a
viabilidade e utilidade de sua adoção na especificação de contratos e
implentação de serviços REST.

\subsection{Objetivos específicos}
\vspace{-6mm}

\begin{enumerate}
  \item Realizar análise empírica de expressividade e reuso da especificação de
  contratos em \neoidl em comparação com \textit{Swagger}, a partir de contratos
  reais do Exército Brasileiro.
  \item Extender a sintaxe da \neoidl para admitir construções de 
  \designbycontract, com pré e pós condições para operações de serviços REST.
  \item Implementar um estudo de caso de geração de código em \textit{Python
  Twisted} com suporte a \designbycontract a partir de contratos especificados
  em \neoidl
  \item Coletar a percepção de desenvolvedores sobre a aceitação da
  especificação de contratos REST com \designbycontract na \neoidl
\end{enumerate}


\subsection{JUSTIFICATIVA E RELEVÂNCIA}
\vspace{-6mm}

- aumento do uso de SOC
- potencial do SOC está no baixo acomplamento dos serviços.
- aumento do uso de REST
- contratos fracos prejudicam a qualidade/aumentam os erros
- baixa qualidade prejudica o reuso

O uso do padrão REST para construção de \ws é crescente 
no contexto do desenvolvimento de soluções basedas em serviço e não se dipõe de
uma linguagem padrão para especificação de contratos fortes. A linguagem
específica de domínio \neoidl foi desenvolvida para ser uma alternativa para
especificação REST, caracterizada pela coesão e simplicidade em se compreender,
mas que não dispunha de suporte a pré e pós condições.



-- Usar na motivação
There is a simple lesson here: Reuse without a precise specification
mechanism is a disastrous risk.
Effective reuse requires design by contract.
Without a precise specification attached to each reusable component—
precondition, postcondition, invariant— no one can trust a supposedly reusable
component. Without a specification, it is probably safer to redo than to reuse.
\cite{jazequel1997design}



-- Usar na motivação --
The problem gets aggravated by
the fact that modern software applications are expected to
make use of these reusable modules as much as possible,
in an effort to reduce both the development costs and the
production time. Unfortunately, this situation opens the door
for the propagation of security vulnerabilities among several
applications as a result of the incorrect enforcement of security
properties in reusable software modules and threatens the
security and safety of applications as a whole.
\cite{rubio2013verifying}


Solution logic designed in accordance with service-orientation can be qualified with
“service-oriented,” and units of service-oriented solution logic are referred to as “services.”
As a design paradigm for distributed computing, service-orientation can be compared
to object-orientation (or object-oriented design). Service-orientation, in fact, has
many roots in object-orientation and has also been influenced by other industry developments,
including EAI, BPM, and Web services.
\cite{erl2009web}


\subsection{ESTRUTURA}
\vspace{-6mm}

% Introdução - Estudos empíricos / DbC para NeoIDL
% Fundamentação teórica - SOC / REST / Contratos / DbC - 10 páginas
% NeoIDL - Objetivos / Resultados / Extensão da Linguagem / Avaliação empírica
% Análise da Aceitação da NeoIDL - Método / GQM / Resultados
% Conclusão 

