

%%%%%%%%%%%%%%%%%% in�cio dos cap�tulos
% cap�tulo 1
\chapter{INTRODUÇÃO}
\pagenumbering{arabic}
\vspace{-6mm}

A computação orientada a serviços ( \emph{Service-oriented computing, SOC)} tem
se mostrado uma solução de \textit{design} de \textit{software} que favorece o
alinhamento às mudanças constantes e urgentes nas instituições
\cite{chen2008towards}. Nessa abordagem, os recursos de software são empacotados
como serviços, módulos bem definidos e auto-contidos, que provêem
funcionalidades negociais e com independência de estado e contexto
\cite{papazoglou2007service}.

Os benefícios de SOC estão diretamente relacionados ao
baixo acoplamento dos serviços que compõem a solução, de forma que as partes
(nesse caso serviços) possam ser substituídas e evoluídas facilmente, ou ainda
rearranjadas em novas composições. Contudo, para que isso seja possível, é
necessário que os serviços possuam contratos bem escritos e independentes da
implementação.

A relação entre quem provê e quem consome o serviço se
dá por meio de um contrato. O contrato de serviço é o documento que descreve os
propósitos e as funcionalidades do serviço, como ocorre a troca de mensagens,
informações sobre as operações e condições para sua execução \cite{erl2009web}.

Nesse contexto, a qualidade da especificação do contrato é fundamental para o
projeto de software baseado em SOC. Este trabalho de pesquisa aborda um aspecto
importante para a melhoria da robustez de contratos de serviços: a construção de
garantias mútuas por meio da especificação formal de contratos, agregando o
conceito de \designbycontract{}.

\section{PROBLEMA DE PESQUISA}
\vspace{-6mm}

As linguagens de especificação de contratos para SOC apresentam
algumas limitações. Por exemplo, a linguagem WSDL (\emph{Web-services
description language}) \cite{WSDLSite} é considerada uma solução
pouco expressiva que desestimula a abordagem \textit{Contract First}. Por essa
razão, especificações WSDL são usualmente derivadas a partir de anotações em código
fonte \textit{Code First}.
Além disso, os conceitos descritos em contratos na linguagem WSDL não são
diretamente mapeados aos elementos que compõem as interfaces do estilo
arquitetural REST\cite{fielding2000architectural} (\emph{Representational State Transfer}).

Outras alternativas para REST, como Swagger\cite{swaggerSite} e
RAML\cite{RAML}, usam linguagens de propósito geral (em
particular JSON\cite{JSon} e YAML\cite{YAML}) adaptadas para especificação de
contratos.
Ainda que façam uso de contratos mais sucintos que WSDL, essas linguagens não se
beneficiam da clareza típica das linguagens específicas para esse fim (como
a IDL\footnote{Interface Description
Languages.} CORBA\cite{corba}) e não oferecem
mecanismos semânticos de extensibilidade e modularidade.

Com o objetivo de mitigar esses problemas, a linguagem \neoidl{} foi proposta
para simplificar a especificação de serviços REST com mecanismos de modularização,
suporte a anotações, herança em tipos de dados definidos pelo desenvolvedor, e
uma sintaxe simples e concisa semelhante às IDLs presentes em \emph{Apache
Thrift}\texttrademark\cite{thrift} e \emph{CORBA}\texttrademark\cite{corba}.

Por outro lado, a \neoidl{}, da mesma forma que WSDL,
Swagger e RAML não oferece construções para especificação de contratos formais
com aspecto comportamental como os presentes em linguagens que
suportam DbC (\emph{Design by Contract})~\cite{meyer1992applying}, como
Eiffel, JML e Spec\#. Em outras palavras, a \neoidl{}  admite apenas
contratos fracos (\textit{weak contracts}), sem suporte a construções com pré e
pós-condições.



\section{OBJETIVO GERAL}
\label{ObjGeral}
\vspace{-6mm}

O objetivo geral deste trabalho é investigar o uso de construções de
\designbycontract{} no contexto de computação orientada a serviços, verificando a
viabilidade e utilidade de sua adoção na especificação de contratos e
implementação de serviços REST. Para atingir o objetivo geral, os seguintes 
objetivos específicos foram definidos, sendo diretamente mapeados nas 
principais contribuições do trabalho. 

%\subsection{Objetivos específicos}
\vspace{-6mm}

\begin{enumerate}[(OE1)]
  \item Realizar análise empírica de expressividade e reuso da especificação de
  contratos em \neoidl{} em comparação com \textit{Swagger}, a partir de contratos
  reais do Exército Brasileiro (Seção \ref{EstudoExpressividadeReuso}); 
  \item Extender a sintaxe da \neoidl{} para admitir construções de 
  \designbycontract, com pré e pós-condições para operações de serviços REST
  (Seção \ref{extensaoNeoIDL-DbC});
  \item Incorporar à infraestrutura de \textit{Plugins} da
  \neoidl{} a capacidade de geração de código para o \framework{} \textit{Python
 Twisted} (Seção \ref{pluginTwisted});
  \item Implementar regras de transformação que traduzem construções de DbC
 \neoidl{} em código de validação para o framework \textit{Python
 Twisted} (Seção \ref{pluginTwisted});% com suporte a \designbycontract{} a
 % partir de contratos especificados em \neoidl;
  \item Coletar a percepção de desenvolvedores sobre a aceitação da
  especificação de contratos REST com \designbycontract{} na \neoidl{} (Seção
  \ref{analiseSubjetiva}).
\end{enumerate}


\section{JUSTIFICATIVA E RELEVÂNCIA}
\label{JustificativaRelevancia}
\vspace{-6mm}

A demanda por integração entre sistemas de várias origens e tecnologias diversas
fez aumentar a adoção de soluções baseada em computação orientada a serviços.
Isso se deve justamente à necessidade de tornar a interoperabilidade de soluções
heterogênias o menos acopladas possível, de modo que mudanças nos requisitos de negócio ou
na inclusão de novos serviços sejam atendidas com simplicidade,
eficiência e rapidez.

O uso de \ws{}\cite{erl2009web} é a forma mais comum de se implementar os serviços. O
desenvolvimento de \ws{}, que eram inicialmente projetados sobre a abordagem
SOAP, com o tráfego de mensagens codificadas em XML\cite{XML}, tem gradativamente
se intensificado no sentido da utilização de REST\cite{fielding2000architectural}.

Um dos principais benefícios do uso de SOC está na possibilidade de reuso de
seus componentes. Porém, reuso requer serviços bem construídos e precisos em
relação a sua especificação \cite{jazequel1997design}. A qualidade e precisão do
contrato de serviço torna-se claramente um elemento fudamental para se auferir
os benefícios da abordagem SOC.

Nesse contexto, REST não dispõe de um meio padrão para especificação de
contratos. Linguangens como Swagger, YAML e WADL cumprem com o propósito de
especificar contratos REST, porém padecem do mesmo problema: são voltados para
computadores e de escrita e leitura complexa para humanos, o que
prejudica a prática de \CtFirst{}. A linguagem \neoidl{} foi concebida com o objetivo de ser mais
expressiva para humanos, além de outros propósitos, como extensibilidade e
modularidade.

Todas essas linguagens tem, entretanto, uma outra limitação em comum: não dão
suporte a contratos robustos, com garantias. A estratégia para superar essa
limitação foi de buscar no paradigma de orientação a objetos, que é uma das
principais influências de orientação a serviços \cite{erl2009web},
o conceito de \designbycontract{}. Ambas as abordagens, orientação a serviços e
a objetos, tem em comum a ênfase no reuso e comunicação entre componentes
(serviços e classes).

A principal contribuição deste trabalho de
pesquisa de mestrado está em incluir garantias na especificação de
contratos REST, extendendo a linguagem \neoidl{} para suportar construções de
\designbycontract{}.



\section{ESTRUTURA}
\vspace{-6mm}

Este trabalho está organizado em quatro capítulos. O capítulo 2 faz uma revisão
teórica sobre o tema computação orientada a serviços (SOC), seus propósitos,
princípios e terminologia. São tratadas estratégias recomendadas para se
atingir os resultados esperados com SOC, com enfâse na prática de padronização
do contrato, sobretudo quanto a abordagem \CtFirst{} e ao uso \wss{}. Ao final,
é feita uma apresentação do conceito de \designbycontract{}, sua finalidade e
implementações.

O capítulo 3 descreve a \neoidl{}, uma linguagem e \framework{} de geração de
código para especificação e geração de serviços REST. É contextualizada a
motivação para a criação da \neoidl{}, sua estrutura sintática e componentes.
Esse capítulo inclui um estudo empírico realizado no decorrer da pesquisa
de mestrado sobre expressividade e reuso da \neoidl{} em comparação a Swagger,
em um cenário real do Exército Brasileiro.

As principais contribuições desta dissertação estão concentradas no capítulo 4.
Esse capítulo se inicia com a proposta para incorporação de construções de pré e
pós-condições à especificação de contratos e qual o modelo de operação do
serviço em execução. A extensão da linguagem é debatida por etapas, apresentando
as condições mais simples, seguindo para as mais completas. Um estudo de caso
com a geração de código para um serviços em \textit{Python Twisted} com suporte
a DbC é apresentado. A última seção demonstra os resultados de um estudo
empírico subjetivo sobre a percepção de utilidade e de facilidade de uso da \neoidl{} com DbC.

Por fim, o capítulo 5 apresenta as conclusões do trabalho de pesquisa e sugere a
realização de trabalhos futuros.
