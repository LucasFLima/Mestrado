 
\chapter{REFERENCIAL TEÓRICO}
\vspace{-6mm}

\section{COMPUTAÇÃO ORIENTADA A SERVIÇO}
\vspace{-6mm}


As empresas precisam estar preparadas para responder rápida e
eficientemente a mudanças impostas por novas regulações, por aumento de
competição ou ainda para usufruir de novas oportunidades. No contexto atual, em que as
informações fluem de modo extremamente veloz, o tempo disperdiçado pelas organizações para se
adaptar a um novo cenário tem um preço elevado, gerando expressiva perda de
receita e, em um determinados casos, podendo causar a falência.

No campo das instituições governamentais, a eficiência na condução das ações do
Estado impõem que a estrutura de troca de informações entre os mais variados
entes seja continuamente adaptável, mutuamente integrada. Pode-se tomar como
exemplo a edição de nova lei que implique alteração no cálculo do tempo de
serviço para aposentadoria. A nova fórmula deve se propagar para ser
aplicada em várias intituições que compõem a máquina pública.

Nessas situações, os sistemas de informação das organizações devem possibilitar
que a dinâmica de adaptação ocorra sem demora, sob pena de, em vez de serem
ferramental para apoiar continuamente os processos de negócio, se tornem entrave
para a ágil incorporação dos novos processos. Por outro lado, a nova
configuração deve se manter integra e funcional com o já complexo cenário de TI.

A eficiência na integração entre as soluções de TI é determinante para que se
consiga alterar uma parte sem comprometer todo o ecossistema. A integração
possibilita a combinação de eficiência e flexibilidade de recursos para otimizar
a operação através e além dos limites de uma organização e a habilita para
inteoperar facilmente \cite{papazoglou2008service}.

A computação orientada a serviços -- SOC -- endereça essas necessidades em uma
plataforma que aumenta a flexibilidade e melhora o alinhamento com o negócio, a
fim de reagir rapidamente a mudanças nos requisitos de negócio. Para obter esses
benefícios, os serviços devem cumprir com determinados quesitos, que incluem
alta autonomia ou baixo acoplamento \cite{erl2008soa}. Assim, o paradigma de SOC
está voltado para o projeto de soluções preparadas para constantes mudanças,
substituindo-se pequenas peças -- os serviços -- por outras.

Portando, o objetivo da SOC é conceber um estilo de projeto, tecnologia e
processos que permitam às empresas desenvolver, interconectar e manter suas
aplicações e serviços corporativos com eficiência e baixo custo. Embora esses
objetivos não sejam novos, SOC procura superar os esforços prévios como
programação modular, reuso de código e técnicas de desenvolvimento orientadas a
objetos \cite{papazoglou2007serviceApprTechRechIss}.

As vertentes mais visionárias -- não ainda concretizada e utópica para muitos
pesquisadores -- da computação orientada a serviços prevêem uma coordenação de
serviços cooperantes por todo o mundo, onde os componentes possam ser conectados
facilmente em uma rede de serviços pouquíssimo acoplados e, assim, criar
processos de negócio dinâmicos e aplicações ágeis entre organizações e plataformas de
computação \cite{leymann2005combining}.


\subsection{Terminologia}
\vspace{-6mm}

\begin{description}
\item [Computação orientada a serviço] é um termo \textit{guarda-chuva} para
descrever uma nova geração de computação distribuída. Desse modo, é um conceito
que engloba várias coisas, como paradigmas e princípios de projeto, catálogo de
padrões de projeto, padronização de linguagem, modelo arquitetural específico, e
conceitos correlacionados, tecnologias e plataformas.
A computação orientada a serviços é baseada em modelos anteriores de computação
distribuída e os extendem com novas camadas de projeto, aspectos de governança,
e uma grande gama de tecnologias de implementações especializadas, em grande
parte baseadas em \ws{} \cite{erl2009web}.

\item [Orientação a serviço] é um paradígma de projeto cuja intenção é a criação
de unidades lógicas moldadas individualmente para podem serem utilizadas
conjutamente e repetidamente para se atender a objetivos e funções específicos
associados com SOA e computação orientada a serviço.

A lógica concebida de acordo com orientação a serviço pode ser designada de
\textbf{orientada a serviço}, e as unidades da lógica orientada a serviço são
referenciadas como \textbf{serviços}. Como um paradigma de computação
distribuída, a orientação a serviço pode ser comparada a orientação a objetos,
de onde advém várias de suas raízes, além da influência de EAI, BMP e \ws
\cite{erl2009web}.

A orientação a serviços é composta principalmente de oito princípios de projeto
(descritos na seção \ref{PrincipiosSOA}).

\item [Arquitetura orientada a serviço - SOA] representa um modelo arquitetural
cujo objetivo é elevar a agilidade e a redução de custos e ao mesmo tempo
reduzir o peso da TI para a organização. Isso é feito colocando o serviço no
como elemento central da representação lógica da solução \cite{erl2009web}.

Como uma arquitetura tecnológica, uma implementação SOA consiste da combinação
de tecnologias, produtos, APIs, extensões da infraestrutura, etc. A implantação
concreta de uma arquitetura orientada a serviço é única para cada organização,
entretanto é caracterizada pela introdução de tecnologias e plataformas que
suportam a criação, execução e evolução de soluções orientadas a serviços. O
resultado é a formação de um ambiente projetado para produzir soluções alinhadas
aos princípios de projeto de orientação a serviço.

Segundo Thomas Erl \cite{erl2009web}, o termo arquitetura orientada a serviço --
SOA -- vem sendo amplamente utilizado na mídia e nos produtos de divugação de
fabricantes que tem se tornado quase que sinônimo de computação orientada a
serviço -- SOC.

\item [Serviço] é a unidade da solução no qual foi aplicada a orientação a
serviço. É a aplicação da orientação dos princípios de projeto de orientação a
serviço que distigue uma unidade de lógica como um serviço comporada a outras
unidades de serviços que podem existir isoladamente como um objeto ou
componente \cite{erl2009web}.

Após a modelagem conceitual do serviço, os estágios de projeto e desenvolvimento
produzem um serviço que é programa de \textit{software} independente com
características específicas para suportar a realização dos objetivos associados
a computação orientada a serviço.

Cada serviço possui um contexto funcional distinto e é composto de uma lista
de capacidades relacionadas a esse contexto. Então um serviço pode ser
considerado um conjunto de capacidades descritas em seu contrato.


\item [Contrato de serviço] é o conjunto de documentos que expressam as
meta-informações do serviço, sendo a parte fundamental a que descreve a
sua interface técnica. Eles compõem o contrato técnico do serviço, cuja essência
é estabelecer uma API com as funcionalidades providas pelo serviço por meio de
suas capacidades \cite{erl2009web}.

Os serviços implementados como \ws SOAP normalmente são descritos em seu WSDL
\footnote{\ws{} \textit{Description Language}}, \textit{XML schemas} and
políticas (\textit{WS-policy}). Já os serviços implementados como \ws{} REST não
possuem uma linguagem padrão para especificação de contratos. Já foram propostas
algumas alternativas como WADL \cite{hadley2006web}, Swagger \cite{swaggerSite},
e \neoidl{} \cite{lima2015neoidl}.

O contrato de serviço também pode ser composto de documentos de leitura humana,
como os que descrevem níveis de serviços (\textit{SLA}), comportamentos e
limitações. Muitas dessas características também podem ser descritas em
linguagens formais (para processamento computacional).

No contexto de orientação a serviço, o projeto do contrato do serviço é de suma
importância de tal forma que o princípio de projeto contrato de serviço
padronizado é dedicado exclusivamente para se padronizar a criação dos contratos
de serviços \cite{erl2009web}.

\end{description}


\subsection{Modelo arquitetural}
\vspace{-6mm}

O 	

In contrast to conventional software architectures pri-
marily delineating the organization of a system in its
(sub)systems and their interrelationships, the SOA cap-
tures a logical way of designing a software system to
provide services to either end-user applications or other
services distributed in a network through published and
discoverable interfaces. SOA is focused on creating a
design style, technology, and process framework that will
allow enterprises to develop, interconnect, and maintain
enterprise applications and services efficiently and cost-
effectively.
\cite{papazoglou2007serviceApprTechRechIss}


Service orientation provides the underlying implementation that can make an on demand IT
operating environment a reality by supporting the functions of both integration and infra-
structure management [Lymann 2005a].
Service-Oriented Computing (SOC) utilizes services as the constructs to support the devel-
opment of rapid, low-cost and easy composition of distributed applications. Services are
autonomous, platform-independent computational entities that can be used in a platform
independent way. Services can be described, published, discovered, and dynamically as-
sembled for developing massively distributed, interoperable, evolvable systems. Services
perform functions that can range from answering simple requests to executing sophisticated
business processes requiring peer-to-peer relationships between possibly multiple layers of
service consumers and providers. Any piece of code and any application component de-
ployed on a system can be reused and transformed into a network-available service. Serv-
ices reflect a "service-oriented" approach to programming, based on the idea of composing
applications by discovering and invoking network-available services rather than building new
applications or by invoking available applications to accomplish some task [Papazoglou
2003]. Services are most often built in a way that is independent of the context in which
they are used. This means that the service provider and the consumers are loosely coupled.
This "service-oriented" approach is independent of specific programming languages or oper-
ating systems.


\cite{erl2008soaDesigPatterns}


\vspace{-6mm}

\subsection{Princípios SOA}
\label{PrincipiosSOA} 
\vspace{-6mm}

O paradigma de orientação a serviço é estruturada em oito princípios
fundamentais \cite{erl2009web}. São eles que caraterizam a abordagem SOA e a sua
aplicação fazem com que um serviço se diferencie de um componente ou de
um módulo.
Os contratos de serviços permeiam a maior parte destes princípios:

\begin{description}
\item[Contrato padronizado] - \textbf{Serviços dentro de um mesmo inventário
estão em conformidade com os mesmos padrões de contrato de serviço}. 
Os contratos de serviços são elementos fundamentais na arquitetura orientada
a serviço, pois é por meio deles que os serviços interagem uns com os outros e
com potenciais consumidores. Este princípio tem como foco principal o contrato de serviço e seus requisitos. O padrão de projeto \textit{contract firts} é
uma consequência direta deste princípio \cite{erl2009web}. 

\item[Baixo acomplamento] - \textbf{Os contratos de serviços impõem aos
consumidores do serviço requisitos de baixo acoplamento e são, os próprios
contratos, desacoplados do seu ambiente}. 
Este princípio também possui forte relação com o contratos de serviço, pois a
forma como o contrato é projetado e posicionado na arquitetura é que gerará o
benefício do baixo acoplamento. O projeto deve garantir que o contrato
possua tão somente as informações necessárias para possibilitar a compreensão e
o consumo do serviço, bem como não possuir outras características que gerem
acoplamento.

São considerados negativos, e que devem ser evitados, os acoplamentos  
\begin{enumerate}[label=(\alph*)] 
\item do contrato com as funcionalidades que ele suporta, agregando ao
contrato características dos processos que o serviço suporta,
\item do contrato com a sua implementação, invertendo a estratégia de conceber
primeiramente o contrato
\item do contrato com a sua lógica interna, expondo aos consumidores
características que levem os consumidores a inadivertidamente aumentarem o
acoplamento
\item do contrato com a tecnologia do serviço, causando impactos idesejáveis em
caso de substituição de tecnologia.
\end{enumerate}

Por outro lado, há um acoplamento positivo que o que gera dependência da
lógica em relação ao contrato \cite{erl2009web}. Ou seja, idealmente a
implementação do serviço deve ser derivada do contrato, pondendo se ter inclusive a geração de código a
partir do contrato.


\item[Abstração] - \textbf{Os constratos de serviços devem conter apenas
informações essenciais e as informações sobre os serviços são limitadas àquelas
publicadas em seus contratos}. O contrato é a forma oficial a partir da qual o
consumidor do serviço faz seu projeto e tudo o que está além do contrato deve
ser desconhecido por ele. Por um lado este princípio busca a ocultação
controlada de informações. Por outro, visa a simplificação de informações do
contrato de modo a assegurar que apenas informações essenciais estão
disponíveis.


\item[Reusabilidade]- \textbf{Serviços contém e expressam lógica agnóstica e
podem ser disponibilizados como recursos reutilizáveis}. Este princípio
contribui para se entender o serviço como um produto e seu contrato com uma API
genérica para potenciais consumidores. Essa abordagem aplicada ao projeto dos
serviços leva a desenhá-lo com lógicas não dependentes de processos de negócio
específicos, de modo a torná-los reutilizáveis em vários processos.

\item[Autonomia]- \textbf{Serviços exercem um elevado nível de controle sobre
o seu ambiente em tempo de execução}. O controle do ambiente não está ligado a
dependência do serviço à sua plataforma em termos de projeto, mas sim ao aumento
da confiabilidade sobre a execução e redução da dependência dos recursos não se tem controle. 
O que se busca é a previsibilidade sobre o comportamento do serviço.

\item[Ausência de estado] - \textbf{Serviços reduzem o consumo de recursos
restringindo a gestão de estado das informações apenas a quando for necessário}.
Este princípio visa reduzir ou mesmo remover a sobrecarga gerada pelo
gerenciamento do estado de cada operação, aumentando a escalabilidade da
plataforma de arquitetura orientação a serviço como um todo. Na composição do
serviço, o serviço deve armazenar apenas os dados necessários para completar o
processamento, enquanto se aguarda o processamento de outro serviço.

\item[Descoberta de serviço] - \textbf{Serviços devem conter metadados por meio
dos quais os serviços possam ser descobertos e interpretados}. Tornar cada
serviço de fácil descoberta e interpreção pelas equipes de projeto é o foco
deste princípio. Os próprios contratos de serviço devem ser projetados para
incorporar informações que auxiliem na sua descoberta.

\item[Composição] - \textbf{Serviços são participantes efetivos de composição,
independentemente do tamanho ou complexidade da composição}. O princípio da
composição faz com que os projetos de serviços sejam projetados para
possibilitar que eles se tornem participantes de composições. Deve-se levar em
conta, entretanto, os outros princípios no planejamento de uma nova composição,
considerando a complexidade de composições formadas.

\end{description}
 
 
% avaliar a necessidade de falar de design patterns
%\subsubsection{Padrões de projeto }
%\vspace{-6mm}

\ldots

\vspace{-6mm}

\section{Web Services}
\vspace{-6mm}

\ldots

\subsection{SOAP (W3C) }
\vspace{-6mm}

\ldots

\subsubsection{Especificação de contratos}
\vspace{-6mm}

\ldots

\vspace{-6mm}

\subsection{REST (Fielding)}
\label{secaoREST}
\vspace{-6mm}

A tecnologia REST é caracterizada principalmente pela
convenção da adoção dos métodos do protocolo HTTP para definição das operações, transferência de estado nas requisições.
- Ausência de padrão para especificação de contratos REST.
- Fragilidade de garantias por quem consome o serviço;
- Risco de a ausência de informações prejudicarem os consumidores.

\subsubsection{Especificação de contratos}
\vspace{-6mm}

\ldots


\subsubsection{Outros padrões }
\vspace{-6mm}

\ldots

\section{Design-by-Contract}
\label{Design-by-Contract}
\vspace{-6mm}

\ldots

\subsection{Origem}
\vspace{-6mm}

- Bertrand Meyer (eifel)

\subsection{Implementações de DbC}

 - Eifel
 - JML
 - Spec\#


