 
\chapter{REFERENCIAL TEÓRICO}
\vspace{-6mm}
% 
\section{COMPUTAÇÃO ORIENTADA A SERVIÇO}
\vspace{-6mm}

\ldots

\vspace{-6mm}

\subsection{Motivação }
\vspace{-6mm}

\ldots

\vspace{-6mm}

\subsection{Modelo arquitetural}
\vspace{-6mm}
\ldots

\vspace{-6mm}

\subsection{Princípios}
\vspace{-6mm}
\ldots

\vspace{-6mm}

\subsubsection{Padrões de projeto }
\vspace{-6mm}

\ldots

\vspace{-6mm}

\section{Web Services}
\vspace{-6mm}

\ldots

\subsection{SOAP (W3C) }
\vspace{-6mm}

\ldots

\subsubsection{Especificação de contratos}
\vspace{-6mm}

\ldots

\vspace{-6mm}

\subsection{REST (Fielding)}
\label{secaoREST}
\vspace{-6mm}

A tecnologia REST é caracterizada principalmente pela
convenção da adoção dos métodos do protocolo HTTP para definição das operações, transferência de estado nas requisições.
- Ausência de padrão para especificação de contratos REST.
- Fragilidade de garantias por quem consome o serviço;
- Risco de a ausência de informações prejudicarem os consumidores.

\subsubsection{Especificação de contratos}
\vspace{-6mm}

\ldots


\subsubsection{Outros padrões }
\vspace{-6mm}

\ldots

\section{Design-by-Contract}
\label{Design-by-Contract}
\vspace{-6mm}

\ldots

\subsection{Origem}
\vspace{-6mm}

- Bertrand Meyer (eifel)

\subsection{Implementações de DbC}

 - Eifel
 - JML
 - Spec\#


\section{NeoIDL}
\vspace{-6mm}

\ldots

\subsection{Objetivos}
\vspace{-6mm}


\subsection{Arquiterura}
\vspace{-6mm}


\subsection{Histórico}
\vspace{-6mm}
