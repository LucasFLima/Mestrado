\chapter{CONCLUSÕES E TRABALHOS FUTUROS}
\vspace{-6mm}

\section{CONCLUSÕES}
\vspace{-6mm}

A necessidade de estratégias de integração de sistemas e soluções adaptáveis às
constantes necessidade des mudanças tem levado às empresas a, cada vez mais,
adotarem o modelo de computação orientada a serviços -- SOC
\cite{papazoglou2008service} \cite{erl2009web}. A qualidade da especificação do
serviço por meio de seu contrato é um dos fatores determinantes para o sucesso
do uso de SOC.

A \neoidl{} foi criada para ser uma alternativa às linguagens de especificação
de \wss REST, uma vez que estas possuem uma sintaxe pouco expressiva para humanos,
além de não disporem de mecanismos com suporte a extensibilidade e modularização. O estudo
empírico da comparação entre especificações Swagger e \neoidl{} demonstrou a capacidade
expressiva e potencial de reuso da \neoidl{}. 
Entretanto, nem a \neoidl{}, nem as demais linguagens possibilitavam especificar 
contratos robustos, como os existentes em linguagens com suporte a \designbycontract{}.

Esse trabalho apresentou uma extensão da \neoidl{} para possibilitar a especificação
de contratos REST com suporte a pré e pós-condições e, a partir do contrato, permitir
a geração de código de serviços REST com essas garantias, seguindo a abordagem
\CtFirst{}. A proposta se baseou no paradgima de orientação a objetos, uma das 
principais influências da orientação a serviços.

Essa proposta foi submetida ao \textit{feedback} de profissionais experientes e também ao 
\textit{Workshop} de teses de dissertações do WTDSoft 2015. Os elementos sintáticos
de linguagens com suporte a \designbycontract{} como Eiffel, JML e Spec\# foram avaliados
e proporcionaram a criação de novas construções sintáticas para a \neoidl{} mantendo a harmonia com 
a linguagem pre-existente, sem que se perdesse o potencial para criação de regras de validação flexíveis e abrangentes.

A \neoidl{} passou a permitir a validação dos parâmetros de entrada e dados
de saída de uma requisição (por meio de pré e pós-condições básicas) assim como
permitir também fazer requisição a outros serviços para realizar validações mais complexas, por meio de pequenas composições de serviços. A construção de um \textit{Plugin Twisted} demonstrou ser
viável produzir código que realize, em tempo de execução, a validação das regras estabelecidas no contrato, sem que o 
desenvolvedor tenha que se preocupar com elas e direcione seu esforço para a implementação das regras
de negócio.

A avaliação subjetiva, realizada por meio de questionário baseado nas técnicas GQM e TAM, apresentou
resultados satisfatórios à hipótese de pesquisa, em que grande parte dos respondentes indicou que a \neoidl{} com suporte
a \designbycontract{} é uma ferramenta com potencial de adoção, demonstrando ser uma linguagem e um 
\framework{} úteis e fáceis de serem utilizados sob a perspectiva de quem escreve contratos e implementa
serviços. 

Ficou demonstrado, no contexto de avaliação desse trabalho, que os conceitos de \designbycontract, quais sejam,
expressar direitos e obrigações entre os clientes e fornecedores de recursos, proporcionando qualidade na
análise, projeto, implementação e comunicação, são também aplicáveis no paradigma de orientação
a serviços. Ademais, que serviços baseados em \wss REST podem ser simples e leves sob a ótica
da especificação mas também serem robustos sobre a ótica da estabilidade e comportamento.


\section{TRABALHOS FUTUROS}
\vspace{-6mm}

O estudo sobre especificação de contratos para serviços REST é um campo de
pesquisa aberto, sobretudo pelo fato de não haver um formato oficial,
estabelecido pelo W3C ou outra entidade de padrões internacionais como o IEEE.
Este trabalho explorou o aspecto do uso de construções de \designbycontract{}
em serviços REST que, embora tenha produzido resultados promissores, ainda não
fecha a questão por completo.

Ainda sobre a proposta apresentada nessa dissertação, a análise dos resultados
(Seção \ref{AnaliseQuestionario}) levanta a possibilidade de exploração de
algumas respostas, em especial dos fatores que levaram uma parte importante dos
respondentes a informar que tem dúvida (nem discordam nem concordam) com o
adoção da \neoidl{} em suas atividades. Um novo questionário poderia ser
elaborado sobre esse ponto.

A realização de um experimento que avalie o uso por completo da abordagem de
pré e pós-condições desde a especificação dos requisitos, a elaboração dos
contratos, até os testes dos serviços implementados em um cenário hipotético
poderia trazer mais insumos para a melhoria da \neoidl{}. Ainda mais relevante
seria a experimentação da \neoidl{} com \designbycontract{} em um cenário de serviços reais.

Sob a perspectiva de implementação, podem ser elaborados \textit{plugins} para
outras linguagens de programação além de \textit{Python Twisted} com suporte a
\designbycontract{}, de modo a ampliar o potencial de utilização do
\framework{}. O projeto da \neoidl{} também pode incorporar mecanismos que
facilitem a especificação de contratos, como recursos ambientes
de desenvolvimento (IDEs) e \textit{code complete}. Ainda um portal de onde
podessem ser gerados os códigos sem a necessidade de instalação local do
\framework{}.

Um outro ponto é exploração do potencial da \neoidl{} na disponibilização de
recursos de \textit{hypermedia} \cite{webber2010rest}, fornecendo mecanismos
em que o cliente possa interagir com o servidor para identificar os serviços
que deseja seguir. A tranformação bidirecional entre o código gerado e a
especificação do contrato, útil em cenários de manutenção, também pode ser
explorada.

