\chapter{CONCLUSÕES E TRABALHOS FUTUROS}
\vspace{-6mm}

\section{CONCLUSÕES}
\vspace{-6mm}

A necessidade de estratégias de integração de sistemas e soluções adaptáveis às
constantes necessidade des mudanças tem levado às empresas a, cada vez mais,
adotarem o modelo de computação orientada a serviços -- SOC
\cite{papazoglou2008service} \cite{erl2009web}. A qualidade da especificação do
serviço por meio de seu contrato é um dos fatores determinantes para o sucesso
do uso de SOC.

A \neoidl{} foi criada para ser uma alternativa às linguagens de especificação
de \wss REST, uma vez que estas possuem uma sintaxe pouco expressiva para humanos,
além de não disporem de mecanismos com suporte a extensibilidade e modularização. O estudo
empírico da comparação entre especificações Swagger e \neoidl{} demonstrou a capacidade
expressiva e potencial de reuso da \neoidl{}. 
Entretanto, nem a \neoidl{}, nem as demais linguagens possibilitavam especificar 
contratos robustos, como os existentes em linguagens com suporte a \designbycontract{}.

Esse trabalho apresentou uma extensão da \neoidl{} para possibilitar a especificação
de contratos REST com suporte a pré e pós-condições e, a partir do contrato, permitir
a geração de código de serviços REST com essas garantias, seguindo a abordagem
\CtFirst{}. A proposta se baseou no paradgima de orientação a objetos, uma das 
principais influências da orientação a serviços.

Essa proposta foi submetida ao \textit{feedback} de profissionais experientes e também ao 
\textit{Workshop} de teses de dissertações do WTDSoft 2015. Os elementos sintáticos
de linguagens com suporte a \designbycontract{} como Eiffel, JML e Spec\# foram avaliados
e proporcionaram a criação de novas construções sintáticas para a \neoidl{} mantendo a harmonia com 
a linguagem pre-existente, sem que se perdesse o potencial para criação de regras de validação flexíveis e abrangentes.

A \neoidl{} passou a permitir a validação dos parâmetros de entrada e saída de uma requisição (por meio
de pré e pós-condições básicas) assim como permitir também fazer requisição a outros serviços para realizar validações mais
complexas, por meio de pequenas composições de serviços. A construção de um \textit{Plugin Twisted} demonstrou ser
viável produzir código que realize, em tempo de execução, a validação das regras estabelecidas no contrato, sem que o 
desenvolvedor tenha que se preocupar com elas e direcione seu esforço para a implementação das regras
de negócio.

A avaliação subjetiva, realizada por meio de questionário baseado nas técnicas GQM e TAM, apresentou
resultados satisfatórios à hipótese de pesquisa, em que grande parte dos respondentes indicou que a \neoidl{} com suporte
a \designbycontract{} é uma ferramenta com potencial de adoção, demonstrando ser uma linguagem e um 
\framework{} úteis e fáceis de serem utilizados sob a ótica de quem escreve contratos e implementa
serviços. 

Ficou demonstrado, no contexto de avaliação desse trabalho, que os conceitos de \designbycontract, quais sejam,
expressar direitos e obrigações entre os clientes e fornecedores de recursos, proporcionando qualidade na
análise, projeto, implementação e comunicação, são também aplicáveis no paradigma de orientação
a serviços. Ademais, que serviços baseados em \wss REST podem ser simples e leves sob a ótica
da especificação mas também serem robustos sobre a ótica da estabilidade e comportamento.


\section{TRABALHOS FUTUROS}
\vspace{-6mm}



- Trabalhos relacionados com hipermedia
