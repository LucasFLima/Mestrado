\section{Exemplo de DBC para SOC}\label{sec:exemplobdc}

Nesta seção apresentamos dois exemplos do uso de DBC para SOC. O primeiro caso,
mais simples, consiste em estabelecer um contrato que restringe o argumento
passado para o serviço. O segundo exemplo é mais elaborado, no qual se
estabelece como pré-condição o resultado positivo na chamada de um outro
serviço. Satisfeita a pré-condição, o serviço principal garante, por meio de
outra chamada, que a necessidade do cliente é atendida.

\subsection{Exemplo simples}

O trecho de módulo \neoidl constante da figura \ref{lst:DBCSimple} destaca a
capacidade \emph{incluirItem} (linha 8). Para que se tenha o item incluído no
catálogo, é necessário fornecer um valor para o atributo descrição
(obrigatório). A pré-condição (linha 6) se inicia com a notação \emph{/@pre} e
possui validação sobre o atributo descrição. Note que o atributo é precedido
da palavra \emph{old}, que indica o valor do atributo antes do processamento do
serviço. De forma inversa, o prefixo \emph{new} indicaria o valor do atributo
após a execução do serviço. É estabelecido ainda a cláusula \emph{/@otherwise}
na linha 7. Sua função é informar ao cliente do serviço a falha no atendimento da pré-condição,
retornando o código HTTP 412 (Falha de pré-condição).
 
\begin{figure}[htb]
\begin{small}
\lstinputlisting[language=NeoIDL,firstnumber=1]{DBCsimple.neo}
\end{small}
\caption{Exemplo da notação DBC no \neoidl}
\label{lst:DBCSimple}
\end{figure}


\subsection{Exemplo de DBC com chamada a serviço}

A figura \ref{lst:DBCService} apresenta a capacidade \emph{excluirItem} (linha
8). Nesse caso são estabelecidas a pré e a pós condições, ambas com chamadas ao
serviço \emph{Catalogo.pesquisarItem}. Antes do processamento da capacidade
\emph{excluirItem}, é verificado se o item exite. Caso exista, a pré-condição é
satisfeita e o item é excluído. A pós-condição (linha 6) confirma que o item não
consta mais do catálogo. Se a pré-condição não for satisfeita, a cláusula
\emph{/@otherwise} (linha 7) informa ao cliente do serviço que o item não foi
localizado (código HTTP 404 - Objeto não encontrado).


\begin{figure}[htb]
\begin{small}
\lstinputlisting[language=NeoIDL,firstnumber=1]{DBCservice.neo}
\end{small}
\caption{Exemplo da notação DBC no \neoidl com chamada a serviço}
\label{lst:DBCService}
\end{figure}