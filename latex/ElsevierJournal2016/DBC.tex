
\subsection{Design-by-Contract}

\designbycontract{} \cite{meyer1992applying} - DbC - é um conceito
oriundo da orientação a objetos, no qual consumidor e fornecedor firmam entre si garantias para
o uso de métodos ou classes. De um lado o consumidor deve garantir que, antes da
chamada a um método, algumas condições sejam por ele satisfeitas.
Do outro lado o fornecedor deve garantir, se respeitadas suas exigências,
o sucesso da execução.

O mecanismo que expressa essas condições são chamados de asserções
(\textit{assertions}, em inglês). As asserções que o consumidor deve respeitar
para fazer uso da rotina são chamadas de \textbf{precondições}. As asserções que
asseguram, de parte do fornecedor, as garantias ao consumidor, são denominadas
\textbf{pós-condições}.

O conceito chave de \designbycontract{} é ver a relação entre a classe e
seus clientes como uma relação formal, que expressa os direitos e as
obrigações de cada parte \cite{meyer1997object}. Se, por um lado, o
cliente tem a obrigação de respeitar as condições impostas pelo fornecedor para fazer uso do módulo, por
outro, o fornecedor deve garantir que o retorno ocorra como esperado. De forma
indireta, \designbycontract{} estimula um cuidado maior na análise das condições necessárias para um funcionamento correto do recurso.

Com o uso de \designbycontract{}, cada rotina é levada a realizar o trabalho
para o qual foi projetada e fazer isso bem: com corretude, eficiência e
genericamente suficiente para ser reusada. Por outro lado, especifica de forma
clara o que a rotina não trata. Esse paradigma é coerente, pois para que a
rotina realize seu trabalho bem, é esperado que se estabeleça bem as
circunstâncias de execução.

Outra característica da aplicação de \designbycontract{} é que o recurso tem sua
lógica concentrada em efetivamente cumprir com sua função principal, deixando
para as precondições o encargo de validar as entradas de dados. Essa abordagem
é o oposto à ideia de programação defensiva, pois vai de encontro à realização
de checagens redundantes. Se os contratos são precisos e explícitos, não há
necessidade de testes redundantes \cite{meyer1992applying}.

Eiffel, JML, Spec\# e Code Contract são exemplos de linguagem e extensões de
linguagens que dão suporte a \designbycontract{}. A linguagem \textbf{Eiffel} foi desenvolvida 
em meados dos anos 80 por
Bertrand Meyer \cite{meyer1988eiffel} com o objetivo de criar ferramentas que
garantissem mais qualidade aos \textit{softwares}. A ênfase do projeto de
Eiffel foi promover reusabilidade, extensibilidade e compatibilidade.
\textbf{JML} -- \textit{Java Modeling Language} -- é uma extensão da linguagem
Java para suporte a especificação comportamental de interfaces, ou seja, controlar o
comportamento de classes em tempo de execução, com amplo suporte a \designbycontract{}.


\textbf{Spec\#} é uma extensão da linguagem C\#, à qual agrega o suporte para
distinguir referência de objetos nulos de referência a objetos possivelmente não
nulos, especificações de pré e pós-condições e um método para gerenciar
exceções entre outros recursos \cite{barnett2004spec}. \textbf{Code Contract} é
uma maneira mais moderna que Spec\# para utilização de construções de DbC no 
\textit{framework} .Net. \textit{Code Contract} inclui classes para registrar as
restrições (pré, pós-condições e invariantes) ao código C\#, um analisador
estático em tempo de compilação e execução e, ainda, ferramentas pra
geração de testes unitários para verificação de précondições 
\cite{codecontractSite}.