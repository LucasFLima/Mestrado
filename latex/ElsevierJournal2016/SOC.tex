\section{Fundamentação}
\label{Fundamentacao}

\subsection{Computação orientada a serviço}

A computação orientada a serviços -- SOC -- visa aumentar a flexibilidade da
arquitetura, a fim de possibilitar rápida reação a mudanças nos requisitos de
negócio.
Para obter esses benefícios, contudo, os serviços devem cumprir com determinados quesitos, que
incluem alta autonomia ou baixo acoplamento \cite{erl2008soa}. Assim, o paradigma de SOC
está voltado para o projeto de soluções preparadas para constantes mudanças,
substituindo-se continuamente pequenas peças -- os serviços -- por outras
atualizadas.

Serviços são pequenos \textit{softwares} que provêem funcionalidades específicas
para serem reutilizadas em várias aplicações. Cada serviço é uma entidade isolada com
dependências limitadas de outros recursos compartilhados
\cite{serrano2014service}. Assim, é formada uma abstração entre os fornecedores
e consumidores dos serviços, por meio de baixo acoplamento, e promovendo a
flexibilidade de mudanças de implementação sem impacto aos consumidores.

O objetivo da SOC está portanto em conceber um estilo de projeto, tecnologia e
processos que permita às empresas desenvolver, interconectar e manter suas
aplicações e serviços corporativos com eficiência e baixo custo. Embora esses
objetivos não sejam novos, SOC procura superar os resultados de esforços
prévios como programação modular, reuso de código e técnicas de desenvolvimento orientadas a
objetos \cite{papazoglou2007serviceApprTechRechIss}.
% 
% De modo diferente de arquiteturas convencionais, ditas monolíticas, em que os
% sistemas são concebidos agregando continuamente funcionalidades a um mesmo pacote de
% \textit{software}, a arquitetura orientada a serviço prega o projeto de pequenas
% aplicações distribuídas -- os serviços -- que podem ser consumidas tanto por
% usuários finais como por outros serviços \cite{papazoglou2007serviceApprTechRechIss}. 

Segundo T. Erl \cite{erl2009web}, a orientação a serviços é estruturada em oito
princípios fundamentais, que caracterizam o modelo: \textit{Contrato
padronizado}, \textit{Baixo acomplamento}, \textit{Abstração},
\textit{Descoberta do serviço}, \textit{Reusabilidade}, \textit{Composição},
\textit{Autonomia} e \textit{Ausência de estado}. Quanto à padronização do
contrato, a abordagem \CtFirst{} preocupa-se sobretudo com a clareza, completude e
estabilidade do contrato para os clientes dos serviços, sem expor ou se ater às
caracteríscas da implementação subjacente. As principais vantagens do \CtFirst{} 
estão no baixo acoplamento do contrato em
relação a sua implementação, na possibilidade de reuso de esquemas de dados (XML
ou JSON Schema), na simplificação do versionamento e na facilidade de manutenção
\cite{karthikeyancontract}.

% \subsection{Princípios SOA}
% 
% 
% O paradigma de orientação a serviço é estruturado em oito princípios
% fundamentais \cite{erl2009web}. São eles que
% caraterizam a abordagem SOA e a sua aplicação faz com que um serviço se diferencie de um componente ou de
% um módulo. Os contratos de serviços permeiam a maior parte destes princípios.
% 
% O princípio do \textbf{Contrato padronizado} estabelece que serviços
% dentro de um mesmo inventário estejam em conformidade com os mesmos padrões de
% contrato de serviço.
% Os contratos de serviços são elementos fundamentais na arquitetura orientada a
% serviço, pois é por meio deles que os serviços interagem uns com os outros e com
% potenciais consumidores. O padrão de projeto \CtFirst{} é uma
% consequência direta deste princípio \cite{erl2009web}.
% 
% Os contratos de serviço devem impor aos seus consumidores requisitos de baixo
% acoplamento. Os contrato também devem ser desacoplados de seu ambiente e da
% implementação subjacente.
% Essas relações são guiadas pelo princípio do \textbf{Baixo acomplamento}. A
% qualidade do projeto do contrato e do desenho arquitetural do serviço que
% permitirá produzir os benefícios derivados do baixo acoplamento.
% 
% % O projeto do serviço deve considerar como negativos os acoplamentos entre
% % o contrato e a funcionalidade suportada pelo serviços, entre o contrato a a sua
% % implementação subjacente, entre o contrato com a tecnologia e a lógica interna
% % adotada para o serviço. Esses acoplamentos devem ser evitados.
% % 
% % Há, porém, um acoplamento desejado, que é o que gera dependência
% % da lógica em relação ao contrato \cite{erl2009web}. Ou seja, idealmente a
% % implementação do serviço deve ser derivada do contrato, pondendo se ter inclusive a geração de código a
% % partir do contrato.
% 
% O princípio da \textbf{Abstração} visa garantir a exposição apenas de
% informações necessárias e essenciais no contrato de serviço. Ou seja, o contrato deve possui tão somente
% as informações necessárias ao consumo e compreensão do serviço. A
% \textbf{Descoberta do serviço}, outro princípio SOA, prega que os contratos de
% serviços contenham informações (metadados) que possibilitem a descoberta e
% interpretação do serviço.
% 
% O aspecto do reuso é expresso em alguns princípios SOA. O princípio da
% \textbf{Reusabilidade} estabelece que os serviços que contenham ou expressem
% lógica agnóstica podem ser disponibilizados com recursos reusáveis. Nessa mesma
% linha, o princípio da \textbf{Composição} orienta para que serviços sejam
% participantes efetivos de composições, independentemente do tamanho ou
% complexidade da composição.
% 
% Por fim, os princípios da \textbf{Autonomia} -- serviços têm controle sobre seu
% ambiente de execução -- e \textbf{Ausência de estado} -- retringindo o controle
% de estado apenas a quanfo for inevitavelmente necessário -- formam, junto com os
% demais, um conjunto de características que buscam fazer com que o projeto dos serviços 
% alcance os objetivos da abordagem SOA.
% 
% \subsection{Contract-First}
% 
% A abordagem \CtFirst{} preocupa-se sobretudo com a clareza, completude e
% estabilidade do contrato para os clientes dos serviços. Toda a
% estrutura da informação é definida sem se ater às restrições ou
% características das implementações subjacentes. Do mesmo modo, as
% capacidades são definidas para atenderem a funcionalidades a que se destinam,
% porém preocupando-se em se promover estabilidade e reuso.
% 
% As principais vantagens do \CtFirst{} estão no baixo acoplamento do contrato em
% relação a sua implementação, na possibilidade de reuso de esquemas de dados (XML
% ou JSON Schema), na simplificação do versionamento e na facilidade de manutenção
% \cite{karthikeyancontract}. A desvantagem está justamente na complexidade de
% escrita do contrato. Porém, várias ferramentas já foram e vem sendo
% desenvolvidas para facilitar essa tarefa.
% 
% A outra forma de se garantir que o contrato expressa o que serviço realiza é a
% abordagem \CdFirst{}, em que o contrato é normalmente gerado a partir de
% anotações no código fonte. Embora muitas vezes preferível pelo desenvolvedor, a desvantagem do uso
% \CdFirst{} está no elevado impacto que alterações na implementação causam ao
% contrato, fazendo com que os clientes dos serviços sejam também afetados.
% Reduz-se ainda a flexibilidade e extensibilidade, de modo que o reuso é
% prejudicado.

%\subsection{\wss{}}

\wss{} são formas típicas de construção de serviços sobre o protocolo HTTP.
Dois padrões de \wss{} são mais populares: SOAP -- \textit{Simple Object Access
Protocol}, protocolo padrão W3C para troca de mensagens em sistemas
distribuídos -- e REST -- \textit{Representational State Transfer}, projetado
por Roy Fielding, em sua tese de doutorado \cite{fielding2000architectural}, para descrever um modelo distribuído de sistemas hipermedia sobre o protocolo HTTP, com uma semântica
específica para cada operação.

Ao contrário de SOAP, REST não dispõe de um padrão para
especificação de contratos. Essa carência, que no início não era considerada um problema, foi se
tornando uma necessidade cada vez mais evidente a medida em que se amplia o
conjunto de \wss{} implantados. Atualmente, existem algumas linguagens com o
propósito de documentar o contrato REST, sendo \textit{Swagger} a mais
popular.

% 
% são
% aplicações modulares e autocontidas que podem ser publicadas, localizadas e acessadas pela \textit{Web} \cite{alonso2004web}. A diferença
% entre o \ws{} e a aplicação \textit{Web} propriamente dita é que o primeiro se
% preocupa apenas com o dado gravado ou fornecido, deixando para o cliente a atribuição de apresentar a
% informação \cite{serrano2014service}.
% 
% SOAP -- \textit{Simple Object Access Protocol} -- é um protocolo padrão W3C
% para troca de mensagens em sistemas distribuídos, normalmente sobre o
% protocolo HTTP e acessível pela Internet. As mensagens em SOAP são codificadas
% XML e seus contratos especificados no padrão WSDL -- \textit{Web Services
% Description Language} -- que define uma gramática XML para descrever os serviços
% como uma coleção de \textit{endpoints} capazes de atuar na troca de mensagens.
% As mensagens e operações são descritas abstratamente na primeira seção do
% documento. Uma segunda seção, dita concreta, estabelece o protocolo de rede e o
% formato das mensagens.
% 
% O estilo arquitetural REST foi projetado por Roy Fielding, em sua tese de
% doutorado \cite{fielding2000architectural}, para descrever um modelo 
% distribuído de sistemas hipermedia sobre o protocolo HTTP, com uma semântica
% específica para cada operação.
% Tanto a requisição como a resposta ocorrem por meio da transferência de
% representações de recursos \cite{mumbaikar2013web}, que podem ser de vários
% formatos, como XML e JSON \cite{serrano2014service}. 
% 
% O uso de REST tem se tornado popular por conta de sua flexibilidade e
% performance em comparação com SOAP, o qual precisa envelopar suas informações em
% um pacote XML \cite{mumbaikar2013web}, de armazenamento, transmissão e
% processamento onerosos. Ao contrário de SOAP, REST não dispõe de um padrão para
% especificação de contratos. Essa carência, que no início não era considerada um problema, foi se
% tornando uma necessidade cada vez mais evidente a medida em que se amplia o
% conjunto de \wss{} implantados. Atualmente, existem algumas linguagens com o
% propósito de documentar o contrato REST, sendo \textit{Swagger} a mais
% popular.



