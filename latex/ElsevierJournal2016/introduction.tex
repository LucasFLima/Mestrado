\section{Introdução}
\label{Introducao}

A computação orientada a serviços ( \emph{Service-oriented computing, SOC)} tem
se mostrado uma solução de \textit{design} de \textit{software} que favorece o
alinhamento às mudanças constantes e urgentes nas instituições
\cite{chen2008towards}. 

Os benefícios de SOC estão diretamente relacionados ao
baixo acoplamento dos serviços que compõem a solução, de forma que as partes
(nesse caso serviços) possam ser subs\-ti\-tu\-í\-das e evoluídas facilmente, ou
ainda rearranjadas em novas composições. 

Um dos principais benefícios do uso de SOC está na possibilidade de reuso de
seus componentes. Porém, reuso requer serviços bem construídos e precisos em
relação a sua especificação \cite{jazequel1997design}. A qualidade e precisão do
contrato de serviço torna-se claramente um elemento fudamental para se auferir
os benefícios da abordagem SOC.

A qualidade da especificação do contrato de serviço é fundamental para o
projeto de software baseado em SOC. Este artigo aborda um aspecto
importante para a melhoria da robustez de contratos de serviços: a construção de
garantias mútuas por meio da especificação formal de contratos, agregando o
conceito de \designbycontract{}.

A seção XXX


% 
% Extender a sintaxe da \neoidl{} para admitir construções de 
% \designbycontract, com pré e pós-condições para operações de serviços REST
% (Seção \ref{extensaoNeoIDL-DbC});
% Incorporar à infraestrutura de \textit{Plugins} da
% \neoidl{} a capacidade de geração de código para o \framework{} \textit{Python
% Twisted} (Seção \ref{pluginTwisted});
% Implementar regras de transformação que traduzem construções de DbC
% \neoidl{} em código de validação para o framework \textit{Python
% Twisted} (Seção \ref{pluginTwisted});% com suporte a \designbycontract{} a
%  % partir de contratos especificados em \neoidl;
% Coletar a percepção de desenvolvedores sobre a aceitação da
% especificação de contratos REST com \designbycontract{} na \neoidl{} (Seção
% \ref{analiseSubjetiva}).
% 
% 
% Por outro lado, a \neoidl{}, da mesma forma que WSDL,
% Swagger e RAML não oferece construções para especificação de contratos formais
% com aspecto comportamental como os presentes em linguagens que
% suportam DbC (\emph{Design by Contract})~\cite{meyer1992applying}, como
% Eiffel, JML e Spec\#. Em outras palavras, a \neoidl{}  admite apenas
% contratos fracos (\textit{weak contracts}), sem suporte a construções com pré e
% pós-condições.
% 
% 
% As linguagens de especificação de contratos para SOC apresentam
% algumas limitações. Por exemplo, a linguagem WSDL (\emph{Web-services
% description language}) \cite{WSDLSite} é considerada uma solução
% pouco expressiva que desestimula a abordagem \textit{Contract First}. Por essa
% razão, especificações WSDL são usualmente derivadas a partir de anotações em código
% fonte (\textit{Code First}).
% 
% Com o objetivo de mitigar esses problemas, a linguagem \neoidl{} foi proposta
% para simplificar a especificação de serviços REST com mecanismos de modularização,
% suporte a anotações, herança em tipos de dados definidos pelo desenvolvedor, e
% uma sintaxe simples e concisa semelhante às IDLs presentes em \emph{Apache
% Thrift}\texttrademark\cite{thrift} e \emph{CORBA}\texttrademark\cite{corba}.

