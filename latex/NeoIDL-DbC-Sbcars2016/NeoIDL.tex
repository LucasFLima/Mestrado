\section{\neoidl{}: DSL para constratos REST}

\subsection{Apresentação}

A \neoidl{} é uma linguagem específica de domínio (\textit{Domain Specific
Language - DSL}) elaborada com o objetivo de possibilitar, em um processo
simples, a elaboração de contratos para serviços REST. Em seu projeto, 
foram considerados os requisitos de concisão, facilidade de compreensão humana,
extensibilidade e suporte à herança simples dos tipos de dados definidos pelo
usuário.

Além de ser uma linguagem, a \neoidl{} é também um \textit{framework} de geração
de código que permite, a partir de contratos especificados na própria
linguagem, a produção da implementação da estrutura do serviço. Os serviços
podem ser construídos em várias linguagens e tecnologias, por meio
de \textit{plugins} da \neoidl{}.

as linguagens de programação disponíveis para especificação de
contratos REST, como Swagger\cite{swaggerSite}, WADL\cite{hadley2006web} e
RAML\cite{RAML}, possuiam (e ainda possuem) limitações importantes para a
abordagem desejada, qual seja elaborar primeiramente o contrato e, a
partir dele, gerar a implementação do serviço. Todas essas linguagens utilizam
notações de propósito geral (XML\cite{XML}, JSON\cite{JSon}, YAML\cite{YAML}),
tornando os contratos extensos e de difícil compreensão humana. Além disso,
elas não possuem mecanismos semânticos de extensibilidade e modularidade.

Partiu-se então para a criação uma nova linguagem, com sintaxe
inspirada em linguagens mais claras e concisas -- como \emph{CORBA
IDL}\texttrademark \cite{corba} e \emph{Apache
Thrift}\texttrademark\cite{thrift} --, e que permitisse ainda a declaração de
tipos de dados definidos pelo usuário e extensibilidade. Ambas, \emph{CORBA} e
\emph{Apache Thrift}, possuem limitações nesses últimos aspectos. A sintaxe e as
características da linguagem \neoidl{} são brevemente discutidas na Subseção
\ref{linguagemNeoIDL}.

Em relação à geração de código, para se viabilizar a geração poliglota de
código, a \neoidl{} foi projetada para possuir uma arquitetura modular, de modo
que novas linguagens ou características de implementação pudessem ser
incorporadas por meio de \textit{plugins} da \neoidl{}. Assim, é possível desenvolver um novo \textit{plugin} para geração de serviços em outras
linguagens, por exemplo PHP, sem alterar qualquer outro componente do \framework,
conforme apresentado na Subseção \ref{frameNeoIDL}.

\subsection{Estudo de expressividade e reuso}

A primeira versão da \neoidl{} foi submetida a um estudo empírico de sua expressividade 
e reuso em um contexto real. As próximas subseções apresentam 
o resultado da análise comparativa da representação de 44 (quarenta e quatro)
contratos escritos em Swagger v1.2 em relação à mesma especificação em
\neoidl{}.

Esse estudo é uma das contribuições deste trabalho de mestrado e foi publicado
no periódico IJSeke \cite{lima2015neoidl}.

Com esse propósito, foi obtido um conjunto de 44 contratos do Exército
Brasileiro especificados em Swagger. A primeira etapa foi reescrever esses
contratos em \neoidl{} e então comparar a quantidade de linhas de código
produzida com a quantidade de linhas de código dos contratos originais.

Assim, para
cada 10 linhas de especificação em Swagger são requeridas 4 linhas de
especificação \neoidl{}. Nessa análise foram consideradas linhas físicas de
código, ignorando-se linhas em branco ou compostas apenas de delimitadores.

A proporção de redução não se deu de forma igual em todos os contratos. Por
exemplo, o contrato de um determinado serviço\footnote{Os nomes reais dos
contratos foram omitidos em razão de acordo de confidencialidade.} requereu 367
linhas na especificação Swagger e 112 linhas na especificação \neoidl{} --
redução da ordem de 69\%.

No conjunto dos 44 contratos Swagger analisados foram identificadas 42 entidades
especificadas em pelo menos dois contratos. Uma entidade específica, de
identificação da posição geográfica, muito utilizada no domínio de Comando e Controle, aparece
declarada 12 vezes em contratos distintos. A Tabela \ref{estatisticaEntidades}
mostra a estatística de repetição da especificação de contratos.

Verificou-se,
ainda, que a \neoidl{} é mais concisa e expressiva que Swagger para representar
as mesmas informações de contratos REST. Sob a ótica do reuso, a abordagem proposta pela
\neoidl{}, baseada em importações de especificações de entidades, tende a
contribuir para a identificação de entidades já declaradas, facilitando que
elas sejam reusadas.

