\section{\neoidl{}: DSL para constratos REST}


A \neoidl{} é uma linguagem específica de domínio (\textit{Domain Specific
Language - DSL}) elaborada com o objetivo de possibilitar, em um processo
simples, a elaboração de contratos para serviços REST. Em seu projeto, 
foram considerados os requisitos de concisão, facilidade de compreensão humana,
extensibilidade e suporte à herança simples dos tipos de dados definidos pelo
usuário.

Além de ser uma linguagem, a \neoidl{} é também um \textit{framework} de geração
de código que permite, a partir de contratos especificados na própria
linguagem, a produção da implementação da estrutura do serviço. Os serviços
podem ser construídos em várias linguagens e tecnologias, por meio
de \textit{plugins} da \neoidl{}.

As linguagens de programação disponíveis para especificação de
contratos REST, como Swagger\cite{swaggerSite}, WADL\cite{hadley2006web} e
RAML\cite{RAML}, possuiam (e ainda possuem) limitações importantes para a
abordagem desejada, qual seja, elaborar primeiramente o contrato e, a
partir dele, gerar a implementação do serviço. Todas essas linguagens utilizam
notações de propósito geral (XML\cite{XML}, JSON\cite{JSon}, YAML\cite{YAML}),
tornando os contratos extensos e de difícil compreensão humana. Além disso,
elas não possuem mecanismos semânticos de extensibilidade e modularidade.

Partiu-se então para a criação uma nova linguagem, com sintaxe
inspirada em linguagens mais claras e concisas -- como \emph{CORBA
IDL}\texttrademark \cite{corba} e \emph{Apache
Thrift}\texttrademark\cite{thrift} --, e que permitisse ainda a declaração de
tipos de dados definidos pelo usuário e extensibilidade. Ambas, \emph{CORBA} e
\emph{Apache Thrift}, possuem contudo limitações nesses últimos aspectos. 

A \neoidl{} suporta a geração de código poliglota, por meio da implementação de
plugins \neoidl{} para cada linguagem alvo. Atualmente, há plugins para Java,
Python \neocortex{}\footnote{Framework proprietário do Exército Brasileiro},
e Python Twisted. Detalhes sobre o
framework de geração de código da \neoidl{} foram publicados em
\cite{lima2015neoidl}. 

O artigo \cite{lima2015neoidl} também descreve uma estudo empírico sobre a
expressividade e potencial de reuso da \neoidl{}. A base desse estudo foi a
análise de contratos reais escritos em Swagger em comparação com os equivalentes
em \neoidl{}. Em resumo, nesse cenário concreto, identificou-se serem
necessárias 4 linhas de especificação em \neoidl{} para cada conjunto de 10
linhas em Swagger.

