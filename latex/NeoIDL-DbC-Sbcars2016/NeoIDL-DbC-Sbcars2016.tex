
%% bare_conf.tex
%% V1.4b
%% 2015/08/26
%% by Michael Shell
%% See:
%% http://www.michaelshell.org/
%% for current contact information.
%%
%% This is a skeleton file demonstrating the use of IEEEtran.cls
%% (requires IEEEtran.cls version 1.8b or later) with an IEEE
%% conference paper.
%%
%% Support sites:
%% http://www.michaelshell.org/tex/ieeetran/
%% http://www.ctan.org/pkg/ieeetran
%% and
%% http://www.ieee.org/

%%*************************************************************************
%% Legal Notice:
%% This code is offered as-is without any warranty either expressed or
%% implied; without even the implied warranty of MERCHANTABILITY or
%% FITNESS FOR A PARTICULAR PURPOSE! 
%% User assumes all risk.
%% In no event shall the IEEE or any contributor to this code be liable for
%% any damages or losses, including, but not limited to, incidental,
%% consequential, or any other damages, resulting from the use or misuse
%% of any information contained here.
%%
%% All comments are the opinions of their respective authors and are not
%% necessarily endorsed by the IEEE.
%%
%% This work is distributed under the LaTeX Project Public License (LPPL)
%% ( http://www.latex-project.org/ ) version 1.3, and may be freely used,
%% distributed and modified. A copy of the LPPL, version 1.3, is included
%% in the base LaTeX documentation of all distributions of LaTeX released
%% 2003/12/01 or later.
%% Retain all contribution notices and credits.
%% ** Modified files should be clearly indicated as such, including  **
%% ** renaming them and changing author support contact information. **
%%*************************************************************************


% *** Authors should verify (and, if needed, correct) their LaTeX system  ***
% *** with the testflow diagnostic prior to trusting their LaTeX platform ***
% *** with production work. The IEEE's font choices and paper sizes can   ***
% *** trigger bugs that do not appear when using other class files.       ***                          ***
% The testflow support page is at:
% http://www.michaelshell.org/tex/testflow/



\documentclass[conference]{IEEEtran}
% Some Computer Society conferences also require the compsoc mode option,
% but others use the standard conference format.
%
% If IEEEtran.cls has not been installed into the LaTeX system files,
% manually specify the path to it like:
% \documentclass[conference]{../sty/IEEEtran}





% Some very useful LaTeX packages include:
% (uncomment the ones you want to load)


% *** MISC UTILITY PACKAGES ***
%
%\usepackage{ifpdf}
% Heiko Oberdiek's ifpdf.sty is very useful if you need conditional
% compilation based on whether the output is pdf or dvi.
% usage:
% \ifpdf
%   % pdf code
% \else
%   % dvi code
% \fi
% The latest version of ifpdf.sty can be obtained from:
% http://www.ctan.org/pkg/ifpdf
% Also, note that IEEEtran.cls V1.7 and later provides a builtin
% \ifCLASSINFOpdf conditional that works the same way.
% When switching from latex to pdflatex and vice-versa, the compiler may
% have to be run twice to clear warning/error messages.






% *** CITATION PACKAGES ***
%
%\usepackage{cite}
% cite.sty was written by Donald Arseneau
% V1.6 and later of IEEEtran pre-defines the format of the cite.sty package
% \cite{} output to follow that of the IEEE. Loading the cite package will
% result in citation numbers being automatically sorted and properly
% "compressed/ranged". e.g., [1], [9], [2], [7], [5], [6] without using
% cite.sty will become [1], [2], [5]--[7], [9] using cite.sty. cite.sty's
% \cite will automatically add leading space, if needed. Use cite.sty's
% noadjust option (cite.sty V3.8 and later) if you want to turn this off
% such as if a citation ever needs to be enclosed in parenthesis.
% cite.sty is already installed on most LaTeX systems. Be sure and use
% version 5.0 (2009-03-20) and later if using hyperref.sty.
% The latest version can be obtained at:
% http://www.ctan.org/pkg/cite
% The documentation is contained in the cite.sty file itself.






% *** GRAPHICS RELATED PACKAGES ***
%
\ifCLASSINFOpdf
  % \usepackage[pdftex]{graphicx}
  % declare the path(s) where your graphic files are
  % \graphicspath{{../pdf/}{../jpeg/}}
  % and their extensions so you won't have to specify these with
  % every instance of \includegraphics
  % \DeclareGraphicsExtensions{.pdf,.jpeg,.png}
\else
  % or other class option (dvipsone, dvipdf, if not using dvips). graphicx
  % will default to the driver specified in the system graphics.cfg if no
  % driver is specified.
  % \usepackage[dvips]{graphicx}
  % declare the path(s) where your graphic files are
  % \graphicspath{{../eps/}}
  % and their extensions so you won't have to specify these with
  % every instance of \includegraphics
  % \DeclareGraphicsExtensions{.eps}
\fi
% graphicx was written by David Carlisle and Sebastian Rahtz. It is
% required if you want graphics, photos, etc. graphicx.sty is already
% installed on most LaTeX systems. The latest version and documentation
% can be obtained at: 
% http://www.ctan.org/pkg/graphicx
% Another good source of documentation is "Using Imported Graphics in
% LaTeX2e" by Keith Reckdahl which can be found at:
% http://www.ctan.org/pkg/epslatex
%
% latex, and pdflatex in dvi mode, support graphics in encapsulated
% postscript (.eps) format. pdflatex in pdf mode supports graphics
% in .pdf, .jpeg, .png and .mps (metapost) formats. Users should ensure
% that all non-photo figures use a vector format (.eps, .pdf, .mps) and
% not a bitmapped formats (.jpeg, .png). The IEEE frowns on bitmapped formats
% which can result in "jaggedy"/blurry rendering of lines and letters as
% well as large increases in file sizes.
%
% You can find documentation about the pdfTeX application at:
% http://www.tug.org/applications/pdftex





% *** MATH PACKAGES ***
%
%\usepackage{amsmath}
% A popular package from the American Mathematical Society that provides
% many useful and powerful commands for dealing with mathematics.
%
% Note that the amsmath package sets \interdisplaylinepenalty to 10000
% thus preventing page breaks from occurring within multiline equations. Use:
%\interdisplaylinepenalty=2500
% after loading amsmath to restore such page breaks as IEEEtran.cls normally
% does. amsmath.sty is already installed on most LaTeX systems. The latest
% version and documentation can be obtained at:
% http://www.ctan.org/pkg/amsmath





% *** SPECIALIZED LIST PACKAGES ***
%
%\usepackage{algorithmic}
% algorithmic.sty was written by Peter Williams and Rogerio Brito.
% This package provides an algorithmic environment fo describing algorithms.
% You can use the algorithmic environment in-text or within a figure
% environment to provide for a floating algorithm. Do NOT use the algorithm
% floating environment provided by algorithm.sty (by the same authors) or
% algorithm2e.sty (by Christophe Fiorio) as the IEEE does not use dedicated
% algorithm float types and packages that provide these will not provide
% correct IEEE style captions. The latest version and documentation of
% algorithmic.sty can be obtained at:
% http://www.ctan.org/pkg/algorithms
% Also of interest may be the (relatively newer and more customizable)
% algorithmicx.sty package by Szasz Janos:
% http://www.ctan.org/pkg/algorithmicx




% *** ALIGNMENT PACKAGES ***
%
%\usepackage{array}
% Frank Mittelbach's and David Carlisle's array.sty patches and improves
% the standard LaTeX2e array and tabular environments to provide better
% appearance and additional user controls. As the default LaTeX2e table
% generation code is lacking to the point of almost being broken with
% respect to the quality of the end results, all users are strongly
% advised to use an enhanced (at the very least that provided by array.sty)
% set of table tools. array.sty is already installed on most systems. The
% latest version and documentation can be obtained at:
% http://www.ctan.org/pkg/array


% IEEEtran contains the IEEEeqnarray family of commands that can be used to
% generate multiline equations as well as matrices, tables, etc., of high
% quality.




% *** SUBFIGURE PACKAGES ***
%\ifCLASSOPTIONcompsoc
%  \usepackage[caption=false,font=normalsize,labelfont=sf,textfont=sf]{subfig}
%\else
%  \usepackage[caption=false,font=footnotesize]{subfig}
%\fi
% subfig.sty, written by Steven Douglas Cochran, is the modern replacement
% for subfigure.sty, the latter of which is no longer maintained and is
% incompatible with some LaTeX packages including fixltx2e. However,
% subfig.sty requires and automatically loads Axel Sommerfeldt's caption.sty
% which will override IEEEtran.cls' handling of captions and this will result
% in non-IEEE style figure/table captions. To prevent this problem, be sure
% and invoke subfig.sty's "caption=false" package option (available since
% subfig.sty version 1.3, 2005/06/28) as this is will preserve IEEEtran.cls
% handling of captions.
% Note that the Computer Society format requires a larger sans serif font
% than the serif footnote size font used in traditional IEEE formatting
% and thus the need to invoke different subfig.sty package options depending
% on whether compsoc mode has been enabled.
%
% The latest version and documentation of subfig.sty can be obtained at:
% http://www.ctan.org/pkg/subfig




% *** FLOAT PACKAGES ***
%
%\usepackage{fixltx2e}
% fixltx2e, the successor to the earlier fix2col.sty, was written by
% Frank Mittelbach and David Carlisle. This package corrects a few problems
% in the LaTeX2e kernel, the most notable of which is that in current
% LaTeX2e releases, the ordering of single and double column floats is not
% guaranteed to be preserved. Thus, an unpatched LaTeX2e can allow a
% single column figure to be placed prior to an earlier double column
% figure.
% Be aware that LaTeX2e kernels dated 2015 and later have fixltx2e.sty's
% corrections already built into the system in which case a warning will
% be issued if an attempt is made to load fixltx2e.sty as it is no longer
% needed.
% The latest version and documentation can be found at:
% http://www.ctan.org/pkg/fixltx2e


%\usepackage{stfloats}
% stfloats.sty was written by Sigitas Tolusis. This package gives LaTeX2e
% the ability to do double column floats at the bottom of the page as well
% as the top. (e.g., "\begin{figure*}[!b]" is not normally possible in
% LaTeX2e). It also provides a command:
%\fnbelowfloat
% to enable the placement of footnotes below bottom floats (the standard
% LaTeX2e kernel puts them above bottom floats). This is an invasive package
% which rewrites many portions of the LaTeX2e float routines. It may not work
% with other packages that modify the LaTeX2e float routines. The latest
% version and documentation can be obtained at:
% http://www.ctan.org/pkg/stfloats
% Do not use the stfloats baselinefloat ability as the IEEE does not allow
% \baselineskip to stretch. Authors submitting work to the IEEE should note
% that the IEEE rarely uses double column equations and that authors should try
% to avoid such use. Do not be tempted to use the cuted.sty or midfloat.sty
% packages (also by Sigitas Tolusis) as the IEEE does not format its papers in
% such ways.
% Do not attempt to use stfloats with fixltx2e as they are incompatible.
% Instead, use Morten Hogholm'a dblfloatfix which combines the features
% of both fixltx2e and stfloats:
%
% \usepackage{dblfloatfix}
% The latest version can be found at:
% http://www.ctan.org/pkg/dblfloatfix




% *** PDF, URL AND HYPERLINK PACKAGES ***
%
%\usepackage{url}
% url.sty was written by Donald Arseneau. It provides better support for
% handling and breaking URLs. url.sty is already installed on most LaTeX
% systems. The latest version and documentation can be obtained at:
% http://www.ctan.org/pkg/url
% Basically, \url{my_url_here}.




% *** Do not adjust lengths that control margins, column widths, etc. ***
% *** Do not use packages that alter fonts (such as pslatex).         ***
% There should be no need to do such things with IEEEtran.cls V1.6 and later.
% (Unless specifically asked to do so by the journal or conference you plan
% to submit to, of course. )

\newcommand{\neoidl}{NeoIDL}
\newcommand{\neocortex}{NeoCortex}
\newcommand{\bnfc}{\texttt{BNFConverter}}
\newcommand{\designbycontract}{\textit{Design-by-Contract}}
\newcommand{\twisted}{\textit{Twisted}}
\newcommand{\ws}{\textit{Web Service}}
\newcommand{\wss}{\textit{Web Services}}
\newcommand{\CdFirst}{\textit{Code-first}}
\newcommand{\CtFirst}{\textit{Contract-first}}
\newcommand{\framework}{\textit{framework}}
\newcommand{\method}[1]{\texttt{#1}}


% correct bad hyphenation here
\hyphenation{op-tical net-works semi-conduc-tor}


\begin{document}
%
% paper title
% Titles are generally capitalized except for words such as a, an, and, as,
% at, but, by, for, in, nor, of, on, or, the, to and up, which are usually
% not capitalized unless they are the first or last word of the title.
% Linebreaks \\ can be used within to get better formatting as desired.
% Do not put math or special symbols in the title.
\title{Lightweight or strong REST contracts?\\The both with \neoidl{} and
\designbycontract{}}


% author names and affiliations
% use a multiple column layout for up to three different
% affiliations
\author{\IEEEauthorblockN{Lucas F. Lima}
\IEEEauthorblockA{Electrical Engineering Department\\
University of Brasilia - UnB\\
Brasilia, Brazil, Campus Darci Ribeiro\\
Email: unb.lucaslima@gmail.com}
\and
\IEEEauthorblockN{Rodrigo Bonifácio}
\IEEEauthorblockA{Computer Science Department\\
University of Brasilia - UnB\\
Brasilia, Brazil, Campus Darci Ribeiro\\
Email: ---}
\and
\IEEEauthorblockN{Edna Dias Canedo}
\IEEEauthorblockA{Faculty of Gama\\
University of Brasilia - UnB\\
Campus of Gama\\
Email: ---}
}
% conference papers do not typically use \thanks and this command
% is locked out in conference mode. If really needed, such as for
% the acknowledgment of grants, issue a \IEEEoverridecommandlockouts
% after \documentclass

% for over three affiliations, or if they all won't fit within the width
% of the page, use this alternative format:
% 
%\author{\IEEEauthorblockN{Michael Shell\IEEEauthorrefmark{1},
%Homer Simpson\IEEEauthorrefmark{2},
%James Kirk\IEEEauthorrefmark{3}, 
%Montgomery Scott\IEEEauthorrefmark{3} and
%Eldon Tyrell\IEEEauthorrefmark{4}}
%\IEEEauthorblockA{\IEEEauthorrefmark{1}School of Electrical and Computer Engineering\\
%Georgia Institute of Technology,
%Atlanta, Georgia 30332--0250\\ Email: see http://www.michaelshell.org/contact.html}
%\IEEEauthorblockA{\IEEEauthorrefmark{2}Twentieth Century Fox, Springfield, USA\\
%Email: homer@thesimpsons.com}
%\IEEEauthorblockA{\IEEEauthorrefmark{3}Starfleet Academy, San Francisco, California 96678-2391\\
%Telephone: (800) 555--1212, Fax: (888) 555--1212}
%\IEEEauthorblockA{\IEEEauthorrefmark{4}Tyrell Inc., 123 Replicant Street, Los Angeles, California 90210--4321}}




% use for special paper notices
%\IEEEspecialpapernotice{(Invited Paper)}




% make the title area
\maketitle

% As a general rule, do not put math, special symbols or citations
% in the abstract
\begin{abstract}
\noindent
{\large {\bf ABSTRACT}}

\vspace{5mm}
\noindent {\bf CONTRATOS REST ROBUSTOS E LEVES: UMA ABORDAGEM EM
DESIGN-BY-CONTRACT COM NEOIDL } 
 
\vspace{5mm} 

\noindent  {\bf
Autor: Lucas Ferreira de Lima}

\noindent {\bf Orientador: Rodigo Bonifácio de Almeida}

\noindent {\bf Programa de Pós-Graduação em Engenharia Elétrica}

\noindent {\bf Brasília, julho de 2016 }

\vspace{5mm}
\noindent
..

\vspace{5mm}
\noindent
..

\vspace{5mm}
\noindent
..

\vspace{5mm}
\noindent
..

\clearpage

\end{abstract}

% no keywords




% For peer review papers, you can put extra information on the cover
% page as needed:
% \ifCLASSOPTIONpeerreview
% \begin{center} \bfseries EDICS Category: 3-BBND \end{center}
% \fi
%
% For peerreview papers, this IEEEtran command inserts a page break and
% creates the second title. It will be ignored for other modes.
\IEEEpeerreviewmaketitle


\section{Introduction}
% no \IEEEPARstart
This demo file is intended to serve as a ``starter file''
for IEEE conference papers produced under \LaTeX\ using
IEEEtran.cls version 1.8b and later.
% You must have at least 2 lines in the paragraph with the drop letter
% (should never be an issue)
I wish you the best of success.

\hfill mds
 
\hfill August 26, 2015

\subsection{Subsection Heading Here}
Subsection text here. \cite{lima2015contratos}


\subsubsection{Subsubsection Heading Here}
Subsubsection text here.

\section{Fundamentação}
\label{Fundamentacao}

\subsection{Computação orientada a serviço}

A eficiência na integração entre as soluções de TI é um fator determinante para
que se consiga alterar uma parte sem comprometer todo um ecossistema. A
integração possibilita a combinação de eficiência e flexibilidade de recursos para otimizar
a operação através e além dos limites de uma organização, proporcionando maior
interoperabilidade \cite{papazoglou2008service}.

A computação orientada a serviços -- SOC -- endereça essas necessidades em uma
plataforma que aumenta a flexibilidade, a
fim de reagir rapidamente a mudanças nos requisitos de negócio. Para obter esses
benefícios, contudo, os serviços devem cumprir com determinados quesitos, que
incluem alta autonomia ou baixo acoplamento \cite{erl2008soa}. Assim, o paradigma de SOC
está voltado para o projeto de soluções preparadas para constantes mudanças,
substituindo-se continuamente pequenas peças -- os serviços -- por outras
atualizadas.

Portando, o objetivo da SOC está em conceber um estilo de projeto, tecnologia e
processos que permitam às empresas desenvolver, interconectar e manter suas
aplicações e serviços corporativos com eficiência e baixo custo. Embora esses
objetivos não sejam novos, SOC procura superar os esforços prévios como
programação modular, reuso de código e técnicas de desenvolvimento orientadas a
objetos \cite{papazoglou2007serviceApprTechRechIss}.

De modo diferente de arquiteturas convencionais, ditas monolíticas, em que os
sistemas são concebidos agregando continuamente funcionalidades a um mesmo pacote de
\textit{software}, a arquitetura orientada a serviço prega o projeto de pequenas
aplicações distribuídas -- os serviços -- que podem ser consumidas tanto por
usuários finais como por outros serviços \cite{papazoglou2007serviceApprTechRechIss}. 

Serviços são pequenos \textit{softwares} que provêem funcionalidades específicas
para serem reutilizadas em várias aplicações. Cada serviço é uma entidade isolada com
dependências limitadas de outros recursos compartilhados
\cite{serrano2014service}. Assim, é formada uma abstração entre os fornecedores
e consumidores dos serviços, por meio de baixo acoplamento, e promovendo a
flexibilidade de mudanças de implementação sem impacto aos consumidores.

\subsection{Princípios SOA}


O paradigma de orientação a serviço é estruturado em oito princípios
fundamentais \cite{erl2009web}. São eles que
caraterizam a abordagem SOA e a sua aplicação faz com que um serviço se diferencie de um componente ou de
um módulo. Os contratos de serviços permeiam a maior parte destes princípios.

O princípio do \textbf{Contrato padronizado} estabelece que serviços
dentro de um mesmo inventário estejam em conformidade com os mesmos padrões de
contrato de serviço.
Os contratos de serviços são elementos fundamentais na arquitetura orientada a
serviço, pois é por meio deles que os serviços interagem uns com os outros e com
potenciais consumidores. O padrão de projeto \CtFirst{} é uma
consequência direta deste princípio \cite{erl2009web}.

Os contratos de serviço devem impor aos seus consumidores requisitos de baixo
acoplamento. Os contrato também devem ser desacoplados de seu ambiente. Essas
relações são guiadas pelo princípio do \textbf{Baixo acomplamento}. A forma como
o contrato é projetado e posicionado na arquitetura é que gerará o
benefício do baixo acoplamento.  

O projeto do serviço deve considerar como negativos os acoplamentos entre
o contrato e a funcionalidade suportada pelo serviços, entre o contrato a a sua
implementação subjacente, entre o contrato com a tecnologia e a lógica interna
adotada para o serviço. Esses acoplamentos devem ser evitados.

Há, porém, um acoplamento desejado, que é o que gera dependência
da lógica em relação ao contrato \cite{erl2009web}. Ou seja, idealmente a
implementação do serviço deve ser derivada do contrato, pondendo se ter inclusive a geração de código a
partir do contrato.

O princípio da \textbf{Abstração} visa garantir a exposição apenas de
informações necessárias e essenciais no contrato de serviço. Ou seja, o contrato deve possui tão somente
as informações necessárias ao consumo e compreensão do serviço. A
\textbf{Descoberta do serviço}, outro princípio SOA, prega que os contratos de
serviços contenham informações (metadados) que possibilitem a descoberta e
interpretação do serviço.

O aspecto do reuso é expresso em alguns princípios SOA. O princípio da
\textbf{Reusabilidade} estabelece que os serviços que contenham ou expressem
lógica agnóstica podem ser disponibilizaddos com recursos reusáveis. Nessa mesma
linha, o princípio da \textbf{Composição} orienta para que serviços sejam
participantes efetivos de composições, independentemente do tamanho ou
complexidade da composição.

Por fim, os princípios da \textbf{Autonomia} -- serviços têm controle sobre seu
ambiente de execlução -- e \textbf{Ausência de estado} -- retringindo o controle
de estado apenas a quanfo for inevitavelmente necessário -- formam, junto com os
demais, um conjunto de características que buscam fazer com que o projeto dos serviços 
alcance os objetivos da abordagem SOA.

\subsection{Contract-First}

A abordagem \CtFirst{} preocupa-se sobretudo com a clareza, completude e
estabilidade do contrato para os clientes dos serviços. Toda a
estrutura da informação é definida sem a preocupação sobre restrições ou
características das implementações subjacentes. Do mesmo modo, as
capacidades são definidas para atenderem a funcionalidades a que se destinam,
porém com a preocupação em se promover estabilidade e reuso.

As principais vantagens do \CtFirst{} estão no baixo acoplamento do contrato em
relação a sua implementação, na possibilidade de reuso de esquemas de dados (XML
ou JSON Schema), na simplificação do versionamento e na facilidade de manutenção
\cite{karthikeyancontract}. A desvantagem está justamente na complexidade de
escrita do contrato. Porém, várias ferramentas já foram e vem sendo
desenvolvidas para facilitar essa tarefa.

A outra forma de se garantir que o contrato expressa o que serviço realiza é a
abordagem \CdFirst{}, em que o contrato é normalmente gerado a partir de
anotações no código fonte. Embora muitas vezes preferível pelo desenvolvedor, a desvantagem do uso
\CdFirst{} está no elevado impacto que alterações na implementação causam ao
contrato, fazendo com que os clientes dos serviços sejam também afetados.
Reduz-se ainda a flexibilidade e extensibilidade, de modo que o reuso é
prejudicado.

\subsection{\wss{}}

\ws{} são aplicações modulares e autocontidas que podem ser publicadas,
localizadas a acessadas pela \textit{Web} \cite{alonso2004web}. A diferença
entre o \ws{} e a aplicação \textit{Web} propriamente dita é que o primeiro se
preocupa apenas com o dado gravado ou fornecido, deixando para o cliente a atribuição de apresentar a
informação \cite{serrano2014service}.

SOAP -- \textit{Simple Object Access Protocol} -- é um protocolo padrão W3C que
provê uma definição de como trocar informações estruturadas, por meio de XML, entre
as partes de ambientes descentralizados ou distribuídos \cite{WSDLSite}. SOAP
é um protocolo mais antigo que REST, e foi desenvolvido para troca de informações
pela Internet se utilizando de protocolos como HTTP, SMTP, FTP, sendo o primeiro
o mais comumente utilizado.

Os contratos em SOAP são especificados no padrão WSDL -- \textit{Web Services
Description Language} -- que define uma gramática XML para descrever os serviços
como uma coleção de \textit{endpoints} capazes de atuar na troca de mensagens.
As mensagens e operações são descritas abstratamente na primeira seção do
documento. Uma segunda seção, dita concreta, estabelece o protocolo de rede e o
formato das mensagens.

O termo REST foi criado por Roy Fielding, em sua tese de doutorado
\cite{fielding2000architectural}, para descrever um modelo arquitetural
distribuído de sistemas hipermedia. 

O estilo arquitetural REST é cliente-servidor, em que o cliente envia uma
requisição por um determinado recurso ao servidor e este retorna uma resposta.
Tanto a requisição como a resposta ocorrem por meio da transferência de
representações de recursos \cite{mumbaikar2013web}, que podem ser de vários
formatos, como XML e JSON \cite{serrano2014service}. Toda troca de informações
ocorre por meio do protocolo HTTP, com uma semântica específica para cada
operação

O uso de REST tem se tornado popular por conta de sua flexibilidade e
performance em comparação com SOAP, que precisa envelopar suas informações em
um pacote XML \cite{mumbaikar2013web}, de armazenamento, transmissão e
processamento onerosos.

Ao contrário de SOAP, REST não dispõe de um padrão para especificação de
contratos. Essa carência, que no início não era considerada um problema, foi se
tornando uma necessidade cada vez mais evidente a medida em que se amplia o
conjunto de \wss{} implantados. Atualmente, existem algumas linguagens com o
propósito de documentar o contrato REST.

A linguagem mais popular atualmente é \textit{Swagger} cujo projeto se iniciou
por volta de 2010 para atender a necessidade de um projeto específico, sendo posteriormente
vendida para uma grande empresa. Em janeiro de 2016, \textit{Swagger} foi doada
para o \textit{Open API Iniciative (OAI)} e denominada de \textit{Open API
Specification}. O propósito da iniciativa é tornar \textit{Swagger} padrão para
especificação de APIs com independencia de fornecedor.

\subsection{Design-by-Contract}

\designbycontract{} \cite{meyer1992applying} - DbC - é um conceito
oriundo da orientação a objetos, no qual consumidor e fornecedor firmam entre si garantias para
o uso de métodos ou classes. De um lado o consumidor deve garantir que, antes da
chamada a um método, algumas condições sejam por ele satisfeitas.
Do outro lado o fornecedor deve garantir, se respeitadas suas exigências,
o sucesso da execução.

O mecanismo que expressa essas condições são chamados de asserções
(\textit{assertions}, em inglês). As asserções que o consumidor deve respeitar
para fazer uso da rotina são chamadas de \textbf{precondições}. As asserções que
asseguram, de parte do fornecedor, as garantias ao consumidor, são denominadas
\textbf{pós-condições}.

O conceito chave de \designbycontract{} é ver a relação entre a classe e
seus clientes como uma relação formal, que expressa os direitos e as
obrigações de cada parte \cite{meyer1997object}. Se, por um lado, o
cliente tem a obrigação de respeitar as condições impostas pelo fornecedor para fazer uso do módulo, por
outro, o fornecedor deve garantir que o retorno ocorra como esperado.

De forma indireta, \designbycontract{} estimula um cuidado maior na análise das
condições necessárias para, de forma consistente, se ter o funcionamento correto
da relação de cooperação cliente-fornecedor.
Essas condições são expressas em cada contrato, o qual especifica as obrigações
a que cada parte está condicionada e, em contra-ponto, os benefícios garantidos.

Com o uso de \designbycontract{}, cada rotina é levada a realizar o trabalho
para o qual foi projetada e fazer isso bem: com corretude, eficiência e
genericamente suficiente para ser reusada. Por outro lado, especifica de forma
clara o que a rotina não trata. Esse paradigma é coerente, pois para que a
rotina realize seu trabalho bem, é esperado que se estabeleça bem as
circunstâncias de execução.

Outra característica da aplicação de \designbycontract{} é que o recurso tem sua
lógica concentrada em efetivamente cumprir com sua função principal, deixando
para as precondições o encargo de validar as entradas de dados. Essa abordagem
é o oposto à ideia de programação defensiva, pois vai de encontro à realização
de checagens redundantes. Se os contratos são precisos e explícitos, não há
necessidade de testes redundantes \cite{meyer1992applying}.

Implementações de \designbycontract{}. Eiffel - A linguagem Eiffel foi desenvolvida em meados dos anos 80 por
Bertrand Meyer \cite{meyer1988eiffel} com o objetivo de criar ferramentas que
garantissem mais qualidade aos \textit{softwares}. A ênfase do projeto de
Eiffel foi promover reusabilidade, extensibilidade e compatibilidade.
Características que só fazem sentido se os programas forem corretos e
robustos.

JML - \textit{Java Modeling Language} é uma extensão da linguagem Java
para suporte a especificação comportamental de interfaces, ou seja, controlar o
comportamento de classes em tempo de execução. Para realizar essa função, JML
possui amplo suporte a \designbycontract{}.

Spec\# - é uma extensão da linguagem C\#, à qual agrega o suporte para
distinguir referência de objetos nulos de referência a objetos possivelmente não
nulos, especificações de pré e pós-condições e um método para gerenciar
exceções entre outros recursos \cite{barnett2004spec}.


% An example of a floating figure using the graphicx package.
% Note that \label must occur AFTER (or within) \caption.
% For figures, \caption should occur after the \includegraphics.
% Note that IEEEtran v1.7 and later has special internal code that
% is designed to preserve the operation of \label within \caption
% even when the captionsoff option is in effect. However, because
% of issues like this, it may be the safest practice to put all your
% \label just after \caption rather than within \caption{}.
%
% Reminder: the "draftcls" or "draftclsnofoot", not "draft", class
% option should be used if it is desired that the figures are to be
% displayed while in draft mode.
%
%\begin{figure}[!t]
%\centering
%\includegraphics[width=2.5in]{myfigure}
% where an .eps filename suffix will be assumed under latex, 
% and a .pdf suffix will be assumed for pdflatex; or what has been declared
% via \DeclareGraphicsExtensions.
%\caption{Simulation results for the network.}
%\label{fig_sim}
%\end{figure}

% Note that the IEEE typically puts floats only at the top, even when this
% results in a large percentage of a column being occupied by floats.


% An example of a double column floating figure using two subfigures.
% (The subfig.sty package must be loaded for this to work.)
% The subfigure \label commands are set within each subfloat command,
% and the \label for the overall figure must come after \caption.
% \hfil is used as a separator to get equal spacing.
% Watch out that the combined width of all the subfigures on a 
% line do not exceed the text width or a line break will occur.
%
%\begin{figure*}[!t]
%\centering
%\subfloat[Case I]{\includegraphics[width=2.5in]{box}%
%\label{fig_first_case}}
%\hfil
%\subfloat[Case II]{\includegraphics[width=2.5in]{box}%
%\label{fig_second_case}}
%\caption{Simulation results for the network.}
%\label{fig_sim}
%\end{figure*}
%
% Note that often IEEE papers with subfigures do not employ subfigure
% captions (using the optional argument to \subfloat[]), but instead will
% reference/describe all of them (a), (b), etc., within the main caption.
% Be aware that for subfig.sty to generate the (a), (b), etc., subfigure
% labels, the optional argument to \subfloat must be present. If a
% subcaption is not desired, just leave its contents blank,
% e.g., \subfloat[].


% An example of a floating table. Note that, for IEEE style tables, the
% \caption command should come BEFORE the table and, given that table
% captions serve much like titles, are usually capitalized except for words
% such as a, an, and, as, at, but, by, for, in, nor, of, on, or, the, to
% and up, which are usually not capitalized unless they are the first or
% last word of the caption. Table text will default to \footnotesize as
% the IEEE normally uses this smaller font for tables.
% The \label must come after \caption as always.
%
%\begin{table}[!t]
%% increase table row spacing, adjust to taste
%\renewcommand{\arraystretch}{1.3}
% if using array.sty, it might be a good idea to tweak the value of
% \extrarowheight as needed to properly center the text within the cells
%\caption{An Example of a Table}
%\label{table_example}
%\centering
%% Some packages, such as MDW tools, offer better commands for making tables
%% than the plain LaTeX2e tabular which is used here.
%\begin{tabular}{|c||c|}
%\hline
%One & Two\\
%\hline
%Three & Four\\
%\hline
%\end{tabular}
%\end{table}


% Note that the IEEE does not put floats in the very first column
% - or typically anywhere on the first page for that matter. Also,
% in-text middle ("here") positioning is typically not used, but it
% is allowed and encouraged for Computer Society conferences (but
% not Computer Society journals). Most IEEE journals/conferences use
% top floats exclusively. 
% Note that, LaTeX2e, unlike IEEE journals/conferences, places
% footnotes above bottom floats. This can be corrected via the
% \fnbelowfloat command of the stfloats package.




\section{Conclusão}

A necessidade de estratégias de integração de sistemas e soluções adaptáveis às
constantes necessidade des mudanças tem levado às empresas a, cada vez mais,
adotarem o modelo de computação orientada a serviços -- SOC
\cite{papazoglou2008service} \cite{erl2009web}. A qualidade da especificação do
serviço por meio de seu contrato é um dos fatores determinantes para o sucesso
do uso de SOC.

A \neoidl{} foi criada para ser uma alternativa às linguagens de especificação
de \wss REST, uma vez que estas possuem uma sintaxe pouco expressiva para humanos,
além de não disporem de mecanismos com suporte a extensibilidade e modularização. O estudo
empírico da comparação entre especificações Swagger e \neoidl{} demonstrou a capacidade
expressiva e potencial de reuso da \neoidl{}. 
Entretanto, nem a \neoidl{}, nem as demais linguagens possibilitavam especificar 
contratos robustos, como os existentes em linguagens com suporte a \designbycontract{}.

Esse trabalho apresentou uma extensão da \neoidl{} para possibilitar a especificação
de contratos REST com suporte a pré e pós-condições e, a partir do contrato, permitir
a geração de código de serviços REST com essas garantias, seguindo a abordagem
\CtFirst{}. A proposta se baseou no paradgima de orientação a objetos, uma das 
principais influências da orientação a serviços.

Essa proposta foi submetida ao \textit{feedback} de profissionais experientes e também ao 
\textit{Workshop} de teses de dissertações do WTDSoft 2015. Os elementos sintáticos
de linguagens com suporte a \designbycontract{} como Eiffel, JML e Spec\# foram avaliados
e proporcionaram a criação de novas construções sintáticas para a \neoidl{} mantendo a harmonia com 
a linguagem pre-existente, sem que se perdesse o potencial para criação de regras de validação flexíveis e abrangentes.

A \neoidl{} passou a permitir a validação dos parâmetros de entrada e saída de uma requisição (por meio
de pré e pós-condições básicas) assim como permitir também fazer requisição a outros serviços para realizar validações mais
complexas, por meio de pequenas composições de serviços. A construção de um \textit{Plugin Twisted} demonstrou ser
viável produzir código que realize, em tempo de execução, a validação das regras estabelecidas no contrato, sem que o 
desenvolvedor tenha que se preocupar com elas e direcione seu esforço para a implementação das regras
de negócio.

A avaliação subjetiva, realizada por meio de questionário baseado nas técnicas GQM e TAM, apresentou
resultados satisfatórios à hipótese de pesquisa, em que grande parte dos respondentes indicou que a \neoidl{} com suporte
a \designbycontract{} é uma ferramenta com potencial de adoção, demonstrando ser uma linguagem e um 
\framework{} úteis e fáceis de serem utilizados sob a perspectiva de quem escreve contratos e implementa
serviços. 

\subsection{TRABALHOS FUTUROS}


O estudo sobre especificação de contratos para serviços REST é um campo de
pesquisa aberto, sobretudo pelo fato de não haver um formato oficial,
estabelecido pelo W3C ou outra entidade de padrões internacionais como o IEEE.
Este trabalho explorou o aspecto do uso de construções de \designbycontract{}
em serviços REST que, embora tenha produzido resultados promissores, ainda não
fecha a questão por completo.

Ainda sobre a proposta apresentada nessa dissertação, a análise dos resultados
(Seção \ref{AnaliseQuestionario}) levanta a possibilidade de exploração de
algumas respostas, em especial dos fatores que levaram uma parte importante dos
respondentes a informar que tem dúvida (nem discordam nem concordam) com o
adoção da \neoidl{} em suas atividades. Um novo questionário poderia ser
elaborado sobre esse ponto.

A realização de um experimento que avalie o uso por completo da abordagem de
pré e pós-condições desde a especificação dos requisitos, a elaboração dos
contratos, até os testes dos serviços implementados em um cenário hipotético
poderia trazer mais insumos para a melhoria da \neoidl{}. Ainda mais relevante
seria a experimentação da \neoidl{} com \designbycontract{} em um cenário de serviços reais.

Sob a perspectiva de implementação, podem ser elaborados \textit{plugins} para
outras linguagens de programação além de \textit{Python Twisted} com suporte a
\designbycontract{}, de modo a ampliar o potencial de utilização do
\framework{}. O projeto da \neoidl{} também pode incorporar mecanismos que
facilitem a especificação de contratos, como recursos ambientes
de desenvolvimento (IDEs) e \textit{code complete}. Ainda um portal de onde
podessem ser gerados os códigos sem a necessidade de instalação local do
\framework{}.

Um outro ponto é exploração do potencial da \neoidl{} na disponibilização de
recursos de \textit{hypermedia} \cite{webber2010rest}, fornecendo mecanismos
em que o cliente possa interagir com o servidor para identificar os serviços
que deseja seguir. A tranformação bidirecional entre o código gerado e a
especificação do contrato, útil em cenários de manutenção, também pode ser
explorada.



% use section* for acknowledgment
\section*{Acknowledgment}


The authors would like to thank...



% conference papers do not normally have an appendix





% trigger a \newpage just before the given reference
% number - used to balance the columns on the last page
% adjust value as needed - may need to be readjusted if
% the document is modified later
%\IEEEtriggeratref{8}
% The "triggered" command can be changed if desired:
%\IEEEtriggercmd{\enlargethispage{-5in}}

% references section

% can use a bibliography generated by BibTeX as a .bbl file
% BibTeX documentation can be easily obtained at:
% http://mirror.ctan.org/biblio/bibtex/contrib/doc/
% The IEEEtran BibTeX style support page is at:
% http://www.michaelshell.org/tex/ieeetran/bibtex/
%\bibliographystyle{IEEEtran}
% argument is your BibTeX string definitions and bibliography database(s)
%\bibliography{IEEEabrv,../bib/paper}
%
% <OR> manually copy in the resultant .bbl file
% set second argument of \begin to the number of references
% (used to reserve space for the reference number labels box)
\begin{thebibliography}{1}

\bibliography{bibdata}
%\bibitem{IEEEhowto:kopka}
%H.~Kopka and P.~W. Daly, \emph{A Guide to \LaTeX}, 3rd~ed.\hskip 1em plus
%  0.5em minus 0.4em\relax Harlow, England: Addison-Wesley, 1999.

\end{thebibliography}




% that's all folks
\end{document}


