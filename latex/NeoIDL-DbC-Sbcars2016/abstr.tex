\emph{Context}. The demand for integratin heterogeneus systems grows up the
adoptions of solutions based on service oriented computing -- SOC, in special
with the increasing use of the REST architectural style. Nevertheless, there
is no standard way to represent REST contracts.
Swagger, YAML and WADL only provide mechanisms to describe services, which
leads to a relevant limitation: they are made for computers and
are hard for humans to write and read. This hinders the adoption of the 
Contract-First approach. This limitation motivated the creation of NeoIDL
language, designed with the aim to be more expressive for humans, besides
providing support to modularization and inheritance. \emph{Problem} None of this
specification languages, including NeoIDL, gives support to strong contracts
as present in languages that supports Design-by-Contract, tipically found in
the  object oriented paradigm. \emph{Objetives} The main objective of this work
is  to investigate the use of Design-by-Contract constructions in the SOC context, 
checking the viability and utility of its adoption at the REST contracts specification
and service implementation. \emph{Results and contributions} This
master thesis contributes technically with the extention of NeoIDL towards
supporting Design-by-Contract, adding to it two types of pre and
post-conditions.
The basic type checks the values of incoming and outgoing atributes. The
service based type makes employes a kind of service composition by
calling another service to check if the main service may be executed (or if it was correctly executed, in case of
post-conditions). By the empirical validation perspective, this thesis
contributes with two studies: the first, verifies the expressiveness and
reusability requirements of NeoIDL, whitin the domain of Command
and Control in colaboration with the Brazilian Army. The second
study focused on the analysis of utility and easy of use perspectives of the 
Design-by-Contract constructions proposed. It gave us interesting answers in
terms of acceptance and easy to use.
