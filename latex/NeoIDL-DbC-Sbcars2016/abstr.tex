\emph{Context}. The demand for integration of heterogeneus systems grows up the
adoptions of solutions based on service oriented computing -- SOC, in special
using Web Services to implement the services, nowadays, with the increasing use
of the REST architectural style. Nevertheless, REST don't has a standard
notations for represent its contracts. Swagger, YAML and WADL only provide the
language to describe services, but both has a relevant limitation: they are made
for computers and they are hard for humans to write and read, what difficults the
recommended SOC Contract-First approach. This limitation motivate the creation of NeoIDL
language, done with the aim to be more expressive for humans, besides
provides support to modularization and inheritance. \emph{Problem} None of this programming languages, including NeoIDL, gives support to strong contracts how the
presents in languages that supports Design-by-Contract, tipically found in the
object oriented paradigm. \emph{Objetives} The main objective of this work is to
investigate the use of Design-by-Contract constructions in the SOC context,
checking the viability and utility of its adoption at the REST contracts
specification and service implementation. \emph{Results and contributions} This
master thesis contributes technically with the extention of NeoIDL to
Design-by-Contract support, adding to it two types of pre and post-conditions.
The basic type checks the values of incoming and outgoing atributes. The
service based type makes a simple service composition by calling another service to
check if the main service may be executed (or if it was correctly executed, in case of
post-conditions). By the empirical validation perspective, this thesis
contributes with two studies: the first, verifies the expressiveness and
reusability requisites of NeoIDL, that was made at the domain of Command and
Control in association with Brazilian Army. The second study focused on the
analysis of utility and easy to use perspectives of the use of 
Design-by-Contract constructions proposed to NeoIDL. It gave us good answers in
terms of acceptance and easy to use.
