\section{Fundamentação}
\label{Fundamentacao}

\subsection{Computação orientada a serviço}

A eficiência na integração entre as soluções de TI é determinante para que se
consiga alterar uma parte sem comprometer todo o ecossistema. A integração
possibilita a combinação de eficiência e flexibilidade de recursos para otimizar
a operação através e além dos limites de uma organização, proporcionando maior
interoperabilidade \cite{papazoglou2008service}.

A computação orientada a serviços -- SOC -- endereça essas necessidades em uma
plataforma que aumenta a flexibilidade e melhora o alinhamento com o negócio, a
fim de reagir rapidamente a mudanças nos requisitos de negócio. Para obter esses
benefícios, contudo, os serviços devem cumprir com determinados quesitos, que
incluem alta autonomia ou baixo acoplamento \cite{erl2008soa}. Assim, o paradigma de SOC
está voltado para o projeto de soluções preparadas para constantes mudanças,
substituindo-se continuamente pequenas peças -- os serviços -- por outras
atualizadas.

Portando, o objetivo da SOC é conceber um estilo de projeto, tecnologia e
processos que permitam às empresas desenvolver, interconectar e manter suas
aplicações e serviços corporativos com eficiência e baixo custo. Embora esses
objetivos não sejam novos, SOC procura superar os esforços prévios como
programação modular, reuso de código e técnicas de desenvolvimento orientadas a
objetos \cite{papazoglou2007serviceApprTechRechIss}.

De modo diferente de arquiteturas convencionais, ditas monolíticas, em que os
sistemas são concebidos agregando continuamente funcionalidades a um mesmo pacote de
\textit{software}, a arquitetura orientada a serviço prega o projeto de pequenas
aplicações distribuídas que podem ser consumidas tanto por usuários finais como
por outros serviços \cite{papazoglou2007serviceApprTechRechIss}. 

A unidade lógica da arquitetura orientada a serviços é exatamente o serviço.
Serviços são pequenos \textit{softwares} que provêem funcionalidades específicas
para serem reutilizadas em várias aplicações. Cada serviço é uma entidade isolada com
dependências limitadas de outros recursos compartilhados
\cite{serrano2014service}. Assim, é formada uma abstração entre os fornecedores
e consumidores dos serviços, por meio de baixo acoplamento, e promovendo a
flexibilidade de mudanças de implementação sem impacto aos consumidores.

\subsection{Princípios SOA}


O paradigma de orientação a serviço é estruturado em oito princípios
fundamentais \cite{erl2009web}, ilustrados na Figura \ref{Fig:PrincipiosSoa}. São eles que
caraterizam a abordagem SOA e a sua aplicação faz com que um serviço se diferencie de um componente ou de
um módulo. Os contratos de serviços permeiam a maior parte destes princípios.

Contrato padronizado - \textbf{Serviços dentro de um mesmo inventário
estão em conformidade com os mesmos padrões de contrato de serviço}.
Os contratos de serviços são elementos fundamentais na arquitetura orientada a
serviço, pois é por meio deles que os serviços interagem uns com os outros e com
potenciais consumidores. Este princípio tem como foco principal o contrato de
serviço e seus requisitos. O padrão de projeto \CtFirst{} é uma
consequência direta deste princípio \cite{erl2009web}.

Baixo acomplamento - \textbf{Os contratos de serviços impõem aos
consumidores do serviço requisitos de baixo acoplamento e são, os próprios
contratos, desacoplados do seu ambiente}. 
Este princípio também possui forte relação com o contratos de serviço, pois a
forma como o contrato é projetado e posicionado na arquitetura é que gerará o
benefício do baixo acoplamento. O projeto deve garantir que o contrato
possua tão somente as informações necessárias para possibilitar a compreensão e
o consumo do serviço, bem como não possuir outras características que gerem
acoplamento.

São considerados negativos, e que devem ser evitados, os acoplamentos  
\begin{enumerate}[(a)] 
\item do contrato com as funcionalidades que ele suporta, agregando ao
contrato características específicas dos processos que o serviço atende;
\item do contrato com a sua implementação, invertendo a estratégia de conceber
primeiramente o contrato;
\item do contrato com a sua lógica interna, expondo aos consumidores
características que levem os consumidores a inadivertidamente aumentarem o
acoplamento;
\item do contrato com a tecnologia do serviço, causando impactos indesejáveis em
caso de substituição de tecnologia.
\end{enumerate}

Por outro lado, há um tipo de acoplamento positivo que é o que gera dependência
da lógica em relação ao contrato \cite{erl2009web}. Ou seja, idealmente a
implementação do serviço deve ser derivada do contrato, pondendo se ter inclusive a geração de código a
partir do contrato.

Abstração - \textbf{Os contratos de serviços devem conter apenas
informações essenciais e as informações sobre os serviços são limitadas àquelas
publicadas em seus contratos}
Reusabilidade - \textbf{Serviços contém e expressam lógica agnóstica e
podem ser disponibilizados como recursos reutilizáveis}
Autonomia- \textbf{Serviços exercem um elevado nível de controle sobre
o seu ambiente em tempo de execução}
Ausência de estado - \textbf{Serviços reduzem o consumo de recursos
restringindo a gestão de estado das informações apenas a quando for necessário}
Descoberta de serviço - \textbf{Serviços devem conter metadados por meio
dos quais os serviços possam ser descobertos e interpretados}
Composição - \textbf{Serviços são participantes efetivos de composição,
independentemente do tamanho ou complexidade da composição}

\subsection{Contract-First}


Duas abordagens podem ser seguidas para se produzir esse efeito. A primeira é a
geração do contrato a partir da lógica implementada, conhecida como
\CdFirst{}. A outra propõe um sentido inverso, partindo-se do contrato
para a geração do código, chamada \CtFirst{}. A abordagem
\CtFirst{} é recomendada para a arquitetura orientada a serviço
\cite{erl2009web}.

A abordagem \CtFirst{} preocupa-se principalmente com a clareza, completude e
estabilidade do contrato para os clientes dos serviços. Toda a
estrutura da informação é definida sem a preocupação sobre restrições ou
características das implementações subjacentes. Do mesmo modo, as
capacidades são definidas para atenderem a funcionalidades a que se destinam,
porém com a preocupação em se promover estabilidade e reuso.

As principais vantagens do \CtFirst{} estão no baixo acoplamento do contrato em
relação a sua implementação, na possibilidade de reuso de esquemas de dados (XML
ou JSON Schema), na simplificação do versionamento e na facilidade de manutenção
\cite{karthikeyancontract}. A desvantagem está justamente na complexidade de
escrita do contrato. Porém, várias ferramentas já foram e vem sendo
desenvolvidas para facilitar essa tarefa.

\subsection{\wss{}}

\ws{} são aplicações modulares e autocontidas que podem ser publicadas,
localizadas a acessadas pela \textit{Web} \cite{alonso2004web}. A diferença
entre o \ws{} e a aplicação \textit{Web} propriamente dita é que o primeiro se
preocupa apenas com o dado gravado ou fornecido, deixando para o cliente a atribuição de apresentar a
informação \cite{serrano2014service}.

SOAP -- \textit{Simple Object Access Protocol} -- é um protocolo padrão W3C que
provê uma definição de como trocar informações estruturadas, por meio de XML, entre
as partes de ambientes descentralizados ou distribuídos \cite{WSDLSite}. SOAP
é um protocolo mais antigo que REST, e foi desenvolvido para troca de informações
pela Internet se utilizando de protocolos como HTTP, SMTP, FTP, sendo o primeiro
o mais comumente utilizado.

Os contratos em SOAP são especificados no padrão WSDL -- \textit{Web Services
Description Language} -- que define uma gramática XML para descrever os serviços
como uma coleção de \textit{endpoints} capazes de atuar na troca de mensagens.
As mensagens e operações são descritas abstratamente na primeira seção do
documento. Uma segunda seção, dita concreta, estabelece o protocolo de rede e o
formato das mensagens.

O termo REST foi criado por Roy Fielding, em sua tese de doutorado
\cite{fielding2000architectural}, para descrever um modelo arquitetural
distribuído de sistemas hipermedia. 

O estilo arquitetural REST é cliente-servidor, em que o cliente envia uma
requisição por um determinado recurso ao servidor e este retorna uma resposta.
Tanto a requisição como a resposta ocorrem por meio da transferência de
representações de recursos \cite{mumbaikar2013web}, que podem ser de vários
formatos, como XML e JSON \cite{serrano2014service}. Toda troca de informações
ocorre por meio do protocolo HTTP, com uma semântica específica para cada
operação

O uso de REST tem se tornado popular por conta de sua flexibilidade e
performance em comparação com SOAP, que precisa envelopar suas informações em
um pacote XML \cite{mumbaikar2013web}, de armazenamento, transmissão e
processamento onerosos.

Ao contrário de SOAP, REST não dispõe de um padrão para especificação de
contratos. Essa carência, que no início não era considerada um problema, foi se
tornando uma necessidade cada vez mais evidente a medida em que se amplia o
conjunto de \wss{} implantados. Atualmente, existem algumas linguagens com o
propósito de documentar o contrato REST.

A linguagem mais popular atualmente é \textit{Swagger} cujo projeto se iniciou
por volta de 2010 para atender a necessidade de um projeto específico, sendo posteriormente
vendida para uma grande empresa. Em janeiro de 2016, \textit{Swagger} foi doada
para o \textit{Open API Iniciative (OAI)} e denominada de \textit{Open API
Specification}. O propósito da iniciativa é tornar \textit{Swagger} padrão para
especificação de APIs com independencia de fornecedor.

\subsection{Design-by-Contract}

\designbycontract{} \cite{meyer1992applying} - DbC - é um conceito
oriundo da orientação a objetos, no qual consumidor e fornecedor firmam entre si garantias para
o uso de métodos ou classes. De um lado o consumidor deve garantir que, antes da
chamada a um método, algumas condições sejam por ele satisfeitas.
Do outro lado o fornecedor deve garantir, se respeitadas suas exigências,
o sucesso da execução.

O mecanismo que expressa essas condições são chamados de asserções
(\textit{assertions}, em inglês). As asserções que o consumidor deve respeitar
para fazer uso da rotina são chamadas de \textbf{precondições}. As asserções que
asseguram, de parte do fornecedor, as garantias ao consumidor, são denominadas
\textbf{pós-condições}.

O conceito chave de \designbycontract{} é ver a relação entre a classe e
seus clientes como uma relação formal, que expressa os direitos e as
obrigações de cada parte \cite{meyer1997object}. Se, por um lado, o
cliente tem a obrigação de respeitar as condições impostas pelo fornecedor para fazer uso do módulo, por
outro, o fornecedor deve garantir que o retorno ocorra como esperado.

De forma indireta, \designbycontract{} estimula um cuidado maior na análise das
condições necessárias para, de forma consistente, se ter o funcionamento correto
da relação de cooperação cliente-fornecedor.
Essas condições são expressas em cada contrato, o qual especifica as obrigações
a que cada parte está condicionada e, em contra-ponto, os benefícios garantidos.

Com o uso de \designbycontract{}, cada rotina é levada a realizar o trabalho
para o qual foi projetada e fazer isso bem: com corretude, eficiência e
genericamente suficiente para ser reusada. Por outro lado, especifica de forma
clara o que a rotina não trata. Esse paradigma é coerente, pois para que a
rotina realize seu trabalho bem, é esperado que se estabeleça bem as
circunstâncias de execução.

Outra característica da aplicação de \designbycontract{} é que o recurso tem sua
lógica concentrada em efetivamente cumprir com sua função principal, deixando
para as precondições o encargo de validar as entradas de dados. Essa abordagem
é o oposto à ideia de programação defensiva, pois vai de encontro à realização
de checagens redundantes. Se os contratos são precisos e explícitos, não há
necessidade de testes redundantes \cite{meyer1992applying}.

Implementações de \designbycontract{}. Eiffel - A linguagem Eiffel foi desenvolvida em meados dos anos 80 por
Bertrand Meyer \cite{meyer1988eiffel} com o objetivo de criar ferramentas que
garantissem mais qualidade aos \textit{softwares}. A ênfase do projeto de
Eiffel foi promover reusabilidade, extensibilidade e compatibilidade.
Características que só fazem sentido se os programas forem corretos e
robustos.

JML - \textit{Java Modeling Language} é uma extensão da linguagem Java
para suporte a especificação comportamental de interfaces, ou seja, controlar o
comportamento de classes em tempo de execução. Para realizar essa função, JML
possui amplo suporte a \designbycontract{}.

Spec\# - é uma extensão da linguagem C\#, à qual agrega o suporte para
distinguir referência de objetos nulos de referência a objetos possivelmente não
nulos, especificações de pré e pós-condições e um método para gerenciar
exceções entre outros recursos \cite{barnett2004spec}.


% An example of a floating figure using the graphicx package.
% Note that \label must occur AFTER (or within) \caption.
% For figures, \caption should occur after the \includegraphics.
% Note that IEEEtran v1.7 and later has special internal code that
% is designed to preserve the operation of \label within \caption
% even when the captionsoff option is in effect. However, because
% of issues like this, it may be the safest practice to put all your
% \label just after \caption rather than within \caption{}.
%
% Reminder: the "draftcls" or "draftclsnofoot", not "draft", class
% option should be used if it is desired that the figures are to be
% displayed while in draft mode.
%
%\begin{figure}[!t]
%\centering
%\includegraphics[width=2.5in]{myfigure}
% where an .eps filename suffix will be assumed under latex, 
% and a .pdf suffix will be assumed for pdflatex; or what has been declared
% via \DeclareGraphicsExtensions.
%\caption{Simulation results for the network.}
%\label{fig_sim}
%\end{figure}

% Note that the IEEE typically puts floats only at the top, even when this
% results in a large percentage of a column being occupied by floats.


% An example of a double column floating figure using two subfigures.
% (The subfig.sty package must be loaded for this to work.)
% The subfigure \label commands are set within each subfloat command,
% and the \label for the overall figure must come after \caption.
% \hfil is used as a separator to get equal spacing.
% Watch out that the combined width of all the subfigures on a 
% line do not exceed the text width or a line break will occur.
%
%\begin{figure*}[!t]
%\centering
%\subfloat[Case I]{\includegraphics[width=2.5in]{box}%
%\label{fig_first_case}}
%\hfil
%\subfloat[Case II]{\includegraphics[width=2.5in]{box}%
%\label{fig_second_case}}
%\caption{Simulation results for the network.}
%\label{fig_sim}
%\end{figure*}
%
% Note that often IEEE papers with subfigures do not employ subfigure
% captions (using the optional argument to \subfloat[]), but instead will
% reference/describe all of them (a), (b), etc., within the main caption.
% Be aware that for subfig.sty to generate the (a), (b), etc., subfigure
% labels, the optional argument to \subfloat must be present. If a
% subcaption is not desired, just leave its contents blank,
% e.g., \subfloat[].


% An example of a floating table. Note that, for IEEE style tables, the
% \caption command should come BEFORE the table and, given that table
% captions serve much like titles, are usually capitalized except for words
% such as a, an, and, as, at, but, by, for, in, nor, of, on, or, the, to
% and up, which are usually not capitalized unless they are the first or
% last word of the caption. Table text will default to \footnotesize as
% the IEEE normally uses this smaller font for tables.
% The \label must come after \caption as always.
%
%\begin{table}[!t]
%% increase table row spacing, adjust to taste
%\renewcommand{\arraystretch}{1.3}
% if using array.sty, it might be a good idea to tweak the value of
% \extrarowheight as needed to properly center the text within the cells
%\caption{An Example of a Table}
%\label{table_example}
%\centering
%% Some packages, such as MDW tools, offer better commands for making tables
%% than the plain LaTeX2e tabular which is used here.
%\begin{tabular}{|c||c|}
%\hline
%One & Two\\
%\hline
%Three & Four\\
%\hline
%\end{tabular}
%\end{table}


% Note that the IEEE does not put floats in the very first column
% - or typically anywhere on the first page for that matter. Also,
% in-text middle ("here") positioning is typically not used, but it
% is allowed and encouraged for Computer Society conferences (but
% not Computer Society journals). Most IEEE journals/conferences use
% top floats exclusively. 
% Note that, LaTeX2e, unlike IEEE journals/conferences, places
% footnotes above bottom floats. This can be corrected via the
% \fnbelowfloat command of the stfloats package.
